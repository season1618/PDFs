\repart{天文学史}

\section{古代の天文学}
	異なる二点から一点を見たときの角度の違いを視差という。地球上の異なる二点からの視差を地心視差という。
\section{天動説と地動説}
	アリスタルコス(B.C.310-230頃)は初めて太陽中心説を唱えたとされる。半月になったときの太陽、地球、月の位置関係から太陽と地球の距離が地球と月の距離の18-20倍であると推測している。数値は大きく違うが方法自体は正しい。天体間の絶対的な距離を測定するには地球上の二点から三角測量する必要がある。ニュートンの時代までは基本的に天体間の相対的な距離のみを用いて計算が行われていた。
	
	周転円と離心円を導入したのは円錐曲線論で有名なアポロ二ウス(B.C.250-200頃)が最初とされる。周転円によって惑星の順行・逆行を説明することが出来た。プトレマイオス(トレミー)は『アルマゲスト』において天動説を含めた当時の天文学を体系的に解説している。出差(月の軌道の摂動・歪み)を発見し、周転円と離心円によって説明した。天動説ではそれぞれの惑星の軌道半径を決定できず任意に拡大縮小できるが、『アルマゲスト』においては月、水星、金星、太陽、木星、土星の順に並べられた。また星の見かけの明るさ(相対等級)を目に見える一番明るい星と一番暗い星の間の6段階に分けている。
	『天体の回転について』の形式はプトレマイオスの『アルマゲスト』を踏襲している。
	地動説では周転円の数が半分に減った。しかし地動説の証拠である年周視差は当時の技術では観測できず、1838年のベッセルの観測まで待たねばならなかった。
\section{暦}
	ヒッパルコス(B.C.190-125頃)はロードス島に天文台を建設し、地球の歳差運動を発見した。地軸は黄道面に対して傾いており、トルクが発生し赤道が移動する。これを赤道の歳差という。また他の惑星の引力の影響から黄道が傾く。これを黄道の歳差という。これらを合わせたものを一般歳差という。黄道の歳差は赤道の歳差に比べて無視できるほど小さい。歳差によって地軸は公転と逆方向に25772年かけて一周することになる。これにより太陽年(回帰年)365.2422日は恒星年365.2564日より20分24秒ほど短い。春分点や北極星はは徐々に移動する。
	太陽暦には太陽年が用いられる。
\section{観測技術の発展}
	ニュートンが万有引力の考えに到達したのは1665年だが、当時は天体の大きさや天体間の距離について正確は値は知られていなかった。ジャン・ピカール(1620-1682)は1669年に三角測量を行い、緯度一度の子午線弧を誤差約0.3\%で測量した(1615年のスネルの観測では誤差3\%)。ニュートンはこの観測結果に基づき、月の自由落下の加速度や地上における重力加速度を計算し直した。
	1675年、ジョン・フラムスティード(1646-1720)の勧告により、後に経度の基準となるグリニッジ天文台が建設された。初代天文台長兼王室天文官にフラムスティードが就任した。フラムスティードは観測機器を改良し、精度を1秒にまで引き上げた。『プリンキピア』出版にあたってフラムスティードは月や彗星の観測データをニュートンに渡している。月や惑星の位置の他にフラムスティードは恒星の赤経と北極距離、黄経・黄緯を載せた目録を作っている。当時は大航海時代であり、海上で船の位置を決定する技術が必要とされていた。緯度を知るには星の高度を観測すれば良く(例えば北極星の高度がそのまま高度になる)、経度を割り出すには星の位置と時刻を知る必要があった。
	エドモンド・ハレー(1656-1742)は1676年にセントヘレナ島を訪れ南半球から見える星の詳細な観測を行った。ハレーは過去の彗星の記録を整理し、その中のいくつかが同一のものであることを示し、周期を75,6年と予測した。この彗星の次の回帰年を予測し実際に観測されたため、ハレー彗星と呼ばれることになった。また金星の太陽面通過を利用して太陽視差を正確に測定できることを指摘している。ハレーは1720年に二代目王室天文官となっている。
	1675年にはレーマーが、木星の衛星の合と衝の時間差が光速が有限であることに起因するとして、光速を計算した。地動説が予測する恒星の年周視差は当時まだ観測されていなかった。三代目王室天文官であるジェームズ・ブラッドレー(1693-1762)は年周視差観測の試みの中で1728年に光行差を、1748年に章動(歳差の微小成分)を発見した(年周視差の観測には至らなかった)。光行差は光速が有限であることと地球の公転運動を要請している。光行差の発見は思いがけず地動説の証明となった。古代に発見された歳差や章動の問題は翌年の1749年にダランベールによって解かれた。
	ニュートンはまた回転体の形状について論じている。当時地球の形状が赤道方向に膨らんでいるのか南北方向に膨らんでいるのかについて、激しい論争が起こっていた。1735年のフランス科学アカデミーによる測地遠征の結果から、地球は赤道方向に膨らんでいる回転楕円体であることが立証された。
\section{新天体の発見}
	ガリレオは月表面のクレーター、太陽黒点、木星のガリレオ衛星を発見。
	クリスティアーン・ホイヘンスは土星の環と衛星タイタンを発見。
	ジョヴァンニ・カッシーニは土星の四つの衛星とカッシーニの間隙を発見。
	エドモンド・ハレーがハレー彗星
	1781年、ウィリアム・ハーシェルが天王星を発見。1783年からは掃天観測を開始し、2500個もの星雲と星団のカタログを作成した。息子のジョン・ハーシェル(1792-1871)も南天の約2500個の星雲・星団のカタログを作成している。19世紀後半には星雲が次々に発見された。
	1801年、ジュゼッペ・ピアッツィが火星と木星の間にケレスを発見した。ケレスはその後見失われてしまったが、ガウスの計算に基づいてハインリヒ・オルバースによって再発見された。ほぼ同じ公転軌道上に小さい惑星が立て続けに発見された。ハーシェルはこれらをAsteroid(小惑星)と名付け、以後現在までに数千個の小惑星が見つかっている。
	1846年、ユルヴァン・ルヴェリエが海王星を予測。その後別の天文学者によって観測された。マクスウェルは土星の環が固体や流体ではなく、無数の粒子からなっていることを示した。

	1660年にロンドンに王立協会が、1666年にパリ王立科学アカデミーが設立された。
	1814年、フラウンホーファーが太陽スペクトル中に多数の暗線を発見した。キルヒホッフは高温に熱せられた物体の放射率と吸収率が等しいことを発見した(キルヒホッフの法則)。ステファンは物体から放射される全波長におけるエネルギーの総体が絶対温度の4乗に比例することを発見し、ボルツマンが理論的に導出した(ステファン・ボルツマンの法則)。太陽大気の熱平衡を仮定すると、表面温度が約6000Kであることが求められた。光のドップラー効果が発見されている。1838年に銀板写真術(ダゲレオタイプ)が発明されると、翌年にはパリ天文台で日食の撮影に利用されている。その後太陽黒点の観測にも用いられている。ハギンス。1868年の日食では、太陽大気中のプロミネンスとコロナのスペクトルが初めて観測された。プロミネンスの主成分が水素であることが判明した。また水素とは別の輝線も見つかり、未知の元素のものと同定され、ヘリウムと命名された。コロナのスペクトル中の輝線を発生する元素はコロニウムと命名されたが、後に電離した鉄イオンであることが判明した。スペクトルから恒星を分類する試みが始まった。また恒星が進化するという発想が生まれた。
	サムソンやカール・シュヴァルツシルトは輻射平衡の理論を研究し、アーサー・エディントンは恒星の内部構造を解明した。チャンドラセカール。サハの電離公式。ポグソンは天体の等級を対数スケールで定義した。