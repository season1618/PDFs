\section{三体問題}
	スウェーデンの数学者ヨースタ・ミッタク=レフラーはノルウェーとスウェーデンの国王オスカー2世に、n体問題の大域的な解を最初に得た者に賞とメダルを授与するよう説得した。審査委員会はカール・ワイエルシュトラス、シャルル・エルミート、そしてミッタク=レフラーにより構成された。委員会はn体問題を含む4つの問題を提出した。1883年頃、ミッタク=レフラーが創刊したActa Mathematica誌第7巻に掲載された。問題の記述は次のようになっている。
	\begin{quotation}
		互いにニュートンの法則によって引力を及ぼし合う任意の数の質点の系が与えられているとし、どの二つの質点も衝突しないと仮定する。このとき各質点の座標を時間のある既知の関数を変数として、この変数の全ての値に対して一様収束するような級数による表現を見つけること。
		\hfill 『天体力学のパイオニアたち』
	\end{quotation}
	懸賞論文提出の締め切りは1888年6月1日、懸賞金は2500クラウンだった。アンリ・ポアンカレは1888年5月17日に論文の大筋となる原稿を提出した。この原稿でポアンカレは制限三体問題に関するある種の安定性を主張しているが、後に誤りに気付いた。1889年1月21日に受賞した。ポアンカレは1890年に「三体問題と力学の方程式について」を発表し、1892年には『天体力学の新しい方法』を出版している。
	\subsection{潮汐力}
	\subsection{制限三体問題}
	ラグランジュ点
\footnote{
	ルジャンドル多項式は1782年、アドリアン=マリ・ルジャンドル(1752-1833)がニュートンポテンシャルの展開係数として導入した。
}