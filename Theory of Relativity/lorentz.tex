\section{ローレンツ変換}

\subsection{導出}
    マクスウェル方程式にガリレイ変換を適用すると形式が変わってしまう。そこで新たに相対性原理と光速度不変の原理を満たすような慣性系間の変換を考えなければならない。これをローレンツ変換という。\\
    二つの慣性系$K,K'$を考える。時刻$t=t'=0$のとき両者の座標は一致していたとして、$K'$系は$K$系に対して$x$軸方向に速さ$v$で進んでいるとする。また開始と同時に原点から光が出発したとき、両方の系で光速は等しいので、
    \begin{align*}
        x^2+y^2+z^2 = (ct)^2\\
        x'^2+y'^2+z'^2 = (ct')^2
    \end{align*}
    となる。つまりローレンツ変換は$s^2=(ct)^2-x^2-y^2-z^2$を不変に保つ。一方で速度$v$の符号を反転させれば元の逆変換になっているはずである。二つの慣性系は互いに等速直線運動をしているので、二次以上の項があってはならない。よって、
    \begin{align*}
        ct' = a_{00}ct + a_{01}x + a_{02}y + a_{03}z\\
        x' = a_{10}ct + a_{11}x + a_{12}y + a_{13}z\\
        y' = a_{20}ct + a_{21}x + a_{22}y + a_{23}z\\
        z' = a_{30}ct + a_{31}x + a_{32}y + a_{33}z
    \end{align*}
    である。ところで相対論では空間座標と次元を合わせるために時間成分を$ct$としている。こちらの方がローレンツ変換などで4つの成分の対応が分かりやすく記述できるのである。$(x',y',z')=(0,0,0)の時、(x,y,z)=(vt,0,0)$なので、
    \begin{align*}
        a_{10}c+a_{11}v = 0\\
        a_{20}c+a_{21}v = 0\\
        a_{30}c+a_{31}v = 0
    \end{align*}
    同様に、$(x,y,z)=(0,0,0)の時、(x',y',z')=(-vt',0,0)$なので、
    \begin{align*}
        \frac{a_{10}}{a_{00}} &= -\frac{v}{c}\\
        \frac{a_{20}}{a_{00}} &= 0\\
        \frac{a_{30}}{a_{00}} &= 0
    \end{align*}
    従って$a_{20}=a_{21}=a_{30}=a_{31}=0$である。\\$y,z$軸は$x$軸の周りに回転させても形式は変わらないので、自明に$a_{02}=a_{03}=a_{12}=a_{13}=0$である。また第3,4式についても適用する。$y,z$軸は対称なので、$a_{22}=a_{33}=p,a_{23}=a_{32}=q$である。よって、
    \begin{align*}
        \pmat{\co & -\si\\ \si & \co}\pmat{p&q\\q&p}
        &= \pmat{p&q\\q&p}\pmat{\co & -\si\\ \si & \co}\\
        \pmat{p\co-q\si&-p\si+q\co\\p\si+q\co&p\co+q\si}
        &= \pmat{p\co+q\si&-p\si+q\co\\p\si+q\co&p\co-q\si}
    \end{align*}
    この式は任意の$y,z及び\theta$で成立するので$q=0$。逆変換を考えれば$p^2=1$。物理的に妥当なのは$p=1$である。\\開始と同時に原点から$x$軸方向に向かって光が放たれたとすると、
    $(x,y,z)=(ct,0,0)の時、(x',y',z')=(ct',0,0)$なので、
        \[ct' = (a_{00}+a_{01})ct = (a_{10}+a_{11})ct\]
    同様に$(x,y,z)=(-ct,0,0)の時、(x',y',z')=(-ct',0,0)$なので、
        \[ct' = (a_{00}-a_{01})ct = (-a_{10}+a_{11})ct\]
    つまり$a_{00}=a_{11}=a,a_{01}=a_{10}=b$である。速度$v$の符号を反転させれば逆変換になる。しかし係数にどのような形で$v$が含まれているか分からないので、代わりに$x$軸を反転させる。
    \begin{align*}
        ct' &= act-b\cdot -x\\
        -x' &= -bct + a\cdot -x
    \end{align*}
    よって、
    \[
        \pmat{a&b\\b&a}\pmat{a&-b\\-b&a}
        =\pmat{a^2-b^2&0\\0&a^2-b^2}=\pmat{1&0\\0&1}
    \]
    なので$a^2-b^2=1$である。$b/a=-v/cなので、a=\dfrac{1}{\sqrt{1-(v/c)^2}}
    ,b=-\dfrac{v/c}{\sqrt{1-(v/c)^2}}$となる。\\
    つまりローレンツ変換は次のようになる。
    \begin{align*}
        ct' &= \frac{1}{\sqrt{1-(v/c)^2}}ct-\frac{v/c}{\sqrt{1-(v/c)^2}}x\\
        x' &= -\frac{v/c}{\sqrt{1-(v/c)^2}}ct+\frac{1}{\sqrt{1-(v/c)^2}}x\\
        y' &= y\\
        z' &= z
    \end{align*}
    $v/c\ll 1$の極限ではガリレイ変換に帰着することが容易にわかる。また$\gamma=\rec{\sqrt{1-(v/c)^2}}$はローレンツ因子と呼ばれ、相対論で良く現れる。\\
    相対性原理とは全ての慣性系で物理法則が不変であるというものだった。特に物理法則はローレンツ変換に対して不変でなければならない。これを特殊相対性原理と呼ぶ。

\subsection{時間の遅れとローレンツ短縮}
    $x'=0と置けばx=vt$を得る。これを$t$について解けば、
        \[t = \frac{t'}{\sqrt{1-(v/c)^2}}\]
    つまり動いている物体は時間が遅れることが分かる。また$K系とK'系$の原点の間の距離を$l,l'と置くと、l=x=vt$。また$x=0と置けばx'=-\gamma vtよりl'=-x'=\gamma vt$よって、
        \[l=\sqrt{1-\frac{v^2}{c^2}}l'\]
    つまり動いている物体はその長さが進行方向に対して縮むことが分かる。これをローレンツ収縮という。

\subsection{速度の合成}
    $K'$系が$K$系に対して$x$軸方向に速度$v_1$で進んでいて、$K'$系で質点が$(x',y') = (v_2t'\co,v_2t'\si)$で移動しているとする。これをローレンツ変換して、
    \begin{align*}
        ct &= \ga ct'+\ga v_1/c\cdot v_2t'\co\\
        x &= \ga v_1/c\cdot ct'+\ga v_2t'\co\\
        y &= v_2t'\si\\
        \intertext{よって合成速度$V$は、}
        V_x &= \frac{x}{t} = \frac{v_1+v_2\co}{1+\dfrac{v_1v_2\co}{c^2}}\\
        V_y &= \frac{y}{t} = \rec{\ga}\frac{v_2\si}{1+\dfrac{v_1v_2\co}{c^2}}\\
        \intertext{$u_1=v_1/c,u_2=v_2/c,U=V/c$と置くと$U$の絶対値は}
        V^2 &= V_x^2+V_y^2 = \frac{(v_1+v_2\co)^2+(1-v_1^2/c^2)v_2^2\si[2]}{\lr{1+\dfrac{v_1v_2\co}{c^2}}^2}\\
        U^2 &= \frac{(u_1+u_2\co)^2+(1-u_1^2)u_2^2\si[2]}{(1+u_1u_2\co)^2}\\
        &= \frac{u_1^2+u_2^2+2u_1u_2\co-u_1^2u_2^2\si[2]}{(1+u_1u_2\co)^2}\\
        \intertext{3点が必ず同一平面上にあることに注意すると、合成速度はベクトルで書くことができて、}
        U^2 &= \frac{u_1^2+u_2^2+2u_1\cdot u_2-|u_1\times u_2|^2}{(1+u_1\cdot u_2)^2}\\
    \end{align*}
    となる。これを変形すると、
    \begin{align*}
        U^2 &= \frac{-1+u_1^2+u_2^2-u_1^2u_2^2+1+2u_1\cdot u_2+(u_1\cdot u_2)^2}
        {(1+u_1\cdot u_2)^2}\\
        &= \frac{(1+u_1\cdot u_2)^2-(1-u_1^2)(1-u_2^2)}
        {(1+u_1\cdot u_2)^2}\\
        1-U^2 &= \frac{(1-u_1^2)(1-u_2^2)}{(1+u_1\cdot u_2)^2}
    \end{align*}
    となるので$u_1,u_2\leq 1ならU\leq 1$より合成速度が光速を超えないことが示せる。

\subsection{光行差}
    光行差とは、移動している観測者が天体を見るとき、天体が移動方向にずれて見える現象またはそのずれを指す。垂直に降っている雨を電車の中から見ると斜めに降っているように見えるのと同じである。ただ、相対論的効果を考慮しなければならない。観測者が速度$v$で移動しており、その進行方向に対して角$\theta$にある天体の光行差を$a$とする。$K$系では時刻$t$における光の位置が$(-ct\co,-ct\si)$であるとする。これをローレンツ変換して、
    \begin{align*}
        ct' &= \ga ct-\ga v/c\cdot (-ct\co)\\
        x' &= -\ga v/c ct+\ga (-ct\co)\\
        y' &= y = -ct\si\\
        \intertext{従って}
        \rec{\tan(\theta-a)} &= \frac{x'}{y'} = \frac{-\ga vt-\ga ct\co)}{-ct\si}\\
        &= \rec{\sqrt{1-(v/c)^2}}\lr{\frac{v}{c\si}+\rec{\tan\theta}}\\
    \end{align*}
    となる。

\subsection{光のドップラー効果}
    $K$系から見て光源は原点にあり、光は$(ct\co,ct\si)$にあるとする。これをローレンツ変換して、
    \begin{align*}
        ct' &= \ga ct-\ga v/c\cdot ct\co\\
        x' &= -\ga v/c\cdot ct+\ga ct\co\\
        y' &= y = ct\si
    \end{align*}
    よって光の進む経路の比は、
    \begin{align*}
        \frac{\sqrt{x'^2+y'^2}}{ct} &= \llr{
        (-\ga\frac{v}{c}+\ga\co)^2+\si[2]}^{1/2}\\
        &= \ga\llr{\frac{v^2}{c^2}-\frac{2v}{c}\co+\co[2]+\si[2]\lr{1-\frac{v^2}{c^2}}}^{1/2}\\
        &= \ga\llr{1-\frac{2v}{c}\co+\frac{v^2}{c^2}\co[2]}^{1/2}\\
        &= \ga\lr{1-\frac{v}{c}\co}
    \end{align*}
    となる。二つの慣性系で光の振動する回数は等しいので、
    \begin{gather*}
        \lambda' = \frac{1-(v/c)\co}{\sqrt{1-(v/c)^2}}\lambda\\
        \nu' = \frac{\sqrt{1-(v/c)^2}}{1-(v/c)\co}\nu
    \end{gather*}
    である。特に$\theta=0^{\circ}$の時は、
        \[\lambda' = \frac{1-(v/c)}{\sqrt{1-(v/c)^2}}\lambda\]
    $v>0$の時光源に近付いていて、$v<0$の時に光源から遠ざかっていることに注意すると、近付くときは波長が縮んで青っぽく見え(青方偏移)、遠ざかるときは波長が伸びて赤っぽく見える(赤方偏移)。また$\theta=90^{\circ}$の時は、
        \[\lambda' = \rec{\sqrt{1-(v/c)^2}}\lambda\]
    となって波長が伸びる。これを横ドップラー効果という。