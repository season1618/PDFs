\section{四次元の時空}

\subsection{ミンコフスキー時空}
    ローレンツ変換ではガリレイ変換と違い、空間だけでなく時間も一体となって変換される。そこで相対論では$x,y,z$に時間成分である$ct$を加えた4次元空間を考える。これをミンコフスキー時空という。4つを添え字で表すために$(x^0,x^1,x^2,x^3) = (ct,x,y,z)$とする。相対論では出来事を事象(event)と呼ぶ。事象はミンコフスキー時空で点として表され、世界点という。また質点の運動はこの時空で曲線となり、世界線という。二つの世界点$(ct_1,x_1,y_1,z_1),(ct_2,x_2,y_2,z_2)$に対して
        \[s_{12}^2 = -(ct_2 - ct_1)^2 + (x_2 - x_1)^2 + (y_2 - y_1)^2 + (z_2 - z_1)^2\]
    を世界間隔という。特に原点との世界間隔
        \[s^2 = -(ct)^2 + x^2 + y^2 + z^2\]
    を世界長さという。微小長さはミンコフスキー計量
        \[\eta_{ij} = \pmat{
            -1 & 0 & 0 & 0\\
            0 & 1 & 0 & 0\\
            0 & 0 & 1 & 0\\
            0 & 0 & 0 & 1
        }\]
    を用いると
        \[ds^2 = \eta_{ij}dx^idx^j\]
    となる。ミンコフスキー時空の原点を通過する光(つまり時刻$t = 0$で観測者から遠ざかる光)は、時間軸から45度傾いた円錐面(実際は面ではない)
        \[s^2 = (ct)^2 - x^2 - y^2 - z^2 = 0\]
    の上を伝播する。これを光円錐と呼ぶ。光円錐の内側$(s^2 < 0)$では、原点での事象と因果関係を持つことができ、時間的(time like)領域と呼ばれる。それに対して外側$(s^2 > 0)$は空間的(space like)領域と呼ばれる。

    世界長さ$s$はローレンツ変換によって変化しない不変量である。ローレンツ変換はミンコフスキー時空内でのある種の回転を表していると考えられる。実際$v/c = \tanh\theta$とおくと
    \begin{align*}
        ct' &= \cosh\theta ct - \sinh\theta x\\
        x' &= -\sinh\theta ct + \cosh\theta x\\
    \end{align*}
    である。

\subsection{テンソル解析}
    単位行列はクロネッカーのデルタ$\delta^i_j$で表すことができる。
    \begin{align*}
        \delta'^i_j &= \pd[x'^i]{x'^j}\\
        &= \pd[x'^i]{x^a}\pd[x^a]{x'^j}\delta^a_a\\
        &= \pd[x'^i]{x^a}\pd[x^b]{x'^j}\delta^a_b
    \end{align*}
    と書けるので混合テンソルである。

    時空の計量は、
        \[ds^2 = g_{kl}dx^kdx^l = g'_{ij}dx'^idx'^j\]
    より、
        \[g'_{ij} = g_{kl}\pd[x^k]{x'^i}\pd[x^l]{x'^j}\]
    なので二階の共変テンソルである。そこでこれを計量テンソルという。計量テンソルの逆行列は$g^{ik}g_{kj}=\delta^i_j$と書けるので、二階の反変テンソルである。\\
    三次元の微分演算子$\nabla$を四次元に拡張する。微分演算子はそれ自身では共変ベクトルだが、テンソルに作用するときは一般にテンソルとはならない。しかし座標変換が一次変換のみを許すならば、テンソルと同じように振る舞う。特にローレンツ変換では形式を変えない。
        \[(\pa_0,\pa_1,\pa_2,\pa_3) = \lr{\rec{c}\pd{t},\pd{x},\pd{y},\pd{z}}\]
    とする。これとミンコフスキー計量との縮約をとると、
        \[(\pa^0,\pa^1,\pa^2,\pa^3) = \lr{-\rec{c}\pd{t},\pd{x},\pd{y},\pd{z}}\]
    となる。更にこれらの縮約をとり、
        \[\pa^i\pa_i = -\rec{c^2}\pd[^2]{t^2} + \pd[^2]{x^2}+\pd[^2]{y^2}+\pd[^2]{z^2}\]
    これはマクスウェル方程式などに出てくるダランベール演算子である。

\subsection{固有時}
    注目している物体に対して静止した座標系で測った時間を固有時(固有時間)$\tau$という。固有時はその静止座標系における通常の時間である。$K'$系においてその物体が原点にあるとすれば、$(ct)^2-x^2-y^2-z^2=(c\tau)^2$なので、$\tau = -s/c$。固有時は慣性系では世界長さと同様ローレンツ変換で変わらない不変量である。これを慣性系だけでなく、一般の加速する座標系でも不変となるように微小量で考え、
    \begin{align*}
        d\tau^2 &= dt^2-\frac{dx^2+dy^2+dz^2}{c^2}\\
        \lr{\di[\tau]{t}}^2 &= 1-\rec{c^2}\llr{\lr{\di[x]{t}}^2+\lr{\di[y]{t}}^2+\lr{\di[z]{t}}^2}\\
        &= 1-\frac{v^2}{c^2}\\
        t &= \int dt = \int \rec{\sqrt{1-(v/c)^2}}d\tau
    \end{align*}
    となる。

\section{四元ベクトル}
    ある物体を異なる慣性系で見たときに速度がどのように変わるか考える。速度の合成則より、
        \[v'_x = \frac{-v + v_x}{1 - vv_x/c^2}\]
    となる。しかし各々の物理量についてこのような変換があったのでは、式の意味を捉え難くなってしまうであろう。つまり速度もローレンツ変換と同じような変更を受けるようにしたいのである。アインシュタインが相対論を作り上げる上で土台とした相対性原理は、いかなる慣性系でも物理法則の形は変わらないというものであった。従ってローレンツ共変性を持つことが一目見て判るような形式にしたいのである。このようなものを共変形式(covariant form)と呼ぶ。

\subsection{四元速度}
    四元速度を$u^i=\di[x^i]{\tau}$と定義する。座標を不変量である固有時で微分しているので確かに座標と同じ座標変換を受ける。ここで速度の時間座標などという量が出てきているが、座標と同じように都合が良いので導入することにする。本来の速度との関係を導いてやると、
    \begin{align*}
        u^0 = \di[ct]{t/\ga} = \ga c\\
        u^1 = \di[x^i]{t\ga} = \ga v^i(i=1,2,3)
    \end{align*}
    となる。よって$u^i=\ga(c,v)$である。またこれ以降は$v^i=(c,v_x,v_y,v_z)$と表すことにする。

\subsection{四元運動量}
    同様に四元運動量を導入する。四元運動量は四元速度に質量を掛けたものである。つまり$p^i=mu^i=\ga mv^i$である。ここで我々は一旦、運動量保存則を見直さなければならない。古典的な運動量の定義である$mv$では物体が光速を超えることを許している。一方$\ga mv$では$v=c$で運動量が発散するようになっている。この点を踏まえると、後者が本来の運動量であり、保存しているのもこちらではないかと考えるのが自然だろう。実際、実験では四元運動量の方が保存していることが確かめられている。そこで新しく四元運動量を本来の運動量として定義し直すことにする。

    ここで$p^0$について考察してみる。
        \[(mc)^2 = (p^0)^2-p_x^2-p_y^2-p_z^2 = (p^0)^2-p^2\]
    なので、
    \begin{align*}
        p^0c = \sqrt{(mc^2)^2 + (pc)^2}\\
        = mc^2\sqrt{1 + \frac{p^2}{(mc)^2}}\\
        \intertext{テイラー展開して、}
        = mc^2 + \frac{p^2}{2m} + \cdots\\
    \end{align*}
    第二項はニュートン力学における運動エネルギーに似ている。そこで運動量の場合と同じように、$p^0$は物体の全エネルギー$E$を$c$で割ったものだと解釈する。特に運動量が0の時は$E = mc^2$となり、有名な公式が導かれる。一般的には、
        \[E^2 = (mc^2)^2 + (pc)^2\]
    或いは$p=\ga mv$より、
        \[1 + \frac{p^2}{(mc)^2} = 1+\ga^2\frac{v^2}{c^2} = \ga^2\]
    なので、
        \[E = \ga mc^2\]
    である。前者の式は$m = 0$とした時に、電磁波の運動量とエネルギーの関係式に一致する。このことから光の質量は0であると考えられるようになった。エネルギーとはもはやスカラーではなく、四元運動量としてベクトルの一成分に取り込まれたのだ。
    \paragraph{質量の増加}
        相対論では、運動量は$p=\ga mv$でエネルギーは$E=\ga mc^2$と表される。これらの式を眺めてみると、物体が運動するときはあたかも質量が$\ga$倍に増えていると解釈できる。しかし力などの他の物理量を考えてみると$\ga$倍になっていないこともある。また運動している物体の重力が増えるわけではないので、少なくとも重力質量は変化していない。よって運動している物体の(慣性)質量が増えているのではなく、運動量は速度に比例するよりも早く増加すると考える方が無難である。