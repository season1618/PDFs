\documentclass{jsarticle}
\usepackage{amssymb,amsmath}
\usepackage{mathrsfs,ascmac}
\begin{document}
\title{相対性理論}
\author{高校n年 season}
\date{2019/05/19}
\maketitle
\tableofcontents
\newcommand{\repart}[1]{\part{#1}\setcounter{section}{0}}
\newcommand{\lr}[1]{\left(#1 \right)}
\newcommand{\mlr}[1]{\left\{#1 \right\}}
\newcommand{\llr}[1]{\left[#1 \right]}
\newcommand{\rec}[1]{\frac{1}{#1}}
\newcommand{\di}[2][]{\frac{d #1}{d #2}}
\newcommand{\dd}[2][]{\frac{d^2 #1}{d #2^2}}
\newcommand{\pd}[2][]{\frac{\partial #1}{\partial #2}}
\newcommand{\ppd}[3]{\frac{\partial^2 #1}{\partial #2\partial #3}}
\newcommand{\pmat}[1]{\begin{pmatrix} #1 \end{pmatrix}}
\newcommand{\co}[1][]{\cos^{#1}\theta}
\newcommand{\si}[1][]{\sin^{#1}\theta}
\newcommand{\na}{\nabla}
\newcommand{\ga}{\gamma}
\newcommand{\Ga}{\Gamma}
\newcommand{\grad}{\mathrm{grad}}
\newcommand{\dive}{\mathrm{div}}
\newcommand{\rot}{\mathrm{rot}}
\newcommand{\pa}{\partial}
\newcommand{\chr}[2]{\left\{#1 \atop #2\right\}}

\input section12.tex % ローレンツ変換の導出 ローレンツ変換の帰結
\input section34.tex % 四次元の時空 四元ベクトル
\input section5.tex % 物理法則の共変形式
\input section6.tex % リーマン幾何学
\input section7.tex % 時空の方程式

\end{document}