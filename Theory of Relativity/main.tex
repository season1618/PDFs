\documentclass{jsarticle}
\usepackage{amssymb,amsmath}
\usepackage{mathrsfs,ascmac}

\newcommand{\newpart}[1]{\part{#1}\setcounter{section}{0}}
\newcommand{\lr}[1]{\left(#1 \right)}
\newcommand{\mlr}[1]{\left\{#1 \right\}}
\newcommand{\llr}[1]{\left[#1 \right]}
\newcommand{\rec}[1]{\frac{1}{#1}}
\newcommand{\di}[2][]{\frac{d #1}{d #2}}
\newcommand{\dd}[2][]{\frac{d^2 #1}{d #2^2}}
\newcommand{\pd}[2][]{\frac{\partial #1}{\partial #2}}
\newcommand{\ppd}[3]{\frac{\partial^2 #1}{\partial #2\partial #3}}
\newcommand{\pmat}[1]{\begin{pmatrix} #1 \end{pmatrix}}
\newcommand{\co}[1][]{\cos^{#1}\theta}
\newcommand{\si}[1][]{\sin^{#1}\theta}
\newcommand{\na}{\nabla}
\newcommand{\ga}{\gamma}
\newcommand{\Ga}{\Gamma}
\newcommand{\grad}{\mathrm{grad}}
\newcommand{\dive}{\mathrm{div}}
\newcommand{\rot}{\mathrm{rot}}
\newcommand{\pa}{\partial}
\newcommand{\chr}[2]{\left\{#1 \atop #2\right\}}

\title{相対性理論}
\author{season07001674}
\date{2019/05/19}

\begin{document}
\maketitle
\tableofcontents

\newpart{特殊相対性理論}

電磁気学のマクスウェル方程式に登場する光速度はどの慣性系を基準に測ったものなのか分からなかった。マクスウェル方程式はガリレイ変換に対して不変ではなかったため、ヘルツは、オリジナルのマクスウェル方程式はエーテルに対して静止した慣性系で成立すると解釈し、ガリレイ変換によって変形したマクスウェル方程式を提出した。これは慣性系によって光の速さは変わることを意味する。しかしマイケルソンとモーレーは、エーテル中を運動している地球の速度を検出しようとして失敗し、光速はどの方向でも常に一定であるという結論を得た。エーテルが地球に引きずられているということも考えられるが、光速は木星の食を利用しても測られているので否定された。このように力学と電磁気学の矛盾が判明したため、少なくともどちらか一方が修正されなければならないことが分かった。アインシュタイン以前の人々が考えた理論の中には、ローレンツ変換や局所時間という概念を提案するなど、数式上は相対論と一致したものもあった。しかしそれらには物理的解釈に不明な点が残り、満足のいくものではなかった。そこでアインシュタインがとった解決法は、次の二つの原理を仮定することであった。
\begin{description}
    \item[相対性原理] 全ての慣性系で物理法則は同じ
    \item[光速度不変の原理] 光源に依らず光速度は一定
\end{description}
これらに加えて$v/c \ll 1$の極限でニュートン力学に一致することが要請される。また二つの原理から直ちに「任意の慣性系、任意の光源に対し、光速は一定」ということが導かれる。

\input section12.tex % ローレンツ変換の導出 ローレンツ変換の帰結
\input section34.tex % 四次元の時空 四元ベクトル
\input section5.tex % 物理法則の共変形式

\newpart{一般相対性理論}

特殊相対論では加速する座標系と重力場を表現できなかった。アインシュタインは一般相対性理論を構築するにあたって次のような仮定を設けた。
\begin{itemize}
    \item 一般相対性原理:物理法則は全ての座標系で同じ
    \item 一般共変性原理:物理法則は全ての座標系で同じ形式でなければならない
    \item 等価原理:任意の世界点で、局所的に慣性系となるような座標系を選ぶことができる
    \item 局所座標系における特殊相対論の成立
    \item 測地線の仮定:自由質点は測地線を描く
\end{itemize}
特殊相対性原理が全ての慣性系で物理法則が同じことを仮定していたのに対し、一般相対性原理ではあらゆる座標系で物理法則が同じであることを主張している。一般共変性原理は数学的には、物理法則はテンソル方程式と共変微分で表現できるということである。これらの仮定は、物理的な要請よりも理論の形式を優先している。

\input section6.tex % リーマン幾何学
\input section7.tex % 時空の方程式

\end{document}