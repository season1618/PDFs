\section{時空の方程式}

\subsection{一般座標系への拡張}
    一般相対性原理によれば、物理法則は全ての座標系において不変である。そして等価原理により、重力場の作用する時空でも適当な座標変換によって、局所的に慣性系に変換することができる。さらに、ミンコフスキー計量で接続係数は0なので、共変微分は通常の微分に一致する。したがって、特殊相対論の数式を一般相対論の数式に拡張するには、通常の微分を共変微分に置き換えるだけで良い。

\subsection{測地線の方程式}
    特殊相対論における自由粒子のラグランジアンは
        \[L = -mc\sqrt{-\eta_{\mu\nu}\di[x^\mu]{\tau}\di[x^\nu]{\tau}}\]
    であった。これの自然な拡張はミンコフスキー計量$\eta_{\mu\nu}$を一般の計量$g_{\mu\nu}$で置き換えることである。ここで作用
        \[S = -mc\int \sqrt{-g_{\mu\nu}u^\mu u^\nu}d\tau\]
    の極値を与える軌道は、計量$-g_{\mu\nu}$の導入された四次元空間における測地線を表している。つまり物質の運動方程式は、$-g_{\mu\nu}$のクリストッフェル記号が$g_{\mu\nu}$と同じなので、
        \[\dd[x^i]{\tau}+\Ga^i_{\mu\nu}\di[x^\mu]{\tau}\di[x^\nu]{\tau} = 0\ (i=0,1,2,3)\]
    となる。

\subsection{アインシュタイン方程式}
    前章で物質の運動方程式を求めたが、軌道を求めるには時空の計量を定める必要がある。一般相対性理論における重力場の方程式は次の条件の下でニュートン力学に一致するはずである。
    \begin{itemize}
        \item 弱い重力場
        \item 重力場は時間変化しない
        \item 質点の速度は光速に比べて十分遅い
    \end{itemize}
    まず加速度系における計量を求める。慣性系$K$に対して加速度系$K'$が$x$軸方向に加速度$g$で進んでいるとする。時刻$t=0$で両者が一致しているとすれば、十分小さい$t$について、
    \begin{align*}
        t' &= t\\
        x' &= x-\rec{2}gt^2\\
        y' &= y\\
        z' &= z
    \end{align*}
    である。従って慣性系ではミンコフスキー計量なので、
    \begin{align*}
        ds^2 &= -c^2dt^2+dx^2+dy^2+dz^2\\
        \intertext{$dx=\pd[x]{x'}dx'+\pd[x]{t'}dt'$より}
        &= -c^2dt'^2+(dx'+gtdt')^2+dy'^2+dz'^2\\
        &= \lr{-1+\frac{(gt)^2}{c^2}}(cdt')^2+2gtdt'dx'
        +dx'^2+dy'^2+dz'^2\\
        \intertext{$K'$系の原点では$x=\rec{2}gt'^2$なので}
        &= \lr{-1+\frac{2gx}{c^2}}(cdt')^2+2\sqrt{2gx}dt'dx'
        +dx'^2+dy'^2+dz'^2\\
        \intertext{重力ポテンシャルは$\phi = -gx$なので}
        &= \lr{-1-\frac{2\phi}{c^2}}(cdt')^2-2\sqrt{-2\phi}dt'dx'
        +dx'^2+dy'^2+dz'^2
    \end{align*}
    重力ポテンシャルはスカラーなので、加速の方向に依らず、$g_{00}=-1-\frac{2\phi}{c^2}$である。\\
    時空の計量は物質の分布、すなわちエネルギー運動量テンソルに依存するはずである。エネルギー運動量テンソルには、$\na_jT_{ij}=0$という関係が成り立っていた。計量に関係する量$X_{ij}$で、$X_{ij}=T_{ij}$となるようなものがあれば、$\na_jX_{ij}=0$を満たす。アインシュタインテンソルや計量テンソルはそのような性質を持っている。ニュートン力学における重力場の方程式はポアソン方程式
        \[\Delta \phi = 4\pi G\rho\]
    である。$\phi$の二階微分は$g_{00}$の二階微分に比例するので、$X_{ij}$は計量の二階微分を含まなければならないとわかる。実は、$g_{ij}$の$x$に関する0,1,2階微分を含み、$\pd[^2g_{ij}]{x^2}$の一次式であり、その共変微分が0となるような二階反変テンソルは$G_{ij}$と$g_{ij}$の線形結合に限るということが分かっている。つまり、
        \[G_{ij}+\Lambda g_{ij} = \kappa T_{ij}\]
    である。ところで左辺を計算するには、$g_{ij}$から$\Ga^i_{jk}$を求め、リーマン曲率テンソルを計算し、リッチテンソルとリッチスカラーを導くという流れになる。リッチスカラーを求めるのは大変なので方程式を簡略化する。両辺に$g^{ij}$を掛けて縮約する。
        \[g^{ij}R_{ij}-\rec{2}Rg^{ij}g_{ij}+\Lambda g^{ij}g_{ij} = \kappa g^{ij}T_{ij}\]
    $g^{ij}g_{ij}$はクロネッカーのデルタのトレースなので4である。
    \begin{gather*}
        R - 2R + 4\Lambda = \kappa T\\
        R = 4\Lambda - \kappa T
    \end{gather*}
    これをもとの式に代入して、
    \begin{align*}
        R_{ij}-\rec{2}(4\Lambda-\kappa T)g_ij+\Lambda g_{ij} = \kappa T_{ij}\\
        R_{ij}-\Lambda g_{ij} = \kappa(T_{ij}+\rec{2}Tg_{ij})
    \end{align*}
    である。\\
    まずリッチテンソルを計算する。$g_{ij}=\eta_{ij}+h_{ij}$とすると、$h_{00}=-\frac{2\phi}{c^2}$である。クリストッフェル記号は、
    \begin{align*}
        \Ga^k_{ij} &= \rec{2}g^{kl}\lr{\pd[g_{lj}]{x^i}+\pd[g_{li}]{x^j}-\pd[g_{ij}]{x^l}}\\
        \intertext{微分すれば$\eta_{ij}$は消える。$h_{ij}$の二次の項を無視すれば、}
        &= \rec{2}\eta_{kl}\lr{\pd[h_{lj}]{x^i}+\pd[h_{li}]{x^j}-\pd[h_{ij}]{x^l}}\\
        \Ga^k_{00} &= \rec{2}\eta_{kl}\lr{\pd[h_{l0}]{x^0}+\pd[h_{l0}]{x^0}-\pd[h_{00}]{x^l}}\\
        \intertext{計量は変化しないので時間微分は消えて}
        &= -\rec{2}\eta_{kl}\pd[h_{00}]{x^l}
    \end{align*}
    である。リッチテンソルは、後ろの二項が$h_{ij}$の二次になるので無視して、
    \begin{align*}
        R_{00} &= R^k_{0k0}\\
        &= \pd[\Ga^k_{00}]{x^k}-\pd[\Ga^k_{0k}]{x^0}\\
        \intertext{時間微分は消えて、}
        &= \pd[\Ga^k_{00}]{x^k}\\
        &= -\rec{2}\eta_{kl}\ppd{h_{00}}{x^k}{x^l}\\
        &= -\rec{2}\pa^i\pa_ih_{00}\\
        \intertext{時間微分は0なので、}
        &= -\rec{2}\Delta h_{00}
    \end{align*}
    次にエネルギー運動量テンソルを計算する。$T_{ij}=\rho u_iu_j$だが、速度が十分小さいので$T_{00}=\rho c^2$で残りの成分は全て0である。よって$T=g_{ij}T^{ij}=g_{00}\rho c^2$である。従って方程式は、
    \begin{align*}
        R_{ij}-\Lambda g_{ij} = \kappa(T_{ij}+\rec{2}Tg_{ij})\\
        \intertext{${00}$成分だけ取り出して、}
        -\rec{2}\Delta h_{00}-\Lambda g_{00} = \kappa\rho c^2(1-\rec{2}g_{00}^2)\\
        \intertext{$h_{00}=-\frac{2\phi}{c^2},g_{00}=-1$を代入すれば、}
        \rec{c^2}\Delta\phi+\Lambda = \frac{c^2}{2}\kappa\rho
    \end{align*}
    ポアソン方程式$\Delta \phi = 4\pi G\rho$と比較すると、$\Lambda=0,\kappa=\frac{8\pi G}{c^4}$
    であることが分かる。つまり、
        \[R_{ij}-\rec{2}Rg_{ij} = \frac{8\pi G}{c^4}T_{ij}\]
    である。これをアインシュタイン方程式または重力場方程式という。
    \paragraph{宇宙項}