\repart{電磁気学}

\section{電気回路}
    電気回路を構成する素子には、入力信号を増幅・整流する能動素子(ダイオード、トランジスタなど)
    と受動素子(抵抗、コンデンサ、コイルなど)がある。能動素子を含む回路を特に電子回路という。
    \subsection{キルヒホッフの法則}
        電気回路に共通して成り立つする法則としてキルヒホッフの法則がある。\\
        アンペールの法則の両辺の$\dive$をとってやると、電荷保存の法則が導かれる。
        電荷密度を一定とすれば、
            \[\dive i = -\pd[\rho]{t} = 0\]
        またファラデーの法則より
            \[\oint E\cdot ds = 0\]
        である。従って電気回路において、
        \begin{enumerate}
            \item 任意の節点に流れ込む電流の総和は0
            \item 任意の閉路にかかる電圧の総和は0
        \end{enumerate}
        が成り立つ。この二つをキルヒホッフの法則と呼ぶ。

\section{線形素子}
    \subsection{抵抗}
        ドルーデモデルとは、電気伝導についてのモデルで、物質(特に金属)内部の電子
        の運動について記述する。\\
        同じ材質中では一様な電場が働く。力は電場に比例し、電子の加速度は力に比例する。つま
        り導線中に一様な電場が働くとき、電子は一様な加速を受け定常電流ではなくなってしまう。
        電場によって加速された電子は物質中の原子核に衝突して減速する。電場が比較的弱い場合
        には、この減速が一様な電場による力とつり合い、電子は十分長い時間で見れば一定の速度で
        運動することになる。これは空気抵抗を受けながら落下する物体の運動に似ている。最終的に物
        体は一定の速度(終端速度)に落ち着く。終端速度が重力場に比例するように、電子の速度に
        比例する電流密度も電場に比例する。
            \[i = \sigma E\]
        比例定数$\sigma$は導電率(電気伝導率)である。この式から直ちにオームの法則が導かれる。
        抵抗の断面積を$S$,長さを$L$とすると、
        \begin{align*}
            \frac{I}{S} &= \sigma \frac{V}{L}\\
            V &= \frac{L}{\sigma S} I\\
            V &= RI
        \end{align*}
        \begin{figure}[H]
            \begin{center}\begin{circuitikz}[american currents]
                    \draw(0,0) to[R=$R$] (2,0);
            \end{circuitikz}\end{center}
            \caption{抵抗}
        \end{figure}
        $R$は電気抵抗(Resistance)、$\rec{\sigma}$は$\rho$と書き、抵抗率と呼ばれる。オー
        ムの法則は電磁気学だけからは導くことができない近似的な法則だが、非常に多くの物質、電場の
        範囲で成り立つことが分かっている。\\
        抵抗を直列に繋いだ場合、$V_1=R_1I,V_2=R_2I$より、
            \[V = (V_1+V_2) = (R_1+R_2)I\]
            \[R = R_1 + R_2\]
        並列に繋いだ場合は、$V=R_1I_1,V=R_2I_2$より、
            \[\frac{V}{R_1} + \frac{V}{R_2} = I_1 + I_2 = I\]
            \[\rec{R} = \rec{R_1} + \rec{R_2}\]
        となる。
    \subsection{コンデンサ}
        \begin{figure}[H]
            \begin{center}\begin{circuitikz}[american currents]
                    \draw(0,0) to[C=$C$] (2,0);
            \end{circuitikz}\end{center}
            \caption{コンデンサ}
        \end{figure}
        絶縁した導体間に電圧がかかると電荷が貯め込まれる。これを利用して電荷を蓄電したり放出し
        たりするものをコンデンサという。この時導体の一方には電荷$Q$が、もう一方には$-Q$が蓄電さ
        れる。導体間の電場が一様だと仮定すると、蓄積される電荷は電圧に比例し、
            \[Q=CV\]
        である。$C[\mathrm{F}]$は静電容量(Capacitance)と呼ばれる。面積$S$、間隔$d$
        の平行平板コンデンサの場合、片方の極版を覆う曲面を考え、ガウスの法則を適用すると、
        \begin{align*}
            ES &= \frac{Q}{\epsilon}\\
            Q &= \frac{\epsilon S}{d}V
        \end{align*}
        となる。\\
        コンデンサを直列に繋いだ場合は、それぞれ$Q=C_1V_1,Q=C_2V_2$より
            \[\frac{Q}{C_1}+\frac{Q}{C_2} = V\] 
            \[\rec{C} = \rec{C_1}+\rec{C_2}\]
        並列に繋いだ場合は、それぞれ$Q_1=C_1V,Q_2=C_2V$より
            \[Q = Q_1+Q_2 = (C_1+C_2)V\]
            \[C = C_1+C_2\]
        となる。\\
        コンデンサに蓄えられるエネルギーは、電場$E=\frac{q}{Cd}$に逆らって微小電荷$dq$を距離
        $d$移動させるときの仕事に等しいので、
        \begin{align*}
            U &= \int_0^Q dq\frac{q}{Cd}d\\
            &= \frac{Q^2}{2C} = \rec{2}CV^2
        \end{align*}
    \subsection{インダクタ}
        \begin{figure}[H]
            \begin{center}\begin{circuitikz}[american currents]
                \draw(0,0) to[L=$L$] (2,0);
            \end{circuitikz}\end{center}
            \caption{インダクタ}
        \end{figure}
        インダクタとは、電流によって形成される磁場にエネルギーを蓄える受動素子であり、一般的にはコイルで作られ
        る。理想的なインダクタは電気抵抗や静電容量を持たないが、実際の計算ではインダクタ単独でRL回路やLC
        回路とみなすこともある。コイルに流れる電流が変化すると、磁場が発生し電流の変化を妨げるように起電力が
        生まれる。インダクタが自分自身に対して起電力を生ずるとき自己誘導と呼び、別のインダクタを誘導するとき
        相互誘導と呼ぶ。起電力は以下の近似の下で電流の変化に比例する。
        \begin{enumerate}
            \item 電場の時間変化が小さい(準静的過程)
            \item インダクタの長さは十分長い。
        \end{enumerate}
        インダクタの式を導く前にこの近似について評価を与える。導体内部の電場が
            \[E(t) = E_0\sin\omega t\]
        であるとする。導電率を$\sigma$とすると、電流密度と変位電流は、
        \begin{align*}
            i(t) &= \sigma E = \sigma E_0\sin\omega t\\
            i_d(t) &= \epsilon_0\pd[E]{t} = \epsilon_0\omega E_0\cos\omega t\\
            \intertext{つまり変位電流が無視できるための条件は、}
            \omega &\ll \frac{\sigma}{\epsilon_0}
            \intertext{通常の金属では$\sigma \simeq 10^7 [\rm{\Omega^{-1}\cdot m^{-1}}]$、
            また誘電率は$\epsilon_0 \simeq 10^{-10}[\rm{A^2\cdot s^2\cdot N^{-1}\cdot m^{-2}}]$なので、}
            \omega &\ll 10^{17} [\rm{s^{-1}}]\\
        \end{align*}
        となる。通常の交流が$\omega \simeq 10^2[\rm{s^{-1}}]$なので変位電流は完全に無視できるといえる。\\
        まずコイルを貫く磁束は電流に比例することを示す。アンペール-マクスウェルの式
            \[\rot B - \epsilon\mu\pd[E]{t} = \mu i\]
        から左辺第二項の電場の時間変化を無視すると、
            \[\rot B = \mu i\]
        となりアンペールの法則が導かれる。ベクトルポテンシャルを使うと$B = \rot A$で、
        $\rot\rot A = \grad\dive A - \Delta A$となる。さらにクーロンゲージ$\dive A = 0$を適用して、
            \[\Delta A = -\mu i\]
        これを解くと、
            \[A = \frac{\mu}{4\pi}\int \frac{i(r')}{|r-r'|}dr'\]
        回転をとり、
            \[B = \frac{\mu}{4\pi}\int \frac{i(r')\times (r-r')}{|r-r'^3|}dr'\]
        となってビオ=サバールの法則が導かれる。したがってコイルを貫く磁束は電流に比例する。この時の比例定数$L$を
        インダクタンスと呼ぶ。起電力は電場の線積分なのでストークスの定理を適用し、
        \begin{align*}
            V &= \int_{\pa S} E\cdot ds = \int_S \rot EdS\\
            \intertext{ファラデー-マクスウェルの式$\rot E = -\pd[B]{t}$より、}
            &= -\pd[\Phi]{t}\\
            \intertext{$N$回巻の場合は、}
            V &= -N\pd[\Phi]{t}\\
        \end{align*}
        となりファラデーの法則を得る。よって$\Phi = LI$なので
            \[V = -\de[\Phi]{t} = -NL\de[I]{t}\]
        を得る。比例定数を自己インダクタンスと呼ぶ。自己インダクタンスも$L$という記号で表すことが多い。\\
        二つのコイルがあり、結合定数を$k$とする。それぞれのコイルが生み出す磁束を$\Phi_1,\Phi_2$
        とすると、
        \begin{align*}
            \rot \rot E = -\mu \pd[i]{t}
            \rot \Phi_1 &= \mu I_1\\
        \end{align*}
        コイルを直列に繋いだ場合、それぞれの磁場が互いに影響を及ぼさないとすると、$V_1=L_1\de[I]{t},
        V_2=L_2\de[I]{t}$より、
            \[V = (V_1+V_2) = (L_1+L_2)\de[I]{t}\]
            \[L = L_1+L_2\]
        並列に繋いだ場合は、$V=L_1\de[I_1]{t},V=L_2\de[I_2]{t}$より、
            \[\frac{V}{L_1}+\frac{V}{L_2} = \de[(I_1+I_2)]{t} = \de[I]{t}\]
            \[\rec{L} = \rec{L_1}+\rec{L_2}\]
        となる。\\
        インダクタに蓄えられるエネルギーは電流を徐々に上げていった時の電力の総和であり、
        \begin{align*}
            U &= \int_0^t L\de[I']{t}I'dt\\
            &= \int_0^I LI'dI'\\
            &= \rec{2}LI^2
        \end{align*}
    \subsection{テレゲンの定理}
    \subsection{テブナンの定理}
    \subsection{ノートンの定理}
    \subsection{ミルマンの定理}
    \subsection{補償定理}