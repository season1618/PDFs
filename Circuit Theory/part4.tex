\repart{アナログ回路}

\section{RC回路}
    \subsection{ハイパスフィルタとローパスフィルタ}
        \begin{figure}[H]
            \begin{minipage}{0.5\hsize}
                \begin{center}\begin{circuitikz}[american currents]
                    \draw(0,2) to[C=$C$] (3,2) to[short] (4,2);
                    \draw(0,0) to[short] (4,0);
                    \draw(3,2) to[R=$R$] (3,0);
                \end{circuitikz}\end{center}
            \caption{ハイパスフィルタ}
            \label{fig:one}
            \end{minipage}
            \begin{minipage}{0.5\hsize}
                \begin{center}\begin{circuitikz}[american currents]
                    \draw(0,2) to[R=$R$] (3,2) to[short] (4,2);
                    \draw(0,0) to[short] (4,0);
                    \draw(3,2) to[C=$C$] (3,0);
                \end{circuitikz}\end{center}
            \caption{ローパスフィルタ}
            \label{fig:two}
            \end{minipage}
        \end{figure}
        抵抗とコンデンサを直列につなぐと
        \begin{align*}
            Ri(t) + \frac{q(t)}{C} &= v_{in}(t)\\
            \intertext{$q(0)=0$として、}
            Ri(t) + \rec{C}\int_0^t i(t')dt' &= v_{in}(t)
        \end{align*}
        フェーザ表示すると、
            \[RI + \rec{Cs}I = V_{in}\]
            \[I = \frac{Cs}{1+RCs}V_{in}\]
        である。抵抗とコンデンサの電圧はそれぞれ、
        \begin{align*}
            V_R &= RI = \frac{RCs}{1+RCs}V_{in}\\
            V_C &= \rec{Cs}I = \rec{1+RCs}V_{in}\\
        \end{align*}
        入力信号が正弦波定常状態($s = j\omega$)のとき振幅の比は、
        \begin{align*}
            \vlr{\frac{V_R}{V_{in}}} &= \vlr{\frac{j\omega RC}{1+j\omega RC}} = \frac{\omega RC}{\sqrt{1+(\omega RC)^2}}\\
            \vlr{\frac{V_C}{V_{in}}} &= \vlr{\rec{1+j\omega RC}} = \rec{\sqrt{1+(\omega RC)^2}}\\
        \end{align*}
        となる。したがって前者がハイパスフィルタ、後者がローパスフィルタとして機能することが分かる。
    \subsection{微分器と積分器}
        低周波数($\omega \ll \rec{RC}$)の信号を考える。
        \begin{align*}
            I &= \frac{V_{in}}{\rec{j\omega C}+R}\\
            \intertext{$R \ll \rec{\omega C}$より、}
            &\approx j\omega CV_{in}\\
            V_{in} &\approx \frac{I}{j\omega C} \approx V_C\\
            \intertext{抵抗器にかかる電圧は、}
            V_R &= R\de[(CV_C)]{t} \approx RC\de[V_{in}]{t}\\
        \end{align*}
        となり、微分器として機能する。

        次に高周波数($\omega \gg \rec{RC}$)の信号を考える。
        \begin{align*}
            I &= \frac{V_{in}}{\rec{j\omega C}+R}\\
            \intertext{$R \gg \rec{\omega C}$より、}
            &\approx \frac{V_{in}}{R}
            \intertext{コンデンサにかかる電圧は、}
            V_C &= \rec{C}\int_0^t I dt \approx \rec{RC}\int_0^t V_{in}\\
        \end{align*}
        となり積分器として機能する。

\section{LC回路} 
    信号が正弦波の場合を考える。
    \subsection{共振回路}
        直列の場合、合成インピーダンスは、
        \begin{align*}
            Z = j\omega L + \rec{j\omega C}\\
            = j(\omega L - \rec{\omega C})\\
            \intertext{電流が最大となるのは振幅の比$|Z|$が最小の時なので、共振周波数は、}
            \omega L - \rec{\omega C} = 0\\
            \omega = \rec{\sqrt{LC}}\\
            \intertext{単位を$\mathrm{Hz}$に直すと、}
            f = \rec{2\pi\sqrt{\omega C}}
        \end{align*}
        となりバンドパスフィルタ(Band-Pass Filter : BPF)として機能する。

        一方並列の場合、合成アドミッタンスは、
        \begin{align*}
            \rec{Z} = \rec{j\omega L} + j\omega C\\
            = j(\omega C - \rec{\omega L})\\
            \intertext{電流が最小となるのは振幅の比$|Z|$が最大の時なので、共振周波数は、}
            \omega C - \rec{\omega L} = 0\\
            \omega = \rec{\sqrt{LC}}\\
            \intertext{単位を$\mathrm{Hz}$に直すと、}
            f = \rec{2\pi\sqrt{\omega C}}
        \end{align*}
        となりバンドストップフィルタ(Band-Stop Filter : BSF)として機能する。ちなみにコイルが抵抗を伴っていると考えると、直列LC回路と並列LC回路の共振周波数はそれぞれ、
        \begin{gather*}
            \omega = \rec{\sqrt{LC}}\\
            \omega = \sqrt{\rec{LC} - \frac{R^2}{L^2}}\\
        \end{gather*}
        である。
\section{増幅回路}
    \subsection{エミッタ(ソース)接地回路}
\section{発振回路}
    発振回路は、持続した交流を作る回路である。原理によって帰還型と弛緩型に分けられる。
    帰還型は、出力の一部を入力に合流させるもので、弛緩型はスイッチングを制御することで
    一定の電気信号を得る。
\section{変調回路}
\section{オペアンプ}
    カレントミラー
\section{電源回路}
\section{変圧器}
\section{アンテナ}
\section{センサ}