\repart{デジタル回路}

\section{論理ゲート}
    真理値あるいは二進数の0と1を、電圧や電流、位相、パルスなどで表現し、論理的な演算
    を実装したものを論理回路という。論理回路の基本となる演算をする素子を論理ゲートと呼ぶ。
    \subsection{NOT:論理否定}
        CMOSトランジスタを使うと低電力で動作することができる。
    \subsection{OR:論理和}
    \subsection{AND:論理積}
    \subsection{XOR:排他的論理和}