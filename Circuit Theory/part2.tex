    \section{非線形回路}
        \subsection{半導体}
        \subsection{ダイオード}
            ダイオードとは、整流作用を持つ素子である。二つの端子、アノード(陽極)とカソード(陰極)がある。
            アノードをA、カソードをKと書く。アノードからカソードへは電流を通すが、カソードからアノードへはほとん
            ど通さない。前者を順方向と呼び、後者を逆方向という。半導体のpnダイオードの場合は、
            pがアノード、nがカソードとなる。逆方向にかけられる電圧には限界があり、逆耐電圧と呼ばれる。\\
            ダイオードの種類
            \begin{enumrate}
                \item npダイオード
                \item 発光ダイオード
            \end{enumrate}
            ダイオードの順方向電圧と流れる電流の関係を表すモデルとして、ショックレーのダイオード方程式がある。
                \[i = I_s(e^{\frac{ev}{nk_BT}}-1)\]
            $I_s,n$はダイオードの種類によって決まる定数である。
        \subsection{トランジスタ}
            トランジスタとは、増幅作用とスイッチング作用を持つ素子である。
            トランジスタの種類
            \begin{enumrate}
                \item バイポーラトランジスタ(npn型,pnp型)
                \item 電解効果トランジスタ(FET:Field Effect Transistor)
                \item フォトトランジスタ
            \end{enumrate}
            \subsubsection{バイポーラトランジスタ(npn型,pnp型)}
                三つの端子はそれぞれエミッタ(E)、ベース(B)、コレクタ(C)と呼ばれる。ベース-エミッタ間に電流が
                流れたときだけエミッタ-コレクタ間に電流が流れるという仕組みである。前者をベース電流、後者を
                コレクタ電流という。ベース電流に対するコレクタ電流の比を$h_{FE}$(直流電流増幅率)という。
            \subsubsection{電解効果トランジスタ(FET:Field Effect Transistor)}
                三つの端子はそれぞれソース(S)、ドレイン(D)、ゲート(G)と呼ばれる。ゲートにかかる電圧によって
                ドレイン-ソース間の電流を制御する。電圧の変化で制御するため消費電力が低く抑えられる。
                ゲート-ソース電流$i_{GS}$とドレイン-ソース間の電圧$v_{DS}$の関係は次の通りである。
                    \[i_{DS} = \frac{I_{DSS}}{V_p^2}(v_{GS}-V_p)^2\]
                ただし、$I_{DSS}$は飽和電流であり、$v_{GS}=0$の時のドレイン-ソース電流である。また$V_p$
                はピンチオフ電圧である。