\repart{波動光学}
    \section{電磁場の境界条件}
        以上は一定の誘電率と透磁率を持つ物質中での電磁場を想定していたが、実際には場所によって誘電率や透
        磁率は異なる。ここでは異なる物質の境界での電磁場について考察する。二つの物質が平面を境に接しているとす
        る。二つの物質にまたがる境界に垂直な長方形の周にファラデーの法則を適用して、
            \[\int(E_1 - E_2)\cdot ds + \int\pd[B]{t}dS = 0\]
        長方形を小さくしていけば、第二項は0に近づくので、
            \[E_1\cdot ds = E_2\cdot ds\]
        同じ領域にアンペールの法則を適用する。電流密度は0として、
        \begin{gather*}
            \int(H_1 - H_2)\cdot ds - \int\pd[D]{t}dS = 0\\
            H_1\cdot ds = H_2\cdot ds\\
        \end{gather*}
        次に境界面の一部を覆うような直方体にガウスの法則を適用する。直方体内の電荷を0として、
        \begin{gather*}
            \int(D_1 - D_2)\cdot n = 0\\
            D_1\cdot n = D_2\cdot n\\
        \end{gather*}
        同じ領域に磁場に関するガウスの法則を適用すると、
        \begin{gather*}
            \int(B_1 - B_2)\cdot n = 0\\
            B_1\cdot n = B_2\cdot n\\
        \end{gather*}
        となる。よって電磁場の境界条件は、
        \begin{itemize}
            \item 境界面に平行な電場は連続
            \item 境界面に平行な磁場は連続
            \item 境界面に垂直な電束密度は連続
            \item 境界面に垂直な磁束密度は連続
        \end{itemize}
        である。
    \section{ローレンツ振動子モデル}
        物質中を電磁波が入射したときのことを考える。原子核は重いのでほとんど動かないと
        してよい。分子の中の電子の運動は量子力学の効果を考慮する必要があるが、今回は
        分子に拘束された電子を、抵抗を受けながら振動する調和振動子とみなす。これをロー
        レンツ振動子モデルという。電子の変位を$u$とすると、固有振動数が$\omega_0$で
        あるときの運動方程式は、
            \[\de[^2u]{t^2} + \gamma \de[u]{t} + \omega_0^2u = \frac{eE}{m}\]
        である。電場が角振動数$\omega$で振動すれば、電子もそれに追随して動く。振動す
        る変数の問題を解くときは、それを複素数で表すと便利である。これをフェーザ表示という。
        \begin{align*}
            E(t) &= E_0\sin(\omega t + \alpha) = {\rm Im} (E_0e^{i\alpha}e^{i\omega t})\\
            u(t) &= u_0\sin(\omega t + \beta) = {\rm Im} (u_0e^{i\beta}e^{i\omega t})\\
        \end{align*}
        と置く。相殺される$e^{i\omega t}$は無視して、$E = E_0e^{i\alpha},u = u_0e^{i\beta}$
        とする。分極ベクトルと電束密度も同様に定義すると、先程の運動方程式は、
        \begin{gather*}
            -\omega^2 u + i\gamma\omega u + \omega_0^2 u = \frac{eE}{m}\\
            u = \frac{eE}{m(\omega_0^2 - \omega^2 + i\gamma\omega)}\\
        \end{gather*}
        各分子に分極を起こす電子が$z$個あり、単位体積中の分子数を$n$とすれば分極ベ
        クトルは、
            \[P = nzeu = \frac{nze^2}{m(\omega_0^2 - \omega^2 + i\gamma\omega)}E\]
        一般に分子レベルのミクロな電場と我々が観測するマクロな電場は同じではない。しかし
        その差が無視できるとすれば、複素電気感受率と複素誘電率は、
        \begin{align*}
            \chi(\omega) &= \frac{nze^2}{m(\omega_0^2 - \omega^2 + i\gamma\omega)}\\
            \epsilon(\omega) &= \epsilon_0 + \frac{nze^2}{m(\omega_0^2 - \omega^2 + i\gamma\omega)}\\
        \end{align*}
        となる。実際には、複数の固有振動数と減衰係数について足し合わせたものとなる。電場
        、電束密度、分極ベクトルは、その複素数表示の虚数部を見ればよいが、誘電率はそうで
        なはいことに注意しよう。複素電気感受率$\chi = \chi_1 + i\chi_2$の実部と虚
        部は、
        \begin{align*}
            \chi_1 &= \frac{nze^2}{m}\frac{\omega_0^2 - \omega^2}{(\omega_0^2 - \omega^2)^2 + \gamma^2\omega^2}\\
            \chi_2 &= - \frac{nze^2}{m}\frac{\gamma\omega}{(\omega_0^2 - \omega^2)^2 + \gamma^2\omega^2}\\
        \end{align*}
        となる。電場が電子に対してする仕事を考える。まず分極ベクトルは、
        \begin{align*}
            P(t) &= {\rm Im}(\chi(\omega)E(t))\\
            &= {\rm Im}((\chi_1(\omega) + i\chi_2(\omega))E_0e^{i(\omega t + \alpha)})\\
            &= \chi_1(\omega)E_0\sin(\omega t + \alpha) + \chi_2(\omega)E_0\cos(\omega t  + \alpha)\\
        \end{align*}
        より変位電流は、
        \begin{align*}
            i_p(t) = \pd[P]{t} = \omega\chi_1(\omega)E_0\cos(\omega t + \alpha)
            - \omega\chi_2(\omega)E_0\sin(\omega t + \alpha)\\
        \end{align*}
        なので、電場が単位体積単位時間当たりに電子に行う仕事を周期
        $T = \frac{2\pi}{\omega}$について平均すると、
        \begin{align*}
            W &= \oint E(t)i_p(t)dt\\
            &= \omega E_0^2\oint \chi_1(\omega)\sin(\omega t + \alpha)\cos(\omega t + \alpha)
            - \chi_2(\omega)\sin^2(\omega t + \alpha) dt\\
            \intertext{第一項は周積分で消える。よって}
            &= - \rec{2}\omega\chi_2(\omega)E_0^2\\
        \end{align*}
        となる。$\chi_2(\omega)$は負なので、電場は常に正の仕事をする。これは電場のエネ
        ルギーが誘電体に吸収されることを意味している。分子内で加速された電子の運動エネルギ
        ーが熱として失われる。
            \[\rec{\omega\chi_2(\omega)} \propto \frac{(\omega_0^2 - \omega^2)^2 + \gamma^2\omega^2}{\omega^2}\]
        より吸収は共鳴点$\omega = \omega_0$で最も大きくなり、低周波や高周波の振動電
        場は誘電体との相互作用が弱く、ほとんど通り抜ける。また、振動数が小さい場合には
        $\omega = 0$に近づくので静電場に対する誘電率で近似できる。磁性体の磁化率や透
        磁率も振動する磁場に対しては振動数に依存するが、強磁性体以外の普通の物質では、
        $\chi_m$は$\mu_0$に比べて非常に小さく、物質の電磁気的な性質にはほとんど影響を
        与えない。
    \section{誘電体中の電磁波}
        物質中のマクスウェル方程式は真空中と全く同じ形をしているので、電磁波の速さも真空中の
        誘電率と透磁率を物質中のものに変えるだけで良い。真空中同様、電荷密度と電流密度を0
        とすると波動方程式は、
            \[\Delta E - \epsilon\mu\pd[^2E]{t^2} = 0\]
        となるので物質中の光速は、
            \[v = \rec{\sqrt{\epsilon\mu}}\]
        $\epsilon > \epsilon_0$で$\mu$は物質によって$\mu_0$より大きくなったり小さくなっ
        たりするが、結果的には物質中では光速は遅くなる。真空中との光速の比
            \[n = \frac{c}{v}\]
        をその物質の絶対屈折率という。電磁場の振動が速くなると、誘電率の振動数依存性が顕
        著になってくる。エネルギーの吸収を無視すれば、
        \begin{gather*}
            \Delta E_\omega - \epsilon(\omega)\mu\pd[E_\omega]{t} = 0\\
            v_\omega = \rec{\sqrt{\epsilon(\omega)\mu}}\\
        \end{gather*}
        となる。振動数によって速さが異なるため、電磁波は時間とともに位相がずれて波の形が変わる
        。これを波の分散という。速さが異なれば屈折率も当然異なるので、異なる物質の境界面で分
        光する。これも分散と呼ばれる。
    \section{導体中の電磁波}
        導体中は自由電子があるため、振動電場による分極電流が無視できない。それに伴いエネルギ
        ーの吸収も大きく、電磁波の減衰も激しい。電場が$x$方向、磁場が$y$方向、$z$方向に伝播
        する電磁波を考える。導体中に振動する電場が入射すると電子の動きに遅れが生じ、電場は相
        殺されないので、誘電率を考えることができる。同様に透磁率も考えられる。電磁場は$z,t$のみに依存するとして、
        \begin{align*}
            \pd[E_x]{z} + \pd[B_y]{t} &= 0\\
            -\pd[B_y]{z} - \epsilon\mu\pd[E_x]{t} &= \mu i_x\\
        \end{align*}
        電場が振動する際は、導電率も振動数に依存する。しかし低周波の場合はほぼ一定とみなすこ
        とができるので、
            \[i_x = \sigma E_x\]
        と表せる。先程の第一式を$z$で、第二式を$t$で偏微分して辺々足すと、磁場の項が消えて、
            \[\pd[^2E_x]{z^2} - \epsilon\mu\pd[^2E_x]{t^2} - \mu\sigma\pd[E_x]{t} = 0\]
        となる。ここで振動電場を先程を同じように複素数で表す。
            \[E_x(z,t) = E(z)\sin(\omega t + \alpha) = {\rm Im}(E(z)e^{i(\omega t + \alpha)})\]
        これを代入すると、
        \begin{align*}
            \pd[E_x]{t} &= i\omega E_x\\
            \pd[^2E_x]{t^2} &= - \omega^2 E_x\\
        \end{align*}
        なので共通の$e^{i(\omega t + \alpha)}$を落とすと、
            \[\pd[^2E_x(z)]{z^2} + (\epsilon\mu\omega^2 - i\mu\sigma\omega)E_x(z) = 0\]
        これは単振動の微分方程式なので、$E(z) = E_0e^{-ikz}$と置くと、
        \begin{gather*}
            [- k^2 + (\epsilon\mu\omega^2 - i\mu\sigma\omega)]E_0e^{-ikz} = 0\\
            k^2 = \epsilon\mu\omega^2 - i\mu\sigma\omega\\
        \end{gather*}
        となる。ここで低周波の場合は第一項を無視できる。実際、
        普通の金属の導電率は$\sigma \simeq 10^7 {\rm \Omega^{-1}\cdot m^{-1}}$、
        誘電率は$\epsilon \simeq \epsilon_0 \simeq 10^{-11}{\rm C^2\cdot N^{-1}\cdot m^{-2}}$
        として、
            \[\omega \ll \frac{\sigma}{\epsilon} \simeq 10^{18}{\rm s^{-1}}\]
        となり、普通の電波($\omega \simeq 10^4-10^{10}$)や可視光($\omega \simeq 10^{14}$)
        では第二項だけで十分近似できる。この時波数は複号を選択して、
            \[k = \frac{1 - i}{\sqrt{2}}\sqrt{\mu\sigma\omega} = \frac{1 - i}{l}\]
        なので電場は、
            \[E_0e^{i(\omega t - kz)} = E_0e^{-z/l}\sin\lr{\omega t - \frac{z}{l}}\]
        となる。これは平らな導体の表面に平面波の電磁波が入射したときの様子を表している。$z = 0$が導体
        の表面だとすると、$t = 0$における導体内部の電場の様子は図のようになる。電場は導体に入った途端
        急激に減衰し、$l$の数倍の距離を進むとほとんど消えてしまう。侵入できる深さは先程の近似が成り立つ
        範囲で振動数が高いほど短い。$\omega = 10^{10}{\rm s^{-1}}$の場合、
            \[l \simeq \lr{\frac{2}{4\times 3.14\times 10^{-7}\times 10^7\times 10^{10}}}^{1/2}
            \simeq 10^{-6}\]
        となるので、電磁波は導体中にほとんど侵入できない。

\repart{散乱理論}
    散乱とは、光が物質に入射したとき光を四方八方に放射する現象である。古典的には、入射光によって誘起された電気双極子の振動により二次波が放出される、と説明される。物質を原点に置き、$z$軸の正の方向から光が入射するとする。単位時間に単位面積当たりに入射する粒子数のうち、半径$r$の球面のある立体角$d\Omega$内に散乱される単位時間当たりの粒子数の割合を微分断面積という。光散乱の場合は粒子数の代わりにエネルギーで測る。電磁波は振動するので平均を取る。すなわち、散乱波のポインティングベクトルを$S_s$、入射波のポインティングベクトルを$S_i$とすれば、
        \[\de[\sigma(\theta)]{\Omega} = \frac{|S_s|}{|S_i|}r^2\]
    である。ここで$\theta$は散乱によって$z$軸から逸れた角度であり、散乱角と呼ばれる。これを全立体角で積分したものを全断面積という。
    \section{トムソン散乱}
        自由電子に振動数の低い電磁波が入射したときを考える。電子の運動方程式は、
            \[m\de[u]{t} = eE_i\]
        である。$E_i = E_0\sin\omega t$とすれば、
            \[u = \frac{eE_i}{m\omega^2}\]
        つまり
            \[p(\omega) = \frac{e^2E_0}{m\omega^2}\]
        である。この振動により双極子が誘起され、双極子放射が起こる。$n,E_i$のなす角を$\phi$とすると
        \begin{align*}
            E_s = \frac{\mu}{4\pi}\frac{\omega^2}{r}{n\times(n\times p(\omega))}e^{i\omega(t - r/c)} = \frac{\mu}{4\pi}\frac{e^2E_0}{mr}\sin\phi e^{i\omega(t - r/c)}
            \intertext{それぞれのポインティングベクトルの時間平均は、}
            |\overline{S_i}| = \frac{|\overline{E_i}|^2}{\mu c} = \frac{|E_0|^2}{2\mu c}\\
            |\overline{S_s}| = \frac{|\overline{E_s}|^2}{\mu c} = \rec{\mu c}\lr{\frac{\mu}{4\pi}\frac{\omega^2}{r}|p(\omega)|}^2\sin^2\phi = \frac{|E_0|^2}{2\mu c}\lr{\frac{\mu}{4\pi}\frac{\omega^2}{r}\alpha}^2\sin^2\phi
            \intertext{ただし$\alpha$は分極率である。したがって微分断面積は}
            \de[\sigma]{\Omega} = \lr{\frac{\mu}{4\pi}\omega^2\alpha}^2\sin^2\phi           
        \end{align*}
        無偏光の場合は、散乱角を$\theta$としたとき$n = (\sin\theta, 0, \cos\theta), E_0 / |E_0| = (\cos\psi, \sin\psi, 0)$と置くと
        \begin{align*}
                \cos\phi = n\cdot E_0 / |E_0| = \cos\psi\sin\theta
            \intertext{なので$\psi$について平均を取ると}
                \sin^2\phi = 1 - |\cos^2\psi\sin^2\theta| = \frac{1 + \cos^2\theta}{2}\\
        \end{align*}
        よって
        \begin{align*}
            \de[\sigma]{\Omega} = \lr{\frac{\mu}{4\pi}\omega^2\alpha}^2\frac{1 + \cos^2\theta}{2}
            \intertext{となる。$\alpha = \frac{e^2}{m\omega^2}$を代入すれば}
            \de[\sigma]{\Omega} = \lr{\frac{e^2}{4\pi mc^2\epsilon_0}}^2\frac{1 + \cos^2\theta}{2} = a_0^2\frac{1 + \cos^2\theta}{2}
            \intertext{である。光の進行方向に対して最も強く散乱することが分かる。$a_0$は静電エネルギーと静止エネルギーが一致する半径を示し、古典的電子半径と呼ばれる。全立体角で積分すれば}
            \int_0^\pi\int_0^{2\pi} \frac{1 + \cos^2\theta}{2}\sin\theta d\theta d\phi = \pi \int_0^\pi 2\sin\theta - \sin^3\theta d\theta = \pi \llr{\cos\theta + \rec{3}\cos^3\theta}_0^\pi = \frac{8\pi}{3}
        \end{align*}
        より
            \[\sigma = \frac{8\pi}{3}a_0^2\]
        となる。このような自由電子による散乱をトムソン散乱と呼ぶ。
    \section{レイリー散乱}
        微粒子のサイズが光の波長よりも十分に小さいとき、半径$a$の誘電体の双極子モーメントはクラウジウス・モソッティの関係式より
            \[p = 4\pi\epsilon_0\frac{\epsilon - \epsilon_0}{\epsilon + 2\epsilon_0}a^3E_0\]
        である。微分散乱断面積は
        \begin{align*}
            \de[\sigma]{\Omega} = \lr{\frac{\mu}{4\pi}\omega^2\alpha}^2\frac{1 + \cos^2\theta}{2}\\
            = \lr{\epsilon_0\mu\omega^2\frac{\epsilon - \epsilon_0}{\epsilon + 2\epsilon_0}a^3}^2\frac{1 + \cos^2\theta}{2} = \lr{\frac{\epsilon - \epsilon_0}{\epsilon + 2\epsilon_0}}^2\lr{\frac{\omega}{c}}^4a^6\frac{1 + \cos^2\theta}{2}\\
            \intertext{全断面積は}
            \sigma = \frac{8\pi}{3}\lr{\frac{\epsilon - \epsilon_0}{\epsilon + 2\epsilon_0}}^2\lr{\frac{\omega}{c}}^4a^6
        \end{align*}
        つまり青い光は赤い光より多く散乱される。青空の原因はこのレイリー散乱である。光は空気中の微粒子にぶつかる度に散乱を繰り返し、青い光は垂直方向に離散していく。結果朝焼けや夕焼けが起こる。それでも地平線から離れた上空では青く見え、その中間当たりは白く見える。
    \section{ミー散乱}
    \section{輝度}
        ある面を単位時間当たりに通過するエネルギーを放射束という。電磁波の放射の場合波長ごとの放射束を分光放射束という。光源が広がりを持った場合を考える。放射束を光源表面の面積とその立体角で微分したものを放射輝度という。分光放射輝度も同様である。