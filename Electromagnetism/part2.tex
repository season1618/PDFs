\repart{物質中の電磁場}
    \section{電気双極子モーメント}
        大きさの等しい正と負の電荷が少しの距離を置いて存在している系を電気双極子と呼ぶ。電荷をそれぞれ$+q,-q$とする。負電荷の位置から正電荷の位置に引いたベクトルを$s$として、電気双極子モーメント$p$を次のように定義する。
            \[p = qs\]
        電気双極子が一様な電場中に置かれているとする。この電気双極子の位置エネルギーは電場の方向となす角度によって変わる。電気双極子モーメントが電場と垂直な向きを基準とすると、
        \begin{align*}
            U &= -\llr{q\frac{s}{2}|E|\cos\theta +
            (-q)\lr{-\frac{s}{2}}|E|\cos\theta}\\
            &= -p\cdot E
        \end{align*}
        次に電位を求める。電気双極子の中心を原点として、正電荷と負電荷をそれぞれ$(-\frac{s}{2},0),(+\frac{s}{2},0)$の位置に置く。$r = (x,y,z)$における電位は、
        \begin{align*}
            U(r) = \frac{q}{4\pi\epsilon_0}\lr{\rec{r_1} - \rec{r_2}}\\
            r_1 = \sqrt{r^2 + (s/2)^2 - rs\cos\theta}\\
            r_2 = \sqrt{r^2 + (s/2)^2 + rs\cos\theta}
        \end{align*}
        ここで$s \ll r$として近似すると、
        \begin{align*}
            \rec{r_1} &= \rec{\sqrt{r^2 + (s/2)^2 - rs\cos\theta}}\\
            &= \rec{r}\lr{1 + \lr{\frac{s}{2r}}^2
                - \frac{s}{r}\cos\theta}^{-\rec{2}}\\
            &\fallingdotseq \rec{r}\lr{1 - \rec{2}
                \llr{\lr{\frac{s}{2r}}^2 - \frac{s}{r}\cos\theta}}\\
            &= \rec{r}\lr{1 - \frac{s^2}{8r^2} + \frac{s}{2r}\cos\theta}
        \end{align*}
        同様に、
            \[\rec{r_2} =\rec{r}
            \lr{1 - \frac{s^2}{8r^2} - \frac{s}{2r}\cos\theta}\]
        よって、
        \begin{align*}
            U(r) &= \frac{q}{4\pi\epsilon_0}\frac{s}{r^2}\cos\theta\\
            &= \frac{p\cdot r}{4\pi\epsilon_0r^3}
        \end{align*}
        次に電場を求める。$E(r) = -\pd[U]{r}$なので、
        \begin{align*}
            -\pd[U]{x} &= -\pd{x}\lr{\rec{4\pi\epsilon_0}\frac{p_x x + p_y y + p_z z}{(x^2+y^2+z^2)^{3/2}}}\\
            &= -\rec{4\pi\epsilon_0}\llr{\frac{p_x}{(x^2 + y^2 + z^2)^{3/2}} - (p_x x + p_y y + p_z z)(\frac{3}{2})(x^2 + y^2 + z^2)^{5/2}(2x)}\\
            &= -\rec{4\pi\epsilon_0}\llr{\frac{p_x}{r^3} - \frac{3x(p\cdot r)}{r^5}}
        \end{align*}
        $y,z$も同様なので、結局
            \[E(r) = \rec{4\pi\epsilon_0}\llr{\frac{p}{r^3} - \frac{3r(p\cdot r)}{r^5}}\]
        となる。
    \section{磁気モーメント}
        モノポールが存在するとして、等しい大きさのNとSの磁荷を少しの距離を置いて配置したものを磁気双極子という。磁荷をそれぞれ$+q_m(N),-q_m(S)$とする。単位は$Wb$で、$F = q_m H$が成り立つ。S極からN極へ向かうベクトルを$s$として、磁気双極子モーメントと次のように定義する。
            \[m = q_m s\]
        電気双極子と同じように、エネルギーと磁場はそれぞれ、
        \begin{align*}
            U &= -m\cdot H\\
            H(r) &= -\rec{4\pi\mu_0}\llr{\frac{m}{r^3} - \frac{3r(m\cdot r)}{r^5}}
        \end{align*}
        となる。

        ところが現代では、モノポールは存在しないとするのが主流であるので、その代わりに微小な円形電流を考える。するとこの円形電流の作る磁場は十分遠方では磁気双極子が作るものと同じと見なせる。磁気モーメント$m'$を、
            \[U = -m'\cdot B\]
        が成り立つようにすると、$m'=m/\mu_0$とすれば良いことがわかる。すると磁束密度は、
            \[B(r) = -\frac{\mu_0}{4\pi}\llr{\frac{m'}{r^3}
            -\frac{3r(m'\cdot r)}{r^5}}\]
        となる。

    周囲に空気や水といった物質が充満しているときは、それらの物質中の電荷を直接方程式に含めることなく扱えると便利である。変化するのは電荷と電流なので真空中のマクスウェル方程式のうち変更すべきなのは次の二式である。
    \begin{gather*}
        \dive E = \frac{\rho}{\epsilon_0}\\
        \rot B - \epsilon_0\mu_0\pd[E]{t} = \mu_0 i
    \end{gather*}

    \section{誘電体と電束密度}
        導体に電場をかけると電流が流れるが、絶縁体(不導体)に電場をかけると内部の電荷がその方向に偏る。これを分極という。絶縁体をそのように見たときこれを誘電体という。外部の電場によって誘電体が分極すると全体的に電荷が弱まる。そのとき単位面積あたりに通過した正電荷の量と方向を分極ベクトル$P$と呼ぶ。このとき積分形のガウスの法則は、
            \[\int \epsilon_0 E\cdot dS = q - \int P\cdot dS\]
        第二項は
            \[- \int P\cdot dS = - \int \dive P dV\]
        と書き換えることができる。誘起された電荷密度$-\dive P$を分極電荷という。
            \[\int (\epsilon_0 E + P)\cdot dS = q\]
        ここで
            \[D = \epsilon_0 E + P\]
        と置くと、
            \[\dive D = \rho\]
        となる。$D$を電束密度という。多くの物質では$P = \chi E$という関係が近似的に成り立つ。$\chi$は電気感受率と呼ばれる。ばねにおけるフックの法則と同じようなもので、比較的弱い電場では電場に比例して分極を起こすが、ばねを伸ばしすぎると壊れるように、大きい電場をかけると絶縁破壊が起きて電気が流れてしまう。$\epsilon = \epsilon_0 + \chi$と置くと、$D = \epsilon E$と表せる。$\epsilon$を物質の誘電率という。常に$\chi > 0$なので$\epsilon > \epsilon_0$である。また$\frac{\epsilon}{\epsilon_0}$を比誘電率という。
    \section{磁性体と磁場}
        磁気モノポールは存在しないと仮定すれば、磁場の起源はすべて電流である。特に永久磁石などの磁場の元は主に電子が作る分子電流である。物質内部の分子電流は普段は別々の方向を向いているが、磁場がかかると一つの向きに揃う。これを磁化という。物質を磁場に対する性質から見たとき、これを磁性体と呼ぶ。外部の磁場と同じ向きに磁場が誘起されるとき常磁性、逆向きのとき反磁性という。誘電体中に磁場がかかると分子電流が向きを揃え、結果的に電流も変化する。この時増加した磁気モーメントを磁化ベクトルと呼び$J$で表すと、誘起される電流密度は、
            \[\rot J = \mu_0i_m\]
        また、分極ベクトルの時間変化によっても電流が発生する。密度$\rho$の電荷が$u$だけ変位したとき、発生する電流は
            \[i_p = \rho\pd[u]{t} = \pd[P]{t}\]
            \[\rot B - \epsilon\mu_0\pd[E]{t} = \mu_0 (i + i_m)\]
        $i_p$を分極電流と呼ぶ。したがってアンペールの法則は、
        \begin{align*}
            \rec{\mu_0}\rot B - \epsilon_0 \pd[E]{t} &= i + i_m + i_p\\
            &= i + \rec{\mu_0}\rot J + \pd[P]{t}\\
            \rec{\mu_0}\rot(B - J) - \pd[(\epsilon E + P)]{t} &= i\\
        \end{align*}
        ここで、
            \[H = \rec{\mu_0}(B - J)\]
        と置くと、
            \[\rot H - \pd[D]{t} = i\]
        となる。多くの物質では$J = \chi_m H$という関係が近似的に成り立つ。$\chi_m$は磁化率と呼ばれる。磁場が一定以上強くなると、分子電流の向きがすべてそろい、それ以上磁化ベクトルが大きくならないため、この比例関係は崩れる。$\mu = \mu_0 + \chi_m$と置くと、$B = \mu H$と表せる。$\mu$を物質の透磁率という。磁性体の場合は$\mu$が正にも負にもなりうる。また$\frac{\mu}{\mu_0}$を比透磁率という。

    よって以下が物質中のマクスウェル方程式である。
    \begin{gather*}
        \dive D = \rho\\
        \dive B = 0\\
        \rot E + \pd[B]{t} = 0\\
        \rot H - \pd[D]{t} =  i
    \end{gather*}