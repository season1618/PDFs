\repart{電磁場の保存量}
    \section{電磁場のエネルギー}
        \subsection{静電場のエネルギー}
            電荷というのは同種の電気が一か所に集まっているので、反発力に逆らって存在して
            いることになる。つまり電荷は運動していなくてもエネルギーを持っている。このときの
            エネルギーというのはポテンシャルエネルギーである。しかし場の理論では、エネルギー
            は物質ではなく空間に蓄えられていると解釈する。つまり、全宇宙のエネルギーは、
            物質の運動エネルギーと場のエネルギーの総和である。\\
            電荷の位置エネルギーを電場の関数で表すために、まず帯電した金属球のエネル
            ギーを求める。半径を$r$、電荷を$q$とする。金属球が帯電していない状態から
            始めて、無限遠から微小電荷を近づけることを考える。なぜこのようなことを考える
            かというと、電荷を近づける際の金属球の反動を無視するためである。電荷が$q'$
            のときの金属表面の電位$\phi$は静電場の式$E = \rec
            {4\pi\epsilon_0}\frac{q'}{r^2}$を積分して、無限遠で0になるように調整
            してやると、
                \[\phi = \frac{q'}{4\pi\epsilon_0 r}\]
            これに微小電荷$dq'$をかけて積分してやれば、
            \begin{align*}
                U(r) = \int_0^q \frac{q'}{4\pi\epsilon_0 r}\\
                = \frac{q^2}{8\pi\epsilon_0 r}
            \end{align*}
            ところで球殻の外側の電場は半径に依らず一定である。よって、半径$r,r+dr$
            の金属球のエネルギーの差をとると、幅$dr$の球殻内の電場は一定とみなせるの
            で、そのエネルギー密度を$u(E)$と置くと、
            \begin{align*}
                U(r+dr)-U(r) &= -u(E)\cdot 4\pi r^2dr\\
                \frac{U(r+dr)-U(r)}{dr} &= -u(E)\cdot 4\pi r^2
            \end{align*}
            微分の定義より、
            \begin{align*}
                u(E) &= -\rec{4\pi r^2}\di[U]{r}\\
                &= -\rec{4\pi r^2}\cdot -\frac{q^2}{8\pi\epsilon_0 r^2}
                = \frac{q^2}{32\pi^2\epsilon_0 r^4}\\
                &= \rec{2}\epsilon_0\lr{\rec{4\pi\epsilon_0}\frac{q}{r^2}}^2\\
                &= \rec{2}\epsilon_0 E^2
            \end{align*}
        \subsection{静磁場のエネルギー}
        \subsection{電磁場のエネルギー}
        運動方程式を変形し、電磁場のエネルギーと同じ形式を作り出す。
            \[m\di[v]{t} = q(E(r) + v\times B(r))\]
        の両辺と$v$との内積を計算し、
        \begin{align*}
            mv\cdot \di[v]{t}
            = E(r)\cdot (qv) + qv\cdot (v\times B(r))\\
            \di{t}\lr{\rec{2}mv^2}
            = E(r)\cdot i(r)
        \end{align*}
        ここで$qv=i(r)$である。また、$v$と$v\times B(r)$は垂直なので、
        $v\cdot (v\times B(r)) = 0$となる。更に多粒子系に拡張すれば、
            \[\di{t}\lr{\sum_j \rec{2}m_jv_j^2} = E\cdot i\]
        右辺はジュール熱を意味する。電場$E$に距離をかけたものが電圧であり、電流密度
        $i$に面積をかけたものが電流だからである。つまりジュール熱の正体は荷電粒子の運
        動エネルギーである。右辺を変形していく。
        \begin{align*}
            i &= \na{rot}H-\pd[D]{t}\\
            E\cdot (\nabla\times H) 
            &= H\cdot (\nabla\times E)-\nabla\cdot (E\times H)\\
            &= -H\cdot\pd[B]{t}-\nabla\cdot (E\times H)
        \end{align*}
        なので、
        \begin{align*}
            E\cdot i &= E\cdot \lr{\na{rot}H-\pd[D]{t}}\\
            &= -E\cdot\pd[D]{t}-H\cdot\pd[B]{t}-\nabla\cdot (E\times H)\\
            &= -\pd{t}\lr{\rec{2}E\cdot D+\rec{2}H\cdot B}
                -\nabla\cdot (E\times H)\\
        \end{align*}
        エネルギー密度$u = \frac{1}{2}(E\cdot D+H\cdot B)$、
        $S = E\times H$とすれば、
            \[-\pd[u]{t} = E\cdot i+\na{div}S\]
        この式はポインティング(Poynting)の定理と呼ばれている。右辺第二項はポインティ
        ングベクトルと呼ばれていて、電磁場のエネルギー流の密度を表している。電磁波の場
        合はその進行方向を指す。
    \section{マクスウェルの応力}
        領域$V$内にある電荷がその外側から受ける力を考える。点電荷の場合のローレンツ
        力の式を少し変えると、単位体積あたりの力は$dF = \rho(r)E(r)+i(r)\times B(r)$
        で表される。また領域内の電磁場は領域内の電荷が作るものと領域外部のものとに
        分けられるが、自己場は作用反作用の法則で打ち消しあうので、電磁場をそのまま
        計算に入れて構わない。
            \[F = \int_V \rho(r)E(r) + i(r)\times B(r)dV\]
        ここで、
        \begin{align*}
            \dive D = \rho\\
            \rot H - \pd[D]{t} = i
        \end{align*}
        を代入すれば、
        \begin{align*}
            F &= \int \llr{E\dive D + \lr{\rot H - \pd[D]{t}}\times B}dV\\
            &= \int \llr{E\dive D - B\times\rot H - \pd[D]{t}\times B}dV
        \end{align*}
        第三項は、
            \[-\pd[D]{t}\times B = -D\times\rot E - \de{t}(D\times B)\]
        と変形できるので、
            \[= \int\lr{\epsilon_0E\dive E - \rec{\mu_0}B\times\rot B
            - \epsilon_0E\times\rot E - \epsilon_0\mu_0\de{t}(E\times H)}dV\]
        $\dive B = 0$なので、対称性を保つために$B\dive B$という項を書き加えると、
        \begin{align*}
            &= \int \llr{\epsilon_0(E\dive E-E\times\rot E)
            + \rec{\mu_0}(B\dive B - B\times\rot B)}dV\\
            & &- \rec{c^2}\de{t}\lr{\int (E\times H)dV}
        \end{align*}
        積分の第二項の$E\times H$はポインティングベクトルである。符号がマイナスになっているのは、
        電磁波が出て行ったことによる反作用を表している。ここで第一項の積分を面積分に変換するため
        に、$E\dive E-E\times\rot E$という量に着目する。そうすることで、荷電粒子の間に直接
        力が働くのではなく、空間の場を通して力が伝わるという解釈が可能になるのだ。$x$成分だけを
        取り出せば、
        \begin{align*}
            (E\dive E - E\times\rot E)_x
            &= E_x\lr{\pd[E_x]{x} + \pd[E_y]{y} + \pd[E_z]{z}}\\
               & &- E_y\lr{\pd[E_y]{x} - \pd[E_x]{y}}\\
               & &\ + E_z\lr{\pd[E_x]{z}-\pd[E_z]{x}}\\
            &= E_x\pd[E_x]{x} - E_y\pd[E_y]{x} - E_z\pd[E_z]{x}\\
               & &+ E_x\pd[E_y]{y} + E_y\pd[E_x]{y}\\
               & &\ + E_x\pd[E_z]{z}+E_z\pd[E_x]{z}\\
            &= \rec{2}\pd[E_x^2]{x} - \rec{2}\pd[E_y^2]{x} - \rec{2}\pd[E_z^2]{x}\\
               & &+ \pd[(E_xE_y)]{y} + \pd[(E_xE_z)]{z}\\
            &= \pd[(E_x^2 - \rec{2}E^2)]{x} + \pd[(E_xE_y)]{y} + \pd[(E_xE_z)]{z}
        \end{align*}
        よって領域内の力は、
        \begin{align*}
            T &= T_e+T_m\\
            T_e &= \epsilon_0\begin{bmatrix}
                  E_x^2-\rec{2}E^2 & E_xE_y & E_xE_z\\
                  E_yE_x & E_y^2-\rec{2}E^2 & E_yE_z\\
                  E_zE_x & E_zE_y & E_z^2-\rec{2}E^2
                  \end{bmatrix}\\
            T_m &= \rec{\mu_0}\begin{bmatrix}
                  B_x^2-\rec{2}B^2 & B_xB_y & B_xB_z\\
                  B_yB_x & B_y^2-\rec{2}B^2 & B_yB_z\\
                  B_zB_x & B_zB_y & B_z^2-\rec{2}B^2
                  \end{bmatrix}
        \end{align*}
        として、
        \begin{align*}
            F &= \int \dive TdV - \rec{c^2}\de{t}\int SdV\\
            &= \int T\cdot dS - \rec{c^2}\de{t}\int SdV
        \end{align*}
        と表される。この$T$をマクスウェルの応力テンソルという。
    \section{電磁場の運動量}
        質量$m$、電荷$q$の粒子の運動方程式は、
            \[m\dd[r]{t} = \int \llr{q\delta(r-r')E
            + q\delta(r-r')\de[r]{t}\times B}dr'\]
        である。多粒子系に拡張すれば、
            \[\sum_i m_i\dd[r_i]{t} = \int \llr{q_i\delta(r_i-r')E
            + q_i\delta(r_i-r')\de[r_i]{t}\times B}dr'\]
        と表せる。ここで、
        \begin{align*}
            \dive D = \sum_i q_i\delta(r_i-r')\\
            \rot H-\pd[D]{t} = \sum_i q_i\delta(r_i-r')\di[r_i]{t}
        \end{align*}
        を代入すれば、右辺は領域$V$内に働く電磁場による力と等しいので、マクスウェルの
        応力を利用すると、
        \begin{align*}
            \sum_i m_i\dd[r_i]{t} = \int T\cdot dS - \rec{c^2}\de{t}\int(E\times H)dV\\
            \di{t}\llr{\sum_i m_i\di[r_i]{t} + \rec{c^2}\int(E\times H)dV}
            = \int T\cdot dS
        \end{align*}
        となる。左辺第二項が電磁場の運動量であると解釈できる。