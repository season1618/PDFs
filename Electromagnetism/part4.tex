\repart{マクスウェル方程式の解}
    \section{電磁波}
        \subsection{波動方程式}
            真空中の電磁波について考える。つまり電荷密度及び電流密度は0とするので
            \begin{align*}
                \rot E + \pd[B]{t} &= 0\\
                \rot B - \epsilon_0\mu_0\pd[E]{t} &= 0\\
                \dive E = 0\\
                \dive B = 0
            \end{align*}
            一番目の式の両辺の$\rot$を計算してやると、
                \[\rot\rot E + \pd{t}(\rot B) = 0\]
            第一項に$\rot\rot X = \grad\dive X - \Delta X$という公式を使えば、
            \begin{align*}
                \grad\dive E - \Delta E + \pd{t}(\rot B) = 0\\
                \lr{\Delta - \epsilon_0\mu_0\pd[^2]{t^2}}E = 0
            \end{align*}
            二番目の式も同様に
                \[\lr{\Delta - \epsilon_0\mu_0\pd[^2]{t^2}}B = 0\]
            となる。この形の微分方程式は波動方程式と呼ばれている。そして波動の速度は
                \[c = \rec{\sqrt{\epsilon_0\mu_0}}\]
            となる。この値が当時の光速の測定値と一致したことは、光が電磁波であることの有力な証拠となった。
        \subsection{平面波}
            平面波は、ダランベールの式を3次元に拡張して
                \[E = F(e\cdot r - ct) + G(e\cdot r + ct)\]
            と表される。$e$は電磁波の進行方向を表す単位ベクトルである。ガウスの法則に代入すると、
            \begin{align*}
                \dive {F(e\cdot r - ct) + G(e\cdot r + ct)}\\
                &= \pd{x}(F_x+G_x) + \pd{y}(F_y+G_y) + \pd{z}(F_z+G_z)\\
                &= e_x(F'_x+G'_x) + e_y(F'_y+G'_y) + e_z(F'_z+G'_z)\\
                &= e\cdot (F'+G') = 0
            \end{align*}
            これはつまり、電場は電磁波の進行方向に対しては変化しないことを意味する。言い換えれば電磁波は横波ということであり、光が電磁波であることのもう一つの有力な証拠となった。次に電場と磁場の関係を導く。電磁波の進行方向を$z$軸に取ると、電場の成分は次のように表される。
                \[(E_x, E_y, E_z) = (F(z-ct) + G(z+ct), 0, 0)\]
            ファラデーの法則に代入すると、
                \[\lr{\pd[B_x]{t}, \pd[B_y]{t}, \pd[B_z]{t}} = (0, -F'(z-ct)-G'(z+ct), 0)\]
            両辺を積分してやれば磁場の形が求められる。このときに現れる積分定数は、静磁場が重なっていても構わないことを示しているので省略する。よって、
                \[(B_x, B_y, B_z) = (0, -\rec{c}F(z-ct) - \rec{c}G(z+ct), 0)\]
            つまり電場と磁場は互いに垂直な方向に存在し、全く同じ形で伝わっていくことになる。
        \subsection{球面波}
            一点から発生またはそこに向かって収束するような球対称な波動を考える。球座標におけるラプラシアンは
                \[\Delta f = \rec{r}\pd[^2]{r^2}(rf)\]
            で表される。$(E_x, E_y, E_z) \rightarrow (E_r, E_\theta, E_\phi)$として、
            \begin{align*}
                \rec{r}\pd[^2]{r^2}(rE) = \rec{c^2}\pd[^2E]{t^2}\\
                \intertext{両辺に$r$を掛けて}
                \pd[^2]{r^2}(rE) = \rec{c^2}\pd[^2]{t^2}(rE)\\
                \intertext{これは$rE$に関する一次元波動方程式なので、}
                E = \rec{r}{F(r - ct) + G(r + ct)}
            \end{align*}
            となる。この式は原点を除いて成り立っている。振幅は波源からの距離に反比例することが分かる。電場と磁場は垂直だが、そのどちらも動径方向と垂直になることはない。$r$と電場が垂直なものをTE球面波、磁場と垂直なものをTM球面波という。
    \section{遅延ポテンシャル}
        ローレンツゲージにおけるマクスウェル方程式を以下の仮定の下で解いた厳密解を遅延ポテンシャルという。
        \begin{itemize}
        	\item 電荷密度$\rho$と電流密度$i$が$r,t$のみに依存する。
        	\item 電磁場の原因は電荷と電流のみである。
        	\item $\rho,i$は無限の過去で0に収束する。
        	\item $\rho,i$は無限遠で0に収束する。
        	\item 自由空間
        	\item 時空因果律が成り立つ。
        \end{itemize}
        これらの仮定により、最初から電磁波が存在している場合や領域の外部から意図しない電磁場が侵入する可能性を排除している。マクスウェル方程式に異なる二つの解があった場合、それらの差は電磁波の解を表すので、最初の仮定だけを考慮したときの一般解は、遅延ポテンシャルに任意の電磁波を重ね合わせたものになる。遅延ポテンシャルは$t_r = t-\frac{|r-r'|}{c}$として、
        \begin{align*}
            \phi(r,t) &= \rec{4\pi\epsilon_0}
            \int \frac{\rho(r',t_r)}{|r-r'|}dr'\\
            A(r,t) &= \frac{\mu_0}{4\pi}
            \int \frac{i(r',t_r)}{|r-r'|}dr'
        \end{align*}
        で与えられる。電荷密度や電流密度の変化はその地点から光速で伝わっていくので、時間変化しない電磁ポテンシャルの解からのずれを考慮した形となっている。遅延ポテンシャルはローレンツ条件を満たしている。ちなみに、遅延ポテンシャルの$t_r = t-\frac{|r-r'|}{c}$の代わりに$t_r = t+\frac{|r-r'|}{c}$としたものは、先進ポテンシャル或いは先進波と呼ばれている。先進ポテンシャルは時間反転に対して対称であることを意味しているが、因果律に反しているため、物理的ではない。遅延ポテンシャルを電磁場の式に代入したものは、$R = r-r'$として、
        \begin{align*}
            E(r,t) &= \rec{4\pi\epsilon_0}
            \int \llr{\frac{\rho(r',t_r)R}{|R|^3}
            +\frac{\rho(r',t_r)R}{c|R|^2}-\frac{i(r',t_r)R}{c^2|R|^2}}dr'\\
            B(r,t) &= \frac{\mu_0}{4\pi}
            \int \llr{\frac{i(r',t_r)R}{|R|^3}
            +\frac{i(r',t_r)R}{c|R|^2}}dr'
        \end{align*}
        となる。これは同じ条件におけるマクスウェル方程式の厳密解であり、ジェフィメンコ方程式と呼ばれている。第一項は近接項で、静電場と同じ形をしている。第二項は放射項で、電磁波による効果を表している。ジェフィメンコ方程式は電荷の保存則を満たすとは限らない。
    \section{リエナール・ヴィーヘルト・ポテンシャル}
        遅延ポテンシャルから運動する点電荷の作る電磁場を求める。点電荷の軌道を$s(t)$とすると、電荷密度と電流密度はそれぞれ次のように表せる。
        \begin{align*}
            \rho(r,t) = q\delta(r-s(t))\\
            i(r,t) = q\dot{s}(t)\delta(r-s(t))
        \end{align*}
        これを先程の遅延ポテンシャルに代入すれば求められる。結果だけ書くと、
        \begin{align*}
            \phi(r,t) &= \frac{q}{4\pi\epsilon_0}
            \frac{1}{|r-s(t_r)|-\beta(t_r)\cdot (r-s(t_r))}\\
            A(r,t) &= \frac{\mu_0q}{4\pi}
            \frac{\dot{s}(t_r)}{|r-s(t_r)|-\beta(t_r)\cdot (r-s(t_r))}
        \end{align*}
        ただし、
        \begin{align*}
            t_r = t-\frac{|r-s(t)|}{c}\\
            \beta(t_r) = \frac{\dot{s}(t_r)}{c}
        \end{align*}
        である。これをリエナール・ヴィーヘルト・ポテンシャルという。
    \section{等速運動する点電荷}
        リエナール・ヴィーヘルト・ポテンシャルを使って、等速運動する点電荷の作る電磁ポテンシャルを求める。点電荷が$x$軸上を速度$v$で進んでおり、時刻$t=0$に原点を通過したとする。その時の電磁ポテンシャルは、
        \begin{align*}
            \phi &= \frac{q}{4\pi\epsilon_0}
            \rec{\sqrt{(x-vt)^2+(1-\frac{v^2}{c^2})(y^2+z^2)}}\\
            A &= \frac{\mu_0q}{4\pi}
            \frac{v}{\sqrt{(x-vt)^2+(1-\frac{v^2}{c^2})(y^2+z^2)}}
        \end{align*}
        となる。点電荷から見た観測点の座標は$(x-vt,y,z)$だから、$x$軸から離れた所では、静止したときよりもポテンシャルが強くなっていることがわかる。等しいポテンシャルの面を考えると、$yz$方向に引き伸ばされた回転楕円体となる。つまり点電荷の立場から見れば、電磁場は同じ形を保ったまま電磁波を発生させることなく移動していく。力学のときと同じように、等速運動にエネルギーは必要ないことがわかる。
    \section{加速運動する点電荷}
        次に点電荷が加速する場合に、放射される電磁波を求める。点電荷の位置を$s(t)$、観測者の位置を$r$とする。$R = |r-s(r)|,n = (r-s(r))/|r-s(r)|$とすれば、ポインティングベクトルは、
            \[S(r,t) = \frac{q^2}{16\pi^2\epsilon_0c}\frac{n}{(1-n\cdot v)^6R^2}[n\times \{(n-v)\times \dot{v}\}]^2\]
        である。出ていく電磁場の方向と強度は、速度と加速度に依存する。\\
        1911年、ラザフォードが原子の中では原子核の周りを電子が回っているという説を提唱した。しかしもしそうだとすると、円軌道を描いて回っている電子からは電磁波が放射され、エネルギーを失って原子核に落ちてしまうであろう。この問題を解決するには量子力学の登場を待たねばならなかった。20世紀初頭に発表されたこのモデルは歴史が浅いにも関わらず、単純で扱いやすいということで化学などでは現在でも利用されている。