\section{光学機器}

\subsection{眼}
	眼の前部には眼を保護する角膜があり、その後ろに光量を調整する虹彩と瞳孔、さらにその後ろにレンズの役割を果たす水晶体がある。眼に入った光は主に角膜で屈折し、水晶体の表面でも多少曲げられる。レンズは両凸レンズで屈折率は中央から離れるにしたがって減少する。これにより球面収差がある程度補正される。水晶体を抜けた後は硝子体を通り網膜で焦点を結ぶ。

	眼のピント調節に関する異常には以下のようなものがある。
	\begin{description}
		\item[近視] 平行光線が網膜の手前で焦点を結ぶ。凹レンズによって補正する。
		\item[遠視] 平行光線が網膜の奥で焦点を結ぶ。凸レンズによって補正する。
		\item[乱視] 光軸を含む面によってpowerが異なる。円筒やトーリック面(トロイダル面)を持ったレンズによって補正する。
	\end{description}
\subsection{光学ガラス}
	レンズやプリズムなどの光学素子に用いられる均一な性質を持ったガラスを光学ガラスという。主に凸レンズ用のクラウンガラスと凹レンズ用のフリントガラスがある。
\subsection{レンズ}
	1839年にシュヴァリエは前方に凹フリント、後方に凸クラウンを張り合わせた色消しレンズを考案した。1840年にはペッツバールが人物写真用にPetzval objective(Petzval lens)を設計した。1841年、対称型レンズには等倍のとき歪曲収差が生じないことが判明した。1866年にダルメイヤーの発明したRapid Rectilinear(aplanat)は45度の画角を持ち多くのカメラに採用された。1900年にゲルツが考案したHypergonは140度の画角を持つ。1933年に発明されたTopogonは

	aplanat球面収差とコマ収差を解消していること。
	stigmat球面収差とコマ収差と非点収差を解消していること。
	anastigmat球面収差、コマ収差、非点収差、像面湾曲を解消していること。
	achromat2色に対して軸上色収差を補正したaplanat
	apochromat3色に対して軸上色収差を補正し、2色についてaplanat
\subsection{カメラ}
	最も単純なカメラは1枚の凸メニスカスレンズを用いたものである。色収差が一切補正されていないため
	写真レンズには以下のようなものがある。画角による分類では
	\begin{description}
		\item[標準レンズ] 
		\item[広角レンズ] 焦点距離が短く広い画角を持つ。
		\item[望遠レンズ] 焦点距離が長く狭い画角を持つ。
	\end{description}
	\begin{description}
		\item[単焦点レンズ] 焦点距離が固定されたもの。
		\begin{description}
			\item[トリプレット] レンズを凸、凹、凸の順に並べたもの。ザイデルの5収差をある程度補正できる。
			\item[テッサー] 
			\item[ヘリアー] 
			\item[クセノター] 
			\item[ダブルガウス] 
			\item[ゾナー] 
			\item[望遠型(テレフォト)] 
			\item[逆望遠型(レトロフォーカス)] 
			\item[反射望遠レンズ] 
		\end{description}
		\item[ズームレンズ] 複数のレンズの配置を変更することで、焦点距離を変更できるもの。
		\item[特殊レンズ]
		\begin{description}
			\item[魚眼レンズ]
		\end{description}
	\end{description}
	
\subsection{屈折望遠鏡}
\subsection{反射望遠鏡}
\subsection{顕微鏡}