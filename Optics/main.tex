\documentclass{jsarticle}
\usepackage{amssymb,amsmath}

\newcommand{\lr}[1]{\left(#1 \right)}
\newcommand{\mlr}[1]{\left\{#1 \right\}}
\newcommand{\llr}[1]{\left[#1 \right]}
\newcommand{\rec}[1]{\frac{1}{#1}}
\newcommand{\de}[2][]{\frac{d #1}{d #2}}
\newcommand{\pd}[2][]{\frac{\partial #1}{\partial #2}}

\newcommand{\grad}{\mathrm{\grad\,}}
\renewcommand{\div}{\mathrm{\div\,}}
\newcommand{\rot}{\mathrm{\rot\,}}

\renewcommand{\(}{\left(}
\renewcommand{\)}{\right)}

\title{光学}
\author{season07001674}
\date{\today (初版)}

\begin{document}
\maketitle
\tableofcontents

\part{幾何光学}
\input geometric-optics-approximation.tex
\input imaging-theory.tex
\input gaussian-optics.tex
\input aberration-theory.tex

\part{波動光学}
\part{量子光学}
\part{フーリエ光学}
\part{非線形光学}
光物性、光エレクトロニクス

\nocite{*}
\bibliographystyle{plain}
\bibliography{refs.bib}

\end{documents}