\reprat{行列力学}

\section{基本原理}
	歴史的には行列力学の方が先に発見された。行列力学は前期量子論にはない零点エネルギーの存在を示した。またパウリは水素スペクトルやシュタルク効果を説明し、共存場を扱えることを示した。
	\begin{enumerate}
		\item 純粋状態は複素ヒルベルト空間の元である状態ベクトルで表される。
		\item オブザーバブルは行列で表される。
		\item ボルンの規則:状態ベクトルのノルムは粒子の存在確率を表す。
		\item 状態ベクトルの時間発展はハイゼンベルクの運動方程式で記述される。
		\item 波束の収縮
	\end{enumerate}
	\subsection{状態ベクトルの性質}
		状態ベクトル$\ket{\psi}$は複素数を成分とする(可算)無限次元ベクトルであり、基底は実関数系であり正規直交基底である。ボルンの規則より状態ベクトルのノルムは1、つまり単位ベクトルである。状態ベクトルのエルミート共役を$\bra{\psi}$と書く。状態ベクトルのノルムが1という性質を保ったまま基底を変換するには変換行列$U$が$U^\dag = U^{-1}$を満たす必要がある。このような$U$をユニタリ行列と呼ぶ。