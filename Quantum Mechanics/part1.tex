\repart{量子力学の発展}

1859 キルヒホッフの法則
1869 ヒットルフ,陰極線の発見
1879 ステファン=ボルツマンの法則
1885 バルマー系列のバルマーの公式
1887 ヘルツ,光電効果の発見
1890 リュードベリ,水素スペクトルの波長の公式
1896 ヴィーンの放射法則
	 ゼーマン効果
1897 トムソン,電子発見
1898 ラザフォード,ベータ線発見
1899 ラザフォード,アルファ線発見
1900 レイリー・ジーンズの公式
	 プランクの法則
	 ヴィラール,ガンマ線発見
1904 トムソンの原子模型
	 長岡半太郎の土星型原子モデル
1905 アインシュタイン,光量子仮説
1906 ライマン系列(m = 1, 紫外線範囲)
1908 パッシェン系列(m = 3)
1911 ラザフォードの原子模型
1913 ボーアの原子模型,ボーアの量子条件
	 シュタルク効果
1914 ライマン系列(m = 1)
1915 ゾンマーフェルトの量子条件
1918 ラザフォード,陽子発見
1922 シュテルン=ゲルラッハの実験
	 ブラケット系列(m = 4)
1923 コンプトン効果
1924 ルイ・ド・ブロイ,物質波
	 パウリの排他原理
	 プント系列(m = 5)
1925 ヴェルナー・ハイゼンベルク,行列力学
1926 エルヴィン・シュレーディンガー,波動力学/シュレーディンガー方程式
	 ボルンの規則
	 クライン・ゴルドン方程式
1927 デイヴィソン=ガーマーの実験
	 不確定性原理
	 波束の収縮
1928 ディラック,相対論的量子力学/ディラック方程式
1932 ジョン・フォン・ノイマン,『量子力学の数学的基礎』
	 チャドウィック,中性子発見
1939 ディラック,ブラケット記法の導入
1990 量子ゼノン効果
\section{熱放射の理論}
	熱放射は物体が放射する電磁波のことである。キルヒホッフは1859年に、ある温度において、同一波長の光の放射エネルギーと吸収率の比は、物体に依らず等しいことを示した。黒体(全ての波長で吸収率が1)の場合には放射エネルギーは温度と波長の関数になる。1879年にヨーゼフ・ステファンは実験から、黒体の放射する全エネルギーは熱力学温度の4乗に比例することを予想した。1884年にはボルツマンが理論的証明を与えている。\footnote{放射圧がマクスウェルによって導出されたのは1871年。}1896年にヴィルヘルム・ヴィーンはボルツマンの思考実験からヴィーンの変位則
	\begin{gather*}
		E_\nu(\nu, T) = \nu^3f(\frac{\nu}{T})\\
		E_\lambda(\lambda, T) = \frac{1}{\lambda^5}g(\lambda T)\\
	\end{gather*}
	を導出した。さらに分子の放射強度は分子の速度のみの関数であると仮定し、マクスウェル・ボルツマン分布から$E_\lambda = g(\lambda)\exp(-\frac{f(\lambda)}{T})$を結論した。ヴィーンの変位則と合わせると結局
	\begin{gather*}
		E_\nu(\nu, T) = \alpha\nu^3\exp(-\frac{\beta\nu}{T})\\
		E_\lambda(\lambda, T) = \frac{c_1}{\lambda^5}\exp(-\frac{c_2}{\lambda T})\\
	\end{gather*}
	となる。ヴィーンの公式を全振動数について積分すればステファン=ボルツマンの法則が導かれる。
	
	ヴィーンの公式は実験的に確認されたが、厳密さに欠ける議論によって導かれていた。マックス・プランクはヴィーンの公式の厳密な導出を提案した。プランクは物体が振動数$\nu$の一次元調和振動子であり、それらは互いに干渉しないという仮定から出発した。電気双極子放射の理論は1889年のヘルツの論文が知られていた。キルヒホッフの法則を仮定すると、一個の振動子のエネルギーを$U(\nu, T)$として
		\[u_\nu = \frac{8\pi\nu^2}{c^3}U(\nu, T)\]
	となる。プランクはこの式とヴィーンの公式から$U(\nu, T) = C\nu\exp(-\frac{\beta\nu}{T})$を得た。熱力学の法則から
		\[\pd[S]{U} = \frac{1}{T} = \frac{\beta\nu}\log(\frac{U}{C\nu})\]
	となる。そこでプランクは一個の振動子のエントロピーを
		\[S = \frac{U}{a\nu}\log(\frac{U}{eb\nu})\]
	で定義した。
	\begin{align*}
		\pd[^2S]{U^2}
		&= \pd{U}\(\frac{1}{a\nu}\log(\frac{U}{eb\nu}) + \frac{1}{a\nu}\)\\
		&= \frac{1}{a\nu U}
	\end{align*}
	ところがヴィーンの公式は振動数の小さい領域では成り立たないことが分かった。1900年にレイリー卿は輻射場が一次元調和振動子からなると仮定し、古典統計力学のエネルギー等分配の法則を適用することでレイリー・ジーンズの公式
	\begin{gather*}
		u_\nu = \frac{8\pi\nu^2kT}{c^3}\\
		u_\lambda = \frac{8\pi kT}{\lambda^4}\\
	\end{gather*}
	を導いた。\footnote{1900年の論文では係数までは書かれていない。}この式は明らかに全エネルギーが発散するので紫外破綻と呼ばれた。プランクはレイリー・ジーンズの公式から先程と同様の議論をした。エネルギー密度と振動子の平均エネルギーの関係とレイリー・ジーンズの公式から$U(\nu, T) = CT$が分かるので、
		\[\pd[S]{U} = \frac{1}{T} = \frac{C}{U}\]
		\[\pd[^2S]{U^2} = -\frac{C}{U^2}\]
	二つの式を折衷して
		\[\pd[^2S]{U^2} = -\frac{ab}{U(U + b)}\]
	とした。
	\begin{align*}
		\frac{1}{T} = \pd[S]{U} = a\log(\frac{U + b}{U})\\
		U = \frac{b}{\exp(\frac{1}{aT}) - 1}\\
		u = \frac{8\pi\nu^2b}{c^3}\frac{1}{\exp(\frac{1}{aT}) - 1}\\
	\end{align*}
	$T$の小さい極限でヴィーンの公式に一致するとすれば
		\[u = \frac{A\nu^3}{\exp(\frac{B\nu}{T}) - 1}\]
	となる。しかし最終的にはプランクはボルツマンの概念を採用し、エネルギーが離散的な値を取ることを仮定した。振動数$\nu$の振動子$N$個からなる系で、全エネルギー$NU$がエネルギー単位の整数倍$Me$なら、系の取りうるエネルギーの分布の総数は
		\[W = \frac{(N + M - 1)!}{(N - 1)!M!} \sim \frac{(N + M)^{N+M}}{N^NP^P}\]
	エントロピーはボルツマンの公式より
	\begin{align*}
		S &= k\log W = k\{(N+M)\log(N+M) - N\log N - M\log M\}\\
		&= k\{N\log(1 + \frac{U}{e}) + M\log(1 + \frac{e}{U})\}\\
		&= kN\{\log(1 + \frac{U}{e}) + \frac{U}{e}\log(1 + \frac{e}{U})\}
	\end{align*}

\section{光電効果}
	ヘルツが光電効果を発見したとき、電子が発見される前だった。1898年にトムソンは光電効果によって放出される粒子が電子と同一であることを示した。1905年にアインシュタインは光がエネルギー$h\nu$の粒子であると仮定することによって説明した。
	この頃の量子概念は個別の現象を説明するために使われたが、1907年にアインシュタインが比熱の理論に応用した。1895年、カール・フォン・リンデによってジュール・トムソン効果を利用した空気の液化が実現すると、低温における比熱の研究が進んだ。1907年にアインシュタインは固体比熱に関するデュロン・プティの法則の破れを量子概念によって説明できることを示した。固体中の全ての原子は同じ振動数で振動しているとすれば、プランクの公式から1molあたりのエネルギーは
		\[E = \frac{3Nh\nu}{\exp(\frac{h\nu}{kT}) - 1}\]
	なので定積モル比熱は
	\begin{align*}
		C_v
		&= \de[E]{T} = -\frac{2Nh\nu}{\llr{\exp{\frac{h\nu}{kT}} - 1}^2}\exp(\frac{h\nu}{kT})-\frac{h\nu}{kT^2}\\
		&= \frac{3R\(\frac{h\nu}{kT}\)^2\exp(\frac{h\nu}{kT})}{\llr{\exp{\frac{h\nu}{kT}} - 1}^2}
	\end{align*}
	となる。この式はアインシュタインの比熱式と呼ばれる。高温では$3R$に等しくなる。この式は実験と一致することが示された。その後デバイやボルンによってより精密な研究がなされた。

\section{線スペクトルの理論と原子模型}
	トムソン模型では電子が準弾性的に束縛される。この性質は、それまでドルーデモデル(1900)、粘弾性を説明するケルビン・フォークトモデル、プランクの量子仮説(1900)、ローレンツモデルの基礎となる仮定だった。当時の物理学者たちはトムソン模型にプランク定数を組み込もうと試み、原子の大きさやバルマー系列を導くことに成功した。しかしトムソン模型はラザフォードの散乱実験やシュタルク効果を説明することが出来なかった。一方でラザフォード模型は定性的なもので、電子の軌道がどのようなものであるかについて言及していない。ボーアが原子の安定性を説明するには、長さの次元を含んだパラメータが必要だった。同じ年にピッカリング系列がイオン化したヘリウムのスペクトル系列であると示し、実験で確認された。ゾンマーフェルトはボーアの量子条件を多自由度の場合に拡張した。変数分離可能な多重周期運動の場合
		\[\cint p_kdq_k = n_kh(n_k = 1, 2, \ldots)\]
	が成り立つ。ゾンマーフェルトの量子条件を水素原子に適用すると、ボーア模型で許される円軌道以外にも主量子数、方位量子数、磁気量子数で特徴付けられる軌道が存在することが分かった。これによって正常ゼーマン効果に一定の説明を与え、相対論的補正を取り入れることで微細構造を説明した。しかしこれらの理論は多電子原子には適用できなかった。
	次第に系の量子論的に扱う方法論が確立した。
	\begin{enumerate}
		\item 古典力学の適用による系の可能な運動の導出。
		\item 実際に許される運動を選択する量子条件
		\item 許された運動間の遷移過程
	\end{enumerate}
ゾンマーフェルトの量子条件
エーレンフェストは断熱定理とゾンマーフェルトの量子条件が一致することを示した。この結果によると水素原子の軌道の離心率に依らずスペクトルは変わらない。ケプラーの第三法則と等しい。量子条件は座標に依存していたが、1916年に正しい座標の選び方がエプスタインとシュバルツシルトによって確立された。彼らはハミルトン=ヤコビ方程式を変数分離することで解けることを示し、古典力学のハミルトン=ヤコビの理論と量子力学を結び付けた。またシュタルク効果を正確に説明することに成功した。同じ年にデバイとゾンマーフェルトは正常ゼーマン効果を説明した。
ステッケルの条件、ベルトラン・ダルブーの定理
1887年にstaudeは自由度2の系に対し、後にstackelは任意の有限自由度の系に対しハミルトン=ヤコビ方程式が変数分離で積分できるような系の運動は多重周期的であることを示した。自由度fならf重フーリエ級数で書ける。libration, rotation ドロネー月の運動の理論、アダムズ潮汐加速
	マクスウェルの電磁気学は原子スペクトルの振動数や強度、偏光状態まで計算することができる。しかしボーアは放射に関するこれらの情報を導出できるとは考えておらず、代わりに対応原理を用いた。1916年にアインシュタインは平衡状態ではカノニカル分布が保たれなければならないことからエネルギー準位間の遷移確率の関係を導いた。アインシュタインの確率概念は暫定的なものであると思われるが、ボーアは確率的に遷移すると考えた。ボーアは対応原理を適用して、アインシュタインの確率と双極子モーメントの多重フーリエ展開の係数を結び付けた。

\section{ゼーマン効果}
	正常ゼーマン効果は発見の直ぐ後にローレンツによって古典的な説明が与えられたが、異常ゼーマン効果については前期量子論が表れるまで未解決だった。
	リュードベリ・リッツの結合原理、リュードベリ補正
	リッツはアルカリ元素のスペクトル系列をsharp系列、principal系列、diffuse系列、fundamental系列に分類し、各系列のエネルギー準位はリュードベリ補正によって$\frac{Rc}{(n+s)^2}$と現した。水素スペクトルとのずれは内部の電子の影響によると考えられた。各系列はそれぞれ方位量子数$l = 1,2,3,4,\ldots$に対応し、$s,p,d,f,\ldots$項と命名された。原子スペクトルの中にはリッツの結合原理を満たさないものがあった。ゾンマーフェルトは仮想の量子数によってこれを説明した。ゾンマーフェルトとランデは電子の角運動量及び磁気モーメントが離散的であると仮定して異常ゼーマン効果を説明した。これによってゾンマーフェルトの想定した量子数に物理的な意味が確立された。しかし実験と合わない部分が出てきたためランデは$g$因子を導入した。
	ボーアの原子模型において、なぜ全ての電子が一番内側の殻に入らないのかという問題が指摘されていた。X線スペクトルなどのデータから周期表の元素についてそれぞれの電子殻に入る電子の個数が調べられた。パウリは4つ目の量子数を導入し、量子数の組が表す状態には一個の電子が入るとした(排他原理)。


熱力学、