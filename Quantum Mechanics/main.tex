\usepackage{amssymb,amsmath}
\usepackage{amsths}
\usepackage{physics}

\newcommand{\repart}[1]{\part{#1}\setcounter{section}{0}}
\newcommand{\lr}[1]{\left(#1 \right)}
\newcommand{\mlr}[1]{\left\{#1 \right\}}
\newcommand{\llr}[1]{\left[#1 \right]}
\newcommand{\de}[2][]{\frac{d #1}{d #2}}
\newcommand{\pd}[2][]{\frac{\partial #1}{\partial #2}}
\newcommand{\ppd}[3][]{\frac{\partial #1}{\partial #2 \partial #3}}

\renewcommand{\(}{\left(}
\renewcommand{\)}{\right)}
\renewcommand{\l[}{\left[}
\renewcommand{\r]}{\right]}

\title{量子力学}
\author{season07001674}
\date{\today (初版 2019/05/19)}

\begin{document}
\maketitle
\tableofcontents

一般に量子状態では物理量は確定せず、観測する度に異なった値が得られる。しかし観測される値は限られており、それぞれの値が得られる確率は確定している。ある系について任意の物理量の確率分布が分かっている状態を純粋状態という。ある物理量が確定した状態をその物理量の固有状態と呼ぶ。純粋状態は固有状態の重ね合わせと考えられる。ある物理量を観測するとそのときの状態の確率分布に従ってある値が選択され測定される。そのとき系の状態は純粋状態からその測定値に関する固有状態へと変化する(波束の収縮)。また二つの物理量について、組合せによってはそれらを同時に確定することはできない(不確定性関係)。系全体の物理量の確率分布は知ることができない場合(多粒子系など)は混合状態と呼び、古典的なある確率でいずれかの純粋状態に確定していると考える。
\end{document}