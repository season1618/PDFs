\repart{軌道計算}

\section{ケプラー方程式}
	ケプラーは万有引力の法則が発見される前『新天文学』(1609)において、ケプラーの第一及び第二法則から惑星の軌道上の位置を決定する方程式を求めている。

	原点を中心とし、焦点が$x$軸上にあるような楕円軌道を考え、離心近点角を$E$とする。惑星の$x, y$座標を考えると、
	\begin{gather*}
		ae + r\cos\theta = a\cos E\\
		r\sin\theta = a\sqrt{1-e^2}\sin E\\
	\end{gather*}
	両辺を$t$で微分すると
	\begin{gather*}
		\dot{r}\cos\theta - r\dot{\theta}\sin\theta = - a\dot{E}\sin E\\
		\dot{r}\sin\theta + r\dot{\theta}\cos\theta = a\sqrt{1-e^2}\dot{E}\cos E\\
	\end{gather*}
	第二式$\times r\cos\theta$ - 第一式$\times r\sin\theta$より
	\begin{align*}
		r^2\dot{\theta}
        &= a\dot{E}( \sqrt{1-e^2}\cos E \cdot r\cos\theta + \sin E \cdot r\sin\theta)\\
		&= a\dot{E}\{ \sqrt{1-e^2}\cos E (a\cos E - ae) + \sin E (a\sqrt{1-e^2}\sin E) \}\\
		&= a^2\sqrt{1-e^2}\dot{E}(1 - e\cos E)
	\end{align*}
	面積速度の関係
        \[\frac{1}{2}r^2\de[\theta]{t} = \frac{\pi a^2\sqrt{1-e^2}}{T}\]
    より
    \begin{align*}
		\dot{E}(1 - e\cos E)
        &= \frac{r^2\dot{\theta}}{a^2\sqrt{1 - e^2}}\\
        &= \frac{2\pi}{T} = n\\
	\end{align*}
	両辺を積分して
		\[E - e\sin E = n(t - t_0) = M\]
	となる。$n$は平均角速度、$t_0$は積分定数で近点通過時刻である。$M$は平均近点角と呼ばれる。

\section{ケプラー方程式の解}
    \subsection{ラグランジュの方法}
        %\newtheorem{laglange}{ラグランジュの定理}
		%\begin{laglange}
			%点$a$を囲む単一閉曲線$C$、$C$内部の領域を$D$とする。$g(z)$を$D$で正則な関数とし、$|z-a|/g(z)$の$C$上の最小値を$\rho$
		%\end{laglange}
			%\[E = -e\sin M + \frac{e^2}{2}\sin 2M - \cdots\]
		%となる。
	\subsection{ベッセル関数による方法}
		$e\sin E$は$M$の周期関数かつ奇関数だから以下のようにフーリエ展開できる。
			\[e\sin E = \sum_{n=1}^\infty A_n\sin nM\]
		係数は
			\[A_n = \frac{2}{\pi}\int_0^\pi e\sin E\sin nM \d M\]
		部分積分して
		\begin{align*}
			A_n &= - \frac{2}{n\pi}[e\sin E\cos nM ]_0^\pi + \frac{2}{n\pi}\int_0^\pi \de[(e\sin E)]{M}\cos nM \d M\\
			&= \frac{2}{n\pi}\int_0^\pi \de[(M + E)]{M}\cos nM \d M\\
			&= \frac{2}{n\pi}\int_0^\pi \cos nM \d M + \frac{2}{n\pi}\int_0^\pi \cos nM \d E
		\end{align*}
		第一項は$\cos$の周期性により消える。
		\begin{align*}
			= \frac{2}{n\pi}\int_0^\pi \cos n(E + e\sin E) \d E
		\end{align*}
		$E$は積分変数だから係数が求められたことになる。$n$次のベッセル関数\footnote{ベッセル関数はフリードリヒ・ヴィルヘルム・ベッセル(1784-1846)がケプラー方程式の解析解を導出する際に導入された。}
			\[J_n(z) = \frac{1}{\pi}\int_0^\pi \cos(z\cos\theta - n\theta)\d \theta\]
		を用いると$\frac{2}{\pi}J_n(-ne)$に等しいので、
			\[E = M + \sum_{n=1}^\infty \frac{2}{n}J_n(-ne)\sin nM\]
	\subsection{コーシーの第一定理}
		平均近点角$M$についての周期$2\pi$の周期関数$S(M)$の複素フーリエ展開
			\[S(M) = \sum_{k=-\infty}^{\infty} c_k \exp(ikM)\]
		を考える。$s = \exp(iE) = \cos E + i\sin E$とおくと、
		\begin{align*}
			\exp(iM) &= \exp(i(E + e\sin E)) = \exp(iE)\exp(e \cdot i\sin E)\\
			&= s\exp\lr{- e \cdot \frac{s - s^{-1}}{2}} = s\exp\lr{- \frac{e(s - s^{-1})}{2}}\\
		\end{align*}
		となる。ここで周期関数$U(M) = \frac{r}{a}\exp\lr{\frac{e(s - s^{-1})}{2}}S(M)$の離心近点角に関する複素フーリエ展開を
			\[U(E) = \sum_{k=-\infty}^{\infty} c_k' \exp(ikE)\]
		とすると
		\begin{align*}
			c_k' 
            &= \frac{1}{2\pi}\int_0^{2\pi} U(E)\exp(-ikE) \d E\\
			&= \frac{1}{2\pi}\int_0^{2\pi} \frac{r}{a}\exp(k\frac{e(s - s^{-1})}{2})S(M)\exp(-ikE) \d E\\
			&= \frac{1}{2\pi}\int_0^{2\pi} \frac{r}{a}\exp(k\frac{e(s - s^{-1})}{2})S(M)\exp(-ikE) \d E\\
			&= \frac{1}{2\pi}\int_0^{2\pi} S(M)\frac{\exp(iE)}{\exp(iM)}^k\exp(-ikE) \frac{r}{a}\d E\\
			&= \frac{1}{2\pi}\int_0^{2\pi} S(M)\exp(ikM) \frac{r}{a}\d E\\
		\end{align*}
		$\de[M]{E} = 1 + e\cos E = \frac{r}{a}$より
		\begin{align*}
			= \frac{1}{2\pi}\int_0^{2\pi} S(M)\exp(ikM) \d M = c_k
		\end{align*}
		となる。\\
		$S = \frac{a}{r}$のとき$U = \exp\lr{i\frac{e(s - s^{-1})}{2}}$はベッセル関数の母関数であり
			\[U = \sum_{k=-\infty}^{\infty} J_k(ke)s^k\]
		なので
			\[\frac{a}{r} = 1 + 2\sum_{i=1}^{\infty}J_i(ie)\cos iM\]
	\subsection{逐次近似}
		離心近点角を以下の漸化式で近似的に求める。
			\[E_0 = M, E_{n+1} = M - e\sin E_n\]
		第$n$項の誤差$\epsilon_n = E_n - E$を評価する。
		\begin{align*}
			\epsilon_{n+1}
            &= E_{n+1} - E = M - e\sin E_n - E\\
			&= e\sin E - e\sin(E + \epsilon_n)\\
			&= - e(\sin E - \sin E\cos \epsilon_n - \cos E\sin \epsilon_n)\\
			&= e\cos E\sin\epsilon_n - e\sin E(1 - \cos\epsilon_n)\\
			&= e\cos E\epsilon_n + O(\epsilon_n^2)
		\end{align*}
		より$E_n$は$E$に収束する。離心率が小さいほど収束は速い。
	\subsection{ニュートン・ラプソン法}
		$f(x) = 0$の近似値を
			\[x_{n+1} = x_n - \frac{f(x_n)}{f'(x_n)}\]
		によって求める。ニュートン・ラプソン法は二次収束であり、$x_n$が$x$に収束するときの誤差は
			\[\epsilon_{n+1} = - \frac{f''(x)}{2f'(x)}\epsilon_n^2\]
		となる。$f(E) = E + e\sin E - M$とおけば
		\begin{gather*}
			E_{n+1} = E_n - \frac{E_n + e\sin E_n - M}{1 - e\cos E_n}\\
			\epsilon_{n+1} = - \frac{e\sin E}{2(1 - e\cos E)}\epsilon_n^2\\
		\end{gather*}
		となる。離心率が大きい場合にはこちらの方が収束が速い。