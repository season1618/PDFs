\section{無限級数}

\begin{align*}
	e^x = \lim_{n \to \infty} \(1 + \frac{x}{n}\)^n\\
\end{align*}

	\[1 - \frac{1}{3} + \frac{1}{5} - \cdots = \frac{\pi}{4} (ライプニッツの公式)\]
各項が0に収束する交代級数なので収束する。
	\[\sqrt{12}\(1 - \frac{1}{3 \cdot 3} + \frac{1}{5 \cdot 3^2} - \frac{1}{7 \cdot 3^3} + \cdots) = \pi\]
各項が0に収束する交代級数なので収束する。

	
	\sqrt{\frac{1}{2}}\sqrt{\frac{1}{2} + \frac{1}{2}\sqrt{\frac{1}{2}}}\sqrt{\frac{1}{2} + \frac{1}{2}\sqrt{\frac{1}{2} + \frac{1}{2}\sqrt{\frac{1}{2}}}} \cdots = \frac{2}{\pi} (1759, ビエト)\\
	\prod_{n = 1}^\infty \(\frac{2n}{2n - 1}\frac{2n}{2n + 1}\) = \frac{\pi}{2} (1655, ウォリス)\\
	\sum_{n = 1}^\infty \frac{1}{n^22^{n-1}} + (\log 2)^2 = \frac{\pi^2}{6} (オイラー)\\
\end{align*}