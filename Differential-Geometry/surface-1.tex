\newpart{曲面}

\section{曲面の表示}
    曲面を滑らかな関数によって
        \[D \ni (u,v) \mapsto p(u,v) = (x(u,v), y(u,v), z(u,v)) \in S\]
    と表す。また便宜上$(u,v)$を$(u_1,u_2)$、$(x,y,z)$を$(x_1,x_2,x_3)$と表すことがある。この曲面が点や曲線に退化しないでかつ、特異点を持たないよう以下の条件を設ける。以下の5つの条件は全て同値である。
    \begin{enumerate}
        \item $D$上に任意の曲線$s(t) = (u(t),v(t))$を与えると、$s'(t) \neq 0$ならば、$\de{t}p(s(t)) \neq 0$である。つまり$D$上の曲線の像はS上の曲線となる。
        \item $S$の点$p$における接ベクトル全体は平面をなす。これを接平面という。
        \item $x,y,z$の内、任意の二つを$x_i,x_j$とすると、ヤコビアン
                  \[\left|\pd[(x_i,x_j)]{(u,v)}\right|\neq 0\]
              である。このとき十分小さい$D$で、写像
                  \[(u,v) \mapsto (x_i,x_j)\]
              は全単射で、逆写像も滑らかである(陰関数定理)。
        \item $x,y,z$の内、任意の二つを$x_i,x_j$とすると、
                  \[dx_i \wedge dx_j \neq 0\]
        \item ベクトル$\pd[p]{u}$と$\pd[p]{v}$は線形独立。
    \end{enumerate}
    曲面を表示するに当たって、正規直交枠とガウス枠を導入する。曲面上の点$p$を原点とし、その点ににおける法線方向を$z$軸とする直交座標系を$(x,y,z)$を取る。それぞれの単位ベクトルを$e_1,e_2,e_3$とする。$e_3$は$p$が曲面上を移動するとき、一定の向きを持つように決める。$e_1,e_2$は$e_1 \times e_2 = e_3$となるように取る。$x$軸と$y$軸は回転の分の自由度を残しておく。$z = f(x,y)$とすれば、$f(0,0) = f_x(0,0) = f_y(0,0) = 0$が成り立つ。$(e_1,e_2,e_3)$を正規直交枠と呼ぶ。一方$B_1 = \pd[p]{u}, B_2 = \pd[p]{v}$で、$n$を$B_1 \times B_2$の方向の単位法線ベクトルとしたとき、$(B_1,B_2,n)$をガウス枠と呼ぶ。

\section{曲面の基本量}
    \subsection{第一基本形式}
        曲面上における二点間の微小長さは、空間内のユークリッド距離によって
        \begin{align*}
            ds^2 &= dp^2\\
            &= \(\pd[p]{u}du + \pd[p]{v}dv\)^2\\
            &= p_u^2du^2 + 2p_up_vdudv + p_v^2dv^2
        \end{align*}
        で与えられる。$g_{ij} = \pd[p]{u_i} \cdot \pd[p]{u_j}$とおいて
        \begin{gather*}
            ds^2 = g_{11}du^2 + 2g_{12}dudv + g_{22}dv^2 \\
            I = 
            \begin{pmatrix}
                g_{11} & g_{12}\\
                g_{21} & g_{22}
            \end{pmatrix}
        \end{gather*}
        これを第一基本形式という。また$I$をリーマン計量(計量テンソル)と呼ぶ。今後はこのような曲面上の計量を考える。

    \subsection{第二基本形式}
        曲面上の点$p(u,v)$とその接平面を考える。点$p(u + du, v + dv)$と接平面の距離は
        \begin{align*}
            dz &= (p(u + du, v + dv) - p(u, v)) \cdot n
            \intertext{二次の項までテイラー展開すると、}
            &= \(\pd[p]{u}du + \pd[p]{v}dv + \frac{1}{2}\pd[^2p]{u^2}du^2 + \frac{\partial^2 p}{\partial u\partial v}dudv + \frac{1}{2}\pd[^2p]{v^2}dv^2\) \cdot n\\
        \end{align*}
        $p_u, p_v \perp n$であることに注意すると、
            \[dz = \(\frac{1}{2}p_{uu}du^2 + p_{uv}dudv + \frac{1}{2}p_{vv}dv^2\) \cdot n\]
        となる。$h_{ij} = \ppd[p]{u_i}{v_i}$とおいて
        \begin{gather*}
            2dz = h_{11}du^2 + 2h_{12}dudv + h_{22}dv^2\\
            II =
            \begin{pmatrix}
                h_{11} & h_{12}\\
                h_{21} & h_{22}
            \end{pmatrix}
        \end{gather*}
        これを第二基本形式という。

        第一基本形式は曲面における計量を表すという意味で内在的であり、第二基本形式は曲面の外部の空間への入り方を表しているという意味で外在的である。

\section{曲面の曲率}
    \subsection{測地的曲率と法曲率}
        曲面上の曲線を弧長パラメータによって$p(u,v) = p(u(s),v(s))$と表し、正規直交枠$(e_1,e_2,e_3)$を導入する。$e_1$を曲線の接線方向、$e_3$を曲面の法線方向の単位ベクトルとすると、$p''(s) = e_1' = k_ge_2 + k_ne_3$と表すことができる。このとき$k_g, k_n$をそれぞれ測地的曲率、法曲率という。$|p''|$は空間曲線としての曲率$k(s)$である。ここで、
        \begin{align*}
            p' &= p_u\de[u]{s} + p_v\de[v]{s}\\
            p'' &= \de[p_u]{s}\de[u]{s} + p_u\de[^2u]{s^2} + \de[p_v]{s}\de[v]{s} + p_v\de[^2v]{s^2}\\
            &= p_{uu}\(\de[u]{s}\)^2 + 2p_{uv}\de[u]{s}\de[v]{s} + p_{vv}\(\de[v]{s}\)^2 + p_u\de[^2u]{s^2} + p_v\de[^2v]{s^2}\\
        \end{align*}
        なので$p_u \cdot e_3 = p_v \cdot e_3 = 0$より、
        \begin{align*}
            k_n &= p'' \cdot e_3\\
            &= (p_{uu} \cdot e_3)\(\de[u]{s}\)^2 + 2(p_{uv} \cdot e_3)\de[u]{s}\de[v]{s} + (p_{vv} \cdot e_3)\(\de[v]{s}\)^2\\
            &= h_{11}\(\de[u]{s}\)^2 + 2h_{12}\de[u]{s}\de[v]{s} + h_{22}\(\de[v]{s}\)^2\\
            &= \frac{h_{11}du^2 + 2h_{12}dudv + h_{22}dv^2}{g_{11}du^2 + 2g_{12}dudv + g_{22}dv^2}\\
        \end{align*}
        となる。つまり曲線上の点$p$における法曲率は$\lambda = dv/du$のみに依存する。

    \subsection{主曲率}
        曲面上の点において、法線ベクトルを含む平面と曲面が交わってできる曲線を法切断または直截線という。点$p$における法切断の測地的曲率は0なので、曲率は法曲率に等しい。法切断の曲率が最大値最小値をとるとき、その曲率$k_1,k_2$を主曲率、接ベクトルを主方向$X_1,X_2$と呼ぶ。法曲率が$k$となる条件は、
        \begin{gather*}
            k = \frac{h_{11} + 2h_{12}\lambda + h_{22}\lambda^2}{g_{11} + 2g_{12}\lambda + g_{22}\lambda^2}\\
            (h_{22} - kg_{22})\lambda^2 + 2(h_{12} - kg_{12})\lambda + (h_{11} - kg_{11}) = 0
        \end{gather*}
        の解が存在することであり、この二次方程式の判別式$D$が0以上になることである。
        \begin{align*}
            \frac{D}{4} &= (h_{12} - kg_{12})^2 - (h_{11} - kg_{11})(h_{22} - kg_{22})\\
            &= (g_{12}^2 - g_{11}g_{22})k^2 + (g_{11}h_{22} - 2g_{12}h_{12} + g_{22}h_{11})k + (h_{12}^2 - h_{11}h_{22}) \geq 0
        \end{align*}
        $g$は正定値行列なので$g_{12}^2 - g_{11}g_{22} < 0$である。よって$k$が極値を取るのは$D = 0$のときである。
        \begin{align*}
            (h_{11} - kg_{11})(h_{22} - kg_{22}) - (h_{12} - kg_{12})(h_{21} - kg_{21}) = 0\\
            \det(II - kI) = 0\\
            \det(I^{-1}II - kE) = 0\\
        \end{align*}
        となる。これは$I^{-1}II$の固有方程式である。$S = I^{-1}II$をシェイプ作用素という。つまり二つの主曲率はシェイプ作用素の固有値である。

        一般の方向における法切断の曲率を求める。
        \begin{lem}
            平面曲線$y = y(x)$で$y'(0) = 0$のとき、原点における曲率は、$y''(0)$である$($平面曲線の場合、曲率に符号を付ける$)$。
        \end{lem}
        \begin{proof}
            $p = (x,y)$、弧長パラメータを$s$とすれば曲率は$\de[p]{s}$である。
                \[\de[p]{x} = (1,y'), \de[s]{x} = \sqrt{1 + y'^2}\]
            より、
            \begin{align*}
                \de[p]{s} &= \frac{(1, y')}{\sqrt{1 + y'^2}}\\
                \de{x}\de[p]{s} &= \frac{(0, y'')\sqrt{1 + y'^2} - (1, y')y'y''/\sqrt{1 + y'^2}}{1 + y'^2}\\
            \end{align*}
            $y'(0) = 0$なので
                \[\de[p]{s} = \frac{(0, y'')(1 + y'^2) - (1, y')y'y''}{(1 + y'^2)^2} = (0,y'')\]
            よって原点における曲率は$y''$となる。
        \end{proof}

        \begin{thm}[オイラーの定理(微分幾何)]
            二つの主方向は直交し、更に$X_1$に対して角$\theta$をなす法切断の曲率を$k_\theta$とすれば、
                \[k_\theta = k_1\cos^2\theta + k_2\sin^2\theta\]
            である。
        \end{thm}
        \begin{proof}
            曲面上に正規直交枠をとる。補題より、$x,y$軸の法切断の曲率は偏微分となり$\pd[^2z]{x^2},\pd[^2z]{y^2}$である。$x$軸と角$\theta$をなす方向を$v = (\cos\theta, \sin\theta)$とすれば、曲率$k_\theta$は方向微分となり、
            \begin{align*}
                k_\theta &= \pd[^2z]{v^2}\\
                &= \pd{v}(z_x\cos\theta + z_y\sin\theta)\\
                &= (z_{xx}\cos\theta + z_{xy}\sin\theta, z_{xy}\cos\theta + z_{yy}\sin\theta) \cdot (\cos\theta, \sin\theta)\\
                &= z_{xx}\cos^2\theta + 2z_{xy}\sin\theta\cos\theta + z_{yy}\sin^2\theta
            \end{align*}
            この二次形式の最大値と最小値は行列$\begin{pmatrix}z_{xx} & z_{xy}\\ z_{yx} & z_{yy}\end{pmatrix}$の固有値であり、主方向はそれぞれの固有値の固有ベクトルである。対称行列の固有ベクトルは直交するので主方向も直交する。この固有ベクトルで主軸変換を行えば、主方向との成す角を新たに$\theta$とおいて
                \[k_\theta = k_1\cos^2\theta + k_2\sin^2\theta\]
        \end{proof}

    \subsection{ガウス曲率と平均曲率}
        二つの主曲率$k_1,k_2$の積をガウス曲率$K$、平均を平均曲率$H$という。主曲率の満たす二次方程式の解と係数の関係より、
        \begin{align*}
            K &= k_1k_2 = \det S = \frac{h_{11}h_{22} - h_{12}^2}{g_{11}g_{22} - g_{12}^2}\\
            2H &= k_1 + k_2 = \tr S = \frac{g_{11}h_{22} - 2g_{12}h_{12} + g_{22}h_{11}}{g_{11}g_{22} - g_{12}^2}\\
        \end{align*}
        となる。ガウス曲率が正の点を楕円点、負の点を双曲点という。また0の点を放物点と呼ぶこともある。