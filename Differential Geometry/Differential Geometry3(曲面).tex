\repart{曲面}
    \section{曲面の表示}
        曲面は2次元なので、二つのパラメータ$u,v$による直交座標での写像
            \[D\ni (u,v)\longmapsto p(u,v) = (x(u,v),y(u,v),z(u,v))\in S\]
        で与えられる。$p(u,v)$は滑らかであるとする。また便宜上$(u,v)$を
        $(u_1,u_2)$、$(x,y,z)$を$(x_1,x_2,x_3)$と表すことがある。曲線の場合
        と同様に、この曲面が点や曲線に退化しないよう、また特異点を持たないよう条件を
        設ける必要がある。以下の5つの条件は全て同等である。
        \begin{enumerate}
            \item D上に任意の曲線$s(t)=(u(t),v(t))$を与えると、$s'(t)\neq 0$
                  ならば、$\di{t}p(s(t))\neq 0$である。つまりD上の曲線の像
                  はS上の曲線となる。\\
            \item Sの点pにおける接ベクトル全体は平面をなす。これを接平面という。\\
            \item $x,y,z$の内、任意の二つを$x_i,x_j$とすると、ヤコビアン
                      \[\left|\pd[(x_i,x_j)]{(u,v)}\right|\neq 0\]
                  である。このとき、十分小さいDで、写像
                      \[(u,v)\longmapsto (x_i,x_j)\]
                  は全単射で、逆写像も滑らかである(微分積分学の陰関数定理)。
            \item $x,y,z$の内、任意の二つを$x_i,x_j$とすると、
                      \[dx_i\wedge dx_j\neq 0\]
            \item ベクトル$\pd[p]{u}$と$\pd[p]{v}$は線形独立である。
        \end{enumerate}
        曲面を表示するに当たって、正規直交枠とガウス標構を導入する。曲面上の点$p$
        を原点とし、その点ににおける法線方向を$z$軸とする直交座標系を$(x,y,z)$と
        する。それぞれの単位ベクトルを$e_1,e_2,e_3$とする。$e_3$は点$p$が曲面上
        を移動するとき、いつも決まった向きを持つように決めておく。$e_1,e_2$は$e_1\times e_2 = e_3$
        となるようにとる。また、$x$軸と$y$軸は回転の分の自由度を残しておく。$z=f(x,y)$
        とすれば、必然的に$f(0,0)=f_x(0,0)=f_y(0,0)=0$が成り立つ。$(e_1,e_2,e_3)$
        を正規直交枠と呼ぶ。一方$B_1=\pd[p]{u},B_2=\pd[p]{v}$で、$n$を$B_1\times B_2$
        の方向をとる単位法線ベクトルとしたとき、$(B_1,B_2,n)$をガウス標構と呼ぶ。


    \section{曲面の基本量}
        \subsection{第一基本形式}
            曲面上における二点間の微小長さは、
            \begin{eqnarray*}
                ds^2 &=& dp^2\\
                &=& \lr{\pd[p]{u}du+\pd[p]{v}dv}^2\\
                &=& p_u^2du^2+2p_up_vdudv+p_v^2dv^2
            \end{eqnarray*}
            で与えられる。$g_{ij}=\pd[p]{u_i}\cdot \pd[p]{u_j}$とおき、
            \begin{gather*}
                ds^2 = g_{11}du^2+2g_{12}dudv+g_{22}dv^2 \\
                I = \begin{pmatrix}
                    g_{11} & g_{12}\\
                    g_{21} & g_{22}
                   \end{pmatrix}
            \end{gather*}
            これを第一基本形式という。またIをリーマン計量(計量テンソル)と呼ぶ。曲面
            上の二点間の距離はもちろんユークリッド距離だが、ここで重要なのはむしろ$u,v$
            の方である。距離に合わせて平面を曲げるのではなく、距離に合わせて計量を
            定義する、というのが曲面の外在的な定義を必要としないリーマン幾何学の特
            徴である。

        \subsection{第二基本形式}
            曲面上の点$p(u,v)$とその接平面を考える。$p(u+du,v+dv)$と接平面
            の距離は
            \begin{align*}
                dz = (p(u+du,v+dv)-p(u,v))\cdot n\\
                \intertext{二次の項までテイラー展開すると、}
                = \lr{\pd[p]{u}du+\pd[p]{v}dv
                +\rec{2}\pd[^2p]{u^2}du^2
                +\frac{\partial^2 p}{\partial u\partial v}dudv
                +\rec{2}\pd[^2p]{v^2}dv^2}\cdot n\\
                \intertext{$p_u,p_v\perp n$であることに注意すると、}
                = \lr{\rec{2}p_{uu}du^2+p_{uv}dudv+\rec{2}p_{vv}dv^2}\cdot n
            \end{align*}
            となる。$h_{ij}=\frac{\partial^2p}{\partial u_i\partial u_j}\cdot n$
            とおき、
            \begin{gather*}
                2dz = h_{11}du^2+2h_{12}dudv+h_{22}dv^2 \\
                II = \begin{pmatrix}
                    h_{11} & h_{12}\\
                    h_{21} & h_{22}
                   \end{pmatrix}
            \end{gather*}
            これを第二基本形式という。\\
            第一基本形式は曲面の計量に関する性質を表すという意味で内在的であり
            、第二基本形式は凹凸などの空間への入り方を表しているという意味で外在
            的である。
            
            
    \section{曲面の曲率}
        \subsection{測地的曲率と法曲率}
            曲面上の曲線を$p(u,v)=p(u(s),v(s))$とする。$s$は弧長パラメータである。こ
            れに正規直交枠$(e_1,e_2,e_3)$を導入する。$e_1$を曲線の接線方向、$e_3$
            を曲面の法線方向の単位ベクトルとすると、$p''(s) = e_1' = k_ge_2+k_ne_3$
            と表すことができる。この時の$k_g,k_n$をそれぞれ測地的曲率、法曲率という。つまり
            $k_g=p''\cdot e_2,k_n=p''\cdot e_3$である。$|p''|$は空間曲線としての曲率$k(s)$である。ここで、
            \begin{eqnarray*}
                p' &=& p_u\di[u]{s}+p_v\di[v]{s}\\
                p'' &=& \di[p_u]{s}\di[u]{s}+p_u\dd[u]{s}
                + \di[p_v]{s}\di[v]{s}+p_v\dd[v]{s}\\
                &=& p_{uu}\lr{\di[u]{s}}^2 + 2p_{uv}\di[u]{s}\di[v]{s} + p_{vv}\lr{\di[v]{s}}^2
                + p_u\dd[u]{s} + p_v\dd[v]{s}\\
            \end{eqnarray*}
            なので$p_u\cdot e_3=p_v\cdot e_3=0$より、
            \begin{eqnarray*}
                k_n &=& p''\cdot e_3\\
                &=& (p_{uu}\cdot e_3)\lr{\di[u]{s}}^2
                + 2(p_{uv}\cdot e_3)\di[u]{s}\di[v]{s}
                + (p_{vv}\cdot e_3)\lr{\di[v]{s}}^2\\
                &=& h_{11}\lr{\di[u]{s}}^2+2h_{12}\di[u]{s}\di[v]{s}+h_{22}\lr{\di[v]{s}}^2\\
                &=&\frac{h_{11}du^2+2h_{12}dudv+h_{22}dv^2}
                {g_{11}du^2+2g_{12}dudv+g_{22}dv^2}\\
            \end{eqnarray*}
            となる。曲線上の点$p$における法曲率は$\lambda = dv/du$のみに依存するこ
            とが分かった。


        \subsection{主曲率}
            曲面上の点において、法線ベクトルを含む平面と曲面が交わってできる曲線を法
            切断または直截線という。点$p$における法切断の測地的曲率は0なので、曲率
            は法曲率に等しい。法切断の曲率が最大値最小値をとるとき、その曲率$k_1,k_2$
            を主曲率、接ベクトルを主方向$X_1,X_2$と呼ぶ。法曲率$k$が極値を持つ条
            件は、
            \begin{gather*}
                k = \frac{h_{11}+2h_{12}\lambda+h_{22}\lambda^2}
                {g_{11}+2g_{12}\lambda+g_{22}\lambda^2}\\
                (h_{22}-kg_{22})\lambda^2
                + 2(h_{12}-kg_{12})\lambda
                + (h_{11}-kg_{11}) = 0
            \end{gather*}
            の解が存在することであり、この二次方程式の判別式$D$が0以上になることである。
            \begin{align*}
                \frac{D}{4} &= (h_{12}-kg_{12})^2 - (h_{11}-kg_{11})(h_{22}-kg_{22})\\
                &= (g_{12}^2-g_{11}g_{22})k^2
                + (g_{11}h_{22}-2g_{12}h_{12}+g_{22}h_{11})k
                + (h_{12}^2-h_{11}h_{22}) \geqq 0
            \end{align*}
            $g$は正定値行列なので、$g_{12}^2-g_{11}g_{22}$は負の値をとる。$D=0$
            になるとき$k_n$は最大または最小になる。ところで$D=0$は、
            \begin{align*}
                (h_{11}-kg_{11})(h_{22}-kg_{22})
                -(h_{12}-kg_{12})(h_{21}-kg_{21}) = 0
            \end{align*}
            と書ける。これは$II-kI$の行列式が0、即ち$I^{-1}II$の固有方程式を表してい
            る。$I^{-1}II$をシェイプ作用素という。つまり二つの主曲率はシェイプ作用素の固
            有値である。


        \subsection{ガウス曲率と平均曲率}
            二つの主曲率$k_1,k_2$の積をガウス曲率$K$、平均を平均曲率$H$という。主曲
            率の満たす二次方程式の解と係数の関係より、
            \begin{eqnarray*}
                K &=& k_1k_2 = \frac{h_{11}h_{22}-h_{12}^2}{g_{11}g_{22}-g_{12}^2}\\
                2H &=& k_1+k_2 = \frac{g_{11}h_{22}-2g_{12}h_{12}+g_{22}h_{11}}
                {g_{11}g_{22}-g_{12}^2}\\
            \end{eqnarray*}
            となる。ガウス曲率が正の点を楕円点、負の点を双曲点という。また0の点を放物点と
            呼ぶこともある。


        \subsection{オイラーの定理(微分幾何)}
            ここでは微分幾何におけるオイラーの定理を証明する。まず補題を一つ証明する。
            \begin{lemma}
                平面曲線$y=y(x)$で$y'(0)=0$のとき、原点における曲率は、$y''(0)$
                である$($平面曲線の場合、曲率に符号を付ける$)$。
            \end{lemma}

            \begin{proof}
                $p=(x,y)$、弧長パラメータを$s$とすれば曲率は$\dd[p]{s}$である。
                    \[\di[p]{x} = (1,y'),\di[s]{x} = \sqrt{1+y'^2}\]
                より、
                \begin{gather*}
                    \di[p]{s} = \frac{(1,y')}{\sqrt{1+y'^2}}\\
                    \di{x}\di[p]{s} = \frac{(0,y'')\sqrt{1+y'^2}
                    -(1,y')y'y''/\sqrt{1+y'^2}}{1+y'^2}\\
                    \intertext{$y'(0)=0$なので、}
                    \dd[p]{s} = \frac{(0,y'')(1+y'^2)-(1,y')y'y''}{(1+y'^2)^2}
                    = (0,y'')
                \end{gather*}
                よって原点における曲率は、$y''$となる。
            \end{proof}

            \begin{euler}
                二つの主方向は直交し、更に$X_1$に対して角$\theta$をなす法切断の
                曲率を$k_\theta$とすれば、
                    \[k_\theta = k_1\cos^2\theta + k_2\sin^2\theta\]
                である。
            \end{euler}

            \begin{proof}
                曲面上に正規直交枠をとる。補題より、$x,y$軸の法切断の曲率は偏微分
                となり$\pd[^2z]{x^2},\pd[^2z]{y^2}$である。$x$軸と角$\theta$
                をなす方向を$v=(\cos\theta,\sin\theta)$とすれば、曲率$k_\theta$
                は方向微分となり、
                \begin{eqnarray*}
                    k_\theta &=& \pd[^2z]{v^2}\\
                    &=& \pd{v} (z_x\cos\theta + z_y\sin\theta)\\
                    &=& (z_{xx}\cos\theta + z_{xy}\sin\theta,
                    z_{xy}\cos\theta + z_{yy}\sin\theta)
                    \cdot(\cos\theta,\sin\theta)\\
                    &=& z_{xx}\cos^2\theta + 2z_{xy}\sin\theta\cos\theta + z_{yy}\sin^2\theta
                \end{eqnarray*}
                この二次形式の最大値と最小値は行列
                $\begin{pmatrix}z_{xx} & z_{xy}\\ z_{yx} & z_{yy}\end{pmatrix}$
                の固有値に等しい。また主方向はそれぞれの固有値の固有ベクトルである。対
                称行列の固有ベクトルは直交するので主方向も直交する。この固有ベクトルで
                主軸変換を行えば、(主方向との成す角を新たに$\theta$とおいて)
                    \[k_\theta = k_1\cos^2\theta + k_2\sin^2\theta\]
                よって示された。
            \end{proof}
            
            
    \section{曲面の分類と驚異の定理}
        この章では曲線と同じように曲面を分類することを試みる。\\
        まずChristoffel記号を導入する。
        \begin{eqnarray*}
            [jk,l] &=& \rec{2}\lr{\pd[g_{kl}]{u^j}+\pd[g_{jl}]{u^k}
            -\pd[g_{jk}]{u^l}}\\
            \chr{i}{jk} &=& \rec{2}g^{ih}\lr{\pd[g_{jh}]{u^k}+\pd[g_{kh}]{u^j}
            -\pd[g_{jk}]{u^h}}
        \end{eqnarray*}
        ここで$[jk,l]=g_{il}\chr{i}{jk},\chr{i}{jk}=g^{ih}[jk,h]$
        が成り立っている。$[jk,l],\chr{i}{jk}$をそれぞれ第一種及び第二種の
        Christoffel記号という。


        \subsection{Gauss-Weingardenの方程式}
            曲線論におけるFrenet-Serretの公式に当たるものを考える。Gauss標構
            $\{B_1,B_2,n\}$をとる。ここで
                \[n = \frac{B_1\times B_2}{|B_1\times B_2|}
                =\frac{B_1\times B_2}{\sqrt{g_{11}g_{22}-g_{12}^2}}\]
            である。三つのベクトルは線形独立なので、
                \[\pd[B_j]{u^k} = \Gamma_{jk}^iB_i + h_{jk}n\]
            となる$\Gamma_{jk}^i,h_{jk}$が一意に定まるはずである。
            $h_{jk}=\pd[B_j]{u^k}\cdot n$は定義より第二基本量である。また
            $g_{ij}=B_i\cdot B_j$を微分した$\pd[B_i]{u^k}\cdot B_j
            +B_i\cdot \pd[B_j]{u^k} = \pd[g_{ij}]{u^k}$に代入して、
                \[\Gamma_{ik}^hg_{hj}+\Gamma_{jk}^hg_{ih} = \pd[g_{ij}]{u^k}\]
            を得る。$\Gamma_{jk|i}=\Gamma_{jk}^hg_{ih}$と置けば、
                \[\Gamma_{ik|j}+\Gamma_{jk|i} = \pd[g_{ij}]{u^k}\]
            となる。$\pd[B_j]{u^k}=\ppd{p}{u^j}{u^k}$より$j,k$につき対称で
            ある。そこで上の式の添え字を循環的に入れ替えて、$\Gamma_{jk|i}$に関
            する連立方程式とみれば、
            \begin{eqnarray*}
                \Gamma_{jk|i} &=& \rec{2}\lr{\pd[g_{ki}]{u^j}+\pd[g_{ij}]{u^k}
                -\pd[g_{jk}]{u^i}} = [jk,i]\\
                \Gamma_{jk}^i &=& g^{ih}\Gamma_{jk|h} = \chr{i}{jk}\\
            \end{eqnarray*}
            となることが分かる。次に$n\cdot n=1$を微分すると$\pd[n]{u^j}\cdot n = 0$
            であるので、
                \[\pd[n]{u^k} = r_k^hB_h\]
            となる$r_j^h$が一意的に決まる。$B_i\cdot n=0$を微分した
            $\pd[B_i]{u^k}\cdot n+B_i\cdot \pd[n]{u^k}=0$に先程の式と
            代入して、
                \[h_{ij}+r_j^hg_{ih} = 0\]
            これは行列の積を成分ごとに表したものなので、両辺に右から$g$の逆行列を
            かけて、
                \[r_k^i = -h_{kl}g^{li}\]
            二つを合わせて、
            \begin{align*}
                \pd[B_j]{u^k} &= \chr{i}{jk}B_i + h_{jk}n \tag{Gaussの方程式}\\
                \pd[n]{u^k} &= -h_{kl}g^{li}B_i \tag{Weingartenの方程式}
            \end{align*}


        \subsection{Gauss-Codazziの積分可能条件}
            Frenet-Serretの式と同じように、Gauss-Weingartenの方程式は連立
            一次偏微分方程式である。係数は第一及び第二基本量から求められ、この
            方程式を解くことで$\{B_1,B_2,n\}$を求めることができる。しかしここで
            $B_j=\pd[p]{u^j}$なので、求められたガウス標構が実際の曲面と矛盾し
            ないためにはいくつか条件が必要である。具体的には、
            \begin{eqnarray*}
                \ppd{p}{u^j}{u^i} &=& \pd{p}{u^i}{u^j}\\
                \ppd{B_i}{u^k}{u^j} &=& \pd{B_i}{u^j}{u^k}\\
                \ppd{n}{u^j}{u^i} &=& \ppd{n}{u^i}{u^j}
            \end{eqnarray*}
            である。第一式に関しては既に成り立っていることがわかる。なぜなら
                \[\ppd{p}{u^j}{u^i} = \pd[B_i]{u^j}
                = \chr{h}{ij}B_h + h_{ij}n\]
            は$i,j$に関して対称だからである。次に第二式の条件を求める。
            \begin{align*}
                \ppd{B_i}{u^k}{u^j}
                &= \pd{u^k}\lr{\chr{h}{ij}B_h + h_{ij}n}\\
                &= \pd{u^k}\chr{h}{ij}B_h + \chr{h}{ij}\pd[B_h]{u^k}
                + \pd[h_{ij}]{u^k}n + h_{ij}\pd[n]{u^k}\\
                \intertext{$\pd[B_h]{u^k}$と$\pd[n]{u^k}$にそれぞれ代入
                し、}
                & = \sum_h\pd{u^k}\chr{h}{ij}B_h
                + \sum_{h,l}\chr{h}{ij}\chr{l}{hk}B_l + \sum_h\chr{h}{ij}h_{hk}n\\
                &\qquad + \pd[h_{ij}]{u^k}n+h_{ij}\sum_{l,h}(-h_{kl}g^{lh}B_h)\\
                \intertext{第二項で添え字$h,l$を付け替えると、}
                &= \llr{\pd{u^k}\chr{h}{ij}
                +\chr{h}{kl}\chr{l}{ij} - h_{ij}h_{kl}g^{lh}}B_h\\
                &\qquad + \llr{\pd[h_{ij}]{u^k} + \chr{h}{ij}h_{hk}}n\\
            \end{align*}
            となる。これと$j,k$を入れ替えたものが等しくなる。また$B_h,n$は線形独
            立なので、
            \begin{align*}
                R^h_{ijk} = h_{ik}h_{jl}g^{lh} - h_{ij}h_{kl}g^{lh}
                \tag{Gaussの積分可能条件}\\
                \pd[h_{ij}]{u^k}-\pd[h_{ik}]{u^j}
                +\chr{h}{ij}h_{hk}-\chr{h}{ik}h_{hj} = 0
                \tag{Codazziの積分可能条件}
            \end{align*}
            となる。ただし
                \[R^h_{ijk} = \pd{u^j}\chr{h}{ik} - \pd{u^k}\chr{h}{ij}
                + \chr{l}{ik}\chr{h}{jl} - \chr{l}{ij}\chr{h}{kl}\]
            である。また両辺に$g^{hl}$をかけて縮約を取ったものを、
                \[R_{lijk} = g^{hl}R^h_{ijk}
                = \pd{u^j}\chr{h}{ik}g^{hl} - \pd{u^k}\chr{h}{ij}g^{hl}
                + \chr{a}{ik}\chr{h}{ja}g^{hl} - \chr{a}{ij}\chr{h}{ka}g^{hl}\]
            これらをリーマン曲率テンソルと呼ぶ。リーマン曲率テンソルを用いてGaussの積
            分可能条件を表すと、
            \begin{eqnarray*}
                g^{ha}R^h_{ijk} &=& h_{ik}h_{ja}g^{ah}g^{hl} - h_{ij}h_{ka}g^{ah}g^{hl}\\
                R_{lijk} &=& h_{ik}h_{ja}\delta_l^a - h_{ij}h_{ka}\delta_l^a\\
                &=& h_{ik}h_{jl} - h_{ij}h_{kl}
            \end{eqnarray*}
            また第三式については、Codazziの積分可能条件と同じものが導かれるので
            、条件はこれで十分である。よって次の定理が成り立つ。
            \begin{surface}
                対称テンソル$g_{ij},h_{ij}$が与えられ、$g_{ij}$は正定値で
                あるとする。このとき、これらを第一及び第二基本量とする曲面$p(u,v)$
                が存在するための必要十分条件は、Gauss-Codazziの積分可能条
                件を満たすことである。またその曲面は、剛体運動を除いてただ一つに決
                定する。
            \end{surface}


        \subsection{驚異の定理}
            上の式は右辺が第二基本量のみに依存している。さて、元々リーマン曲率テン
            ソル及びその定義に含まれるクリストッフェル記号は第一基本量から求められ
            たのだった。つまり第一基本テンソルと第二基本テンソルは独立ではなく一部依
            存しているところがあるということである。特に
                \[R_{1212} = h_{11}h_{22}-h_{12}^2\]
            であり、これを用いると、
                \[K = \frac{h_{11}h_{22}-h_{12}^2}{g_{11}g_{22}-g_{12}^2}
                = \frac{R_{1212}}{g_{11}g_{22}-g_{12}^2}\]
            のように曲面のガウス曲率を第一基本テンソルだけから求めることができる。こ
            れを驚異の定理(Theorem Egregium)という。元々外在的な量から定義
            されたガウス曲率が、内在的な量のみから決定できることは、外の世界につい
            ての情報を全く必要とせず、空間の曲がり方を考察することができるという可能
            性を示唆している。この事実は、第一基本テンソルの等しい多様体を同一視す
            るリーマン幾何学誕生のきっかけとなった。


        \paragraph{等長写像}
            二点間の距離を保つ写像を等長写像という。二つの曲面が等長写像によって
            移りあうとき、それらは局所等長的であるという。局所等長的な二つの曲面は、
            伸縮したりせず連続的に折り曲げるだけで変形させることができる。また等長変
            換の前後で第一基本形式は変化しない。従ってガウス曲率もそれぞれの点で変
            化しない。対偶をとれば、ガウス曲率の一致しない曲面は局所等長的ではない
            。平面のガウス曲率は至る所0であり、半径$r$の球面では$\rec{r^2}$なの
            で、球面を歪ませることなく平面に展開することはできない。この事実は地図学に
            とって重要である。地球の正確な平面図を作成することは不可能であることが示
            されるからである。また折り紙では、平面を折り曲げるだけで変形可能な曲面を
            作ることになる。歪みなく平面に展開可能な曲面を可展面というが、目標となる
            造形をいかに可展面に落とし込むかが重要になってくる。
        \paragraph{共形写像}
            始点が同じ任意の二つのベクトルの成す角を保存する写像を共形写像(等角
            写像)という。
        \paragraph{等温座標}
            パラメータ$(u,v)$から平面への共形写像が存在するとき、$(u,v)$を等温
            座標という。これは第一基本形式が$ds^2 = E(du^2+dv^2)$となることと
            同値である。複素数$z=u+iv$を使えば、$ds^2=E|dz|^2$とも書ける。曲
            面上の任意の点で局所的には等温座標が存在することが証明できる。ここで、
            \begin{eqnarray*}
                \partial = \pd[]{z} = \rec{2}\lr{\pd[]{u}-i\pd[]{v}}\\
                \overline{\partial} = \pd[]{\overline{z}}
                = \rec{2}\lr{\pd[]{u}+i\pd[]{v}}
            \end{eqnarray*}
            とするとガウス曲率は、
                \[K = -\frac{2\partial\overline{\partial}\log E}{E}
                = -\frac{\Delta \log E}{2E}\]
            となる。
            
            
    \section{測地線と極小曲面}
        \subsection{測地線}
            曲面上の二点を結ぶ曲線のうち、最も短いものを最短線という。ここでは曲線
            が最短線となる必要条件を考える。助変数を$(u_1(t),u_2(t))(a\leqq t\leqq b)$
            とし、対応する端点を$A,B$とする。すると$A,B$の長さは汎関数
                \[s[r(t)]=\int_a^b F(u(t),\dot{u}(t))dt\]
            で与えられる。ただし$F(u(t),\dot{u}(t))=\sqrt{g_{ij}(u)\dot{u}^i\dot{u}^j}$
            である。また$ds=Fdt$である。汎関数$s[r(t)]$が極値をとる条件は$F$が
            Euler-Lagrange方程式を満たすことである。そしてそのような条件を満たす
            曲線を測地線と呼ぶ。最短線は測地線だが測地線は最短線ではない。
            まずEuler-Lagrange方程式は、
                \[\di{t}\pd[F]{\dot{u}^i}-\pd[F]{u^i} = 0\]
            で表される。第一項は、
            \begin{align*}
                \pd[F]{\dot{u}^l} &= \rec{2F}g_{jk}
                \left\{\pd[\dot{u}^j]{\dot{u}^l}\dot{u}^k
                +\dot{u}^j\pd[\dot{u}^k]{\dot{u}^l}\right\}\\
                \intertext{$\pd[\dot{u}^j]{\dot{u}^l}$は$j=l$のときに限
                り1なので、}
                {} &= \rec{2F}\lr{\sum_k g_{lk}\dot{u}^k
                +\sum_j g_{jl}\dot{u}^j}\\
                &= \rec{F}g_{il}\dot{u}^i = g_{il}\di[u^i]{s}\\
                \intertext{従って}
                \di{t}\lr{\pd[F]{\dot{u}^l}} &= \lr{\sum_i g_{il}\dd[u^i]{s}
                +\sum_{j,k} \pd[g_{jl}]{u^k}\di[u^k]{s}\di[u^j]{s}}\di[s]{t}\\
                \intertext{$\pd[g_{jl}]{u^k}=[jk,l]+[lk,j]$より、}
                &= \llr{g_{il}\dd[u^i]{s}
                +([jk,l]+[lk,j])\di[u^j]{s}\di[u^k]{s}}\di[s]{t}\\
            \end{align*}
            そして第二項は、
            \begin{align*}
                \pd[F]{u^l} &= \rec{2F}\pd[g_{jk}]{u^l}\dot{u}^j\dot{u}^k\\
                \intertext{$[jk,l]=[kj,l],\pd[g_{jk}]{u^l}=[jl,k]+[kl,j]$
                であることに注意すれば、}
                &= \rec{2F}([jl,k]+[kl,j])\dot{u}^j\dot{u}^k\\
                &= \rec{F}[kl,j]\dot{u}^j\dot{u}^k\\
                &= [kl,j]\di[u^j]{s}\di[u^k]{s}\di[s]{t}\\
            \end{align*}
            なので、
            \begin{align*}
                \di{t}\pd[F]{\dot{u}^l}-\pd[F]{u^l}
                &= \lr{g_{il}\dd[u^i]{s}+[jk,l]\di[u^j]{s}\di[u^k]{s}}\di[s]{t}\\
                &= g_{il}\lr{\dd[u^i]{s}+\chr{i}{jk}\di[u^j]{s}\di[u^k]{s}}\di[s]{t}\\
                &= 0\\
            \end{align*}
            $\det g_{il}\neq 0,ds/dt=F\neq 0$であるので、
                \[\dd[u^i]{s}+\chr{i}{jk}\di[u^j]{s}\di[u^k]{s} = 0\quad (i=1,2,...,n)\]
            これを測地線の方程式という。またこれを
            \[\begin{cases}
                \displaystyle\di[u^i]{s} = v^i\\
                \displaystyle\di[v^i]{s} = -\chr{i}{jk}v^jv^k
            \end{cases}\]
            と書き換えると、1階の連立常微分方程式となるので、次の定理が導かれる。
            \begin{geodesic}
                Riemann多様体Mの任意の点で任意の方向にただ1本の測地線が引け
                る。
            \end{geodesic}

            \paragraph{Weierstrassの表現}
                二次元Reimann多様体、つまり曲面の場合を考える。このとき$k>0$と
                して$F(u,v,k\dot{u},k\dot{v})=kF(u,v,\dot{u},\dot{v})$となって
                $F$は$\dot{u},\dot{v}$について正斉次である。この式を$k$で微分し、
                $k=1$とすると、
                \begin{align*}
                    \pd[F]{\dot{u}}\dot{u}+\pd[F]{\dot{v}}\dot{v} = F\\
                    \intertext{となる。両辺を$\dot{u^i}$で微分して}
                    \pd[^2F]{\dot{u}^2}\dot{u}+\ppd{F}{\dot{v}}{\dot{u}}\dot{v} = 0\\
                    \ppd{F}{\dot{u}}{\dot{v}}\dot{u}+\pd[^2F]{\dot{v}^2}\dot{v} = 0\\
                    \intertext{が導かれ、}
                    \pd[^2F]{\dot{u}^2}:\ppd{F}{\dot{u}}{\dot{v}}:\pd[^2F]{\dot{v}^2}
                    &= \dot{v}^2:-\dot{u}\dot{v}:\dot{u}^2\\
                    \intertext{となる。比例因子を$c$として、}
                    \pd[^2F]{\dot{u}^2} = c\dot{v}^2,\quad
                    \ppd{F}{\dot{u}}{\dot{v}} &= -c\dot{u}\dot{v},\quad
                    \pd[^2F]{\dot{v}^2} = c\dot{u}^2\\
                    \intertext{また}
                    \pd[F]{\dot{u}} = \frac{g_{11}\dot{u}+g_{12}\dot{v}}{F},&\quad
                    \pd[F]{\dot{v}} = \frac{g_{21}\dot{u}+g_{22}\dot{v}}{F}
                    \intertext{なので、}
                    c =& \frac{F_{\dot{u}\dot{u}}}{\dot{v}^2}\\
                    =& \rec{\dot{v}^2}\frac{g_{11}F-(g_{11}\dot{u}+g_{12}\dot{v})F_{\dot{u}}}{F^2}\\
                    =& \rec{\dot{v}^2}\frac{g_{11}F^2-(g_{11}\dot{u}+g_{12}\dot{v})^2}{F^3}\\
                    =& \frac{g_{11}g_{22}-g_{12}^2}{F^3}
                    \intertext{これをEuler-Lagrange方程式に代入すると、}
                    F_{u}-\di{t}F{\dot{u}} = \dot{v}T
                    F_{v}-\di{t}F{\dot{v}} = -\dot{u}T
                    \intertext{ただし、}
                    T = F_{u\dot{v}}-F_{\dot{u}v}&+c(\dot{u}\ddot{v}-\dot{v}\ddot{u})
                \end{align*}
                である。従って曲面の場合、測地線の方程式は$T=0$と同値である。こ
                れをWeierstrassの表現という。
                
                
    \section{Gauss-Bonnetの定理}
        \begin{gaussbonnet}
            $M$をコンパクトな二次元リーマン多様体とする。$K$を$M$のガウス曲率、$k_g$
            を$\pa M$の測地的曲率とすると、
                \[\int_M KdA+\int_{\pa M} k_gds = 2\pi\chi(M)\]
            ただし$\chi(M)$は$M$のオイラー標数である。
        \end{gaussbonnet}