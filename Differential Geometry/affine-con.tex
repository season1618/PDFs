\section{アフィン接続}

$n$次元多様体$M$、構造群$GL(n, \R)$、主ファイバー束として接$n$枠束$L(M)$を考える。$u \in L(M)$は線形同型写像
	\[u: \R^n \rightarrow T_p(M) (p = \pi(u))\]
を定める。

ユークリッド空間の点$p$において、$X(p)$に沿ったベクトル場$Y$の方向微分とは
	\[D_XY(p) = \lim_{t \to 0} \frac{Y(p + X(p)t) - Y(p)}{t}\]
である。$X(c(t)) = \dot{c}(t)$なる$M$上の曲線$c(t)(a \leq t \leq b)$が存在し、この曲線上において方向微分が常に0となるとき、ベクトル場$Y$は曲線に沿って平行であるという。座標に依存しない形式で方向微分を拡張し、平行を定義する。一般に二点における接ベクトルが平行かどうかは経路に依存する。
\begin{dfn}{アフィン接続,共変微分}
	写像$\nabla: (X, Y) \in \mathfrak{X} \times \mathfrak{X} \mapsto \nabla_XY \in \mathfrak{X}$が以下の条件を満たすとき共変微分という。
	\begin{enumerate}
		\item $\nabla_X(Y_1 + Y_2) = \nabla_XY_1 + \nabla_XY_2$
		\item $\nabla_{X_1 + X_2}(Y) = \nabla_{X_1}Y + \nabla_{X_2}Y$
		\item $\nabla_{fX} = f\nabla_XY$
		\item $\nabla_X(fY) = (Xf)Y + f\nabla_XY$
	\end{enumerate}
\end{dfn}

座標系$\{x^i\}$を定めたとき$\nabla_{\pd{x^i}}Y$を単に$\nabla_iY$と書く。
	\[\nabla_{\pd{x^i}}\pd{x^j} = \sum_k \Gamma^k_{ij}\pd{x^k}\]
となる$\Gamma^k_{ij}$を${\pd{x^i}}$に関する接続係数またはクリストッフェル記号という。$X = \sum X^i\pd{x^i}, Y = \sum Y^i\pd{x^i}$のとき
\begin{align*}
	\nabla_XY &= \sum_i X^i\nabla_{\pd{x^i}}\(\sum_j Y^j\pd{x^j}\)\\
	&= \sum_{i,j} X^i\nabla_{\pd{x^i}}\(Y^j\pd{x^j}\)\\
	&= \sum_{i,j} X^i\pd[Y^j]{x^i}\pd{x^j} + X^iY^j\nabla_{\pd{x^i}}\pd{x^j}\\
	&= \sum_{i,j} X^i\pd[Y^j]{x^i}\pd{x^j} + X^iY^j\sum_k \Gamma^k_{ij}\pd{x^k}\\
\end{align*}

\begin{dfn}{標準形式}
	$u \in L(M), X \in T_p(L(M)), p = \pi(u)$に対して
		\[\theta(X) = u^{-1}d\pi(X)\]
	なる1次微分形式$\theta: L(M) \rightarrow \R^n$を標準形式(solder form)と呼ぶ。
\end{dfn}
\begin{dfn}{捩率形式}
	標準形式の共変外微分
		\[\Theta = D\theta = d\theta + \omega \wedge \theta\]
	で定義される2次微分形式$\Theta: L(M) \times L(M) \rightarrow \R^n$を捩率形式(torsion form)と呼ぶ。
\end{dfn}

アフィン接続の不変量として曲率と捩率がある。
\begin{dfn}{曲率テンソル}
	ベクトル場$X, Y$に対して$(1, 3)$型テンソル場
		\[R(X, Y)Z = u(2\Omega(\pi^{-1}(X), \pi^{-1}(Y))(u^{-1}(Z)))\]
	を曲率テンソル場という。曲率テンソルまたは単に曲率ともいう。
\end{dfn}
共変微分を用いて表すと
\begin{align*}
	R(X, Y)Z = \nabla_X\nabla_YZ - \nabla_Y\nabla_XZ - \nabla_[X, Y]Z\\
\end{align*}
\begin{dfn}{捩率テンソル}
	ベクトル場$X, Y$に対して$(1, 2)$型テンソル場
		\[T(X, Y) = u(2\Theta(\pi^{-1}(X), \pi^{-1}(Y))) = \nabla_XY - \nabla_YX - [X, Y]\]
	を捩率テンソル場という。捩率テンソルまたは単に捩率ともいう。
\end{dfn}

共変微分を用いて表すと
\begin{align*}
	T(X, Y) &= u(2\Theta(\pi^{-1}(X), \pi^{-1}(Y)))\\
	2\Theta(\pi^{-1}(X), \pi^{-1}(Y))
	&= 2D\theta(\pi^{-1}(X), \pi^{-1}(Y))\\
	&= 2d\theta()
\end{align*}
\begin{proof}
	\begin{align*}
		T(fX, Y) &= (\nabla_{fX}Y - \nabla_Y(fX)) - (fXY - YfX)\\
		&= (f\nabla_XY - ((Yf)X + f\nabla_YX)) - (fXY - ((Yf)X + fYX))\\
		&= f\nabla_XY - f\nabla_YX - f(XY - YX)\\
		&= fT(X, Y)
	\end{align*}
	同様に
		\[T(X, fY) = fT(X, Y)\]
	だから、$T(X, Y)$は多重線形写像つまりテンソルである。
\end{proof}

$R, T$の成分を接続係数を用いて表す。
	\[R\(\pd{x^i}, \pd{x^j}\)\pd{x^k} = \sum_l R^l_{ijk}\pd{x^l}\]
とおくと
\begin{align*}
	R\(\pd{x^i}, \pd{x^j}\)\pd{x^k}
	&= \nabla_{\pd{x^i}}\sum_l \Gamma^l_{jk}\pd{x^l} - \nabla_{\pd{x^j}}\sum_l \Gamma^l_{ik}\pd{x^l} - \nabla_0Z\\
	&= \sum_l \nabla_{\pd{x^i}}\(\Gamma^l_{jk}\pd{x^l}\) - \sum_l \nabla_{\pd{x^j}}\(\Gamma^l_{ik}\pd{x^l}\)\\
	&= \sum_l \pd[\Gamma^l_{jk}]{x^i}\pd{x^l} + \Gamma^l_{jk}\nabla_{\pd{x^i}}\pd{x^l} - \sum_l \pd[\Gamma^l{ik}]{x^j}\pd{x^l} + \Gamma^l_{ik}\nabla_{\pd{x^j}}\pd{x^l}\\
	&= \sum_l \pd[\Gamma^l_{jk}]{x^i}\pd{x^l} + \Gamma^l_{jk}\sum_m \Gamma^m_{il}\pd{x^m} - \sum_l \pd[\Gamma^l_{ik}]{x^j}\pd{x^l} + \Gamma^l_{ik}\sum_m \Gamma^m_{jl}\pd{x^m}\\
	&= \sum_l \pd[\Gamma^l_{jk}]{x^i}\pd{x^l} + \sum_l\sum_m \Gamma^l_{jk}\Gamma^m_{il}\pd{x^m} - \sum_l \pd[\Gamma^l_{ik}]{x^j}\pd{x^l} - \sum_l\sum_m \Gamma^l_{ik}\Gamma^m_{jl}\pd{x^m}\\
	\intertext{第二項と第四項で添え字$l, m$を入れ替える。}
	&= \sum_l \pd[\Gamma^l_{jk}]{x^i}\pd{x^l} + \sum_l\sum_m \Gamma^m_{jk}\Gamma^l_{im}\pd{x^l} - \sum_l \pd[\Gamma^l_{ik}]{x^j}\pd{x^l} - \sum_l\sum_m \Gamma^m_{ik}\Gamma^l_{jm}\pd{x^l}\\
	&= \sum_l \(\pd[\Gamma^l_{jk}]{x^i} - \pd[\Gamma^l_{ik}]{x^j} + \sum_m \Gamma^m_{jk}\Gamma^l_{im} - \Gamma^m_{ik}\Gamma^l_{jm}\)\pd{x^l}\\
	R^l_{ijk} &= \pd[\Gamma^l_{jk}]{x^i} - \pd[\Gamma^l_{ik}]{x^j} + \sum_m \Gamma^m_{jk}\Gamma^l_{im} - \Gamma^m_{ik}\Gamma^l_{jm}\\
\end{align*}
となる。$R^l_{ijk} = - R^l_{jik}$が成り立つ。また
	\[T\(\pd{x^i}, \pd{x^j}\) = \sum_k T^k_{ij}\pd{x^k}\]
とおくと
\begin{align*}
	T\(\pd{x^i}, \pd{x^j}\)
	&= \nabla_{\pd{x^i}}\pd{x^j} - \nabla_{\pd{x^j}}\pd{x^i} - 0\\
	&= \sum_k \(\Gamma^k_{ij} - \Gamma^k_{ji}\)\pd{x^k}\\
	T^k_{ij} = \Gamma^k_{ij} - \Gamma^k_{ji}
\end{align*}
となる。$T^k_{ij} = - T^k_{ji}$が成り立つ。

曲率形式を曲率テンソルを用いて書くと
捩率形式を捩率テンソルを用いて書くと

$K(X, Y, Z)$の巡回和を
	\[\mathfrak{G}{K(X, Y, Z)} = K(X, Y, Z) + K(Y, Z, X) + K(Z, X, Y)\]
とすると以下の定理が成り立つ。
\begin{thm}{ビアンキの恒等式}
	任意のベクトル場$X, Y, Z$に対して
	\begin{gather*}
		\mathfrak{G}{R(X, Y)Z} = \mathfrak{G}{T(T(X, Y), Z)} + \mathfrak{G}{(\nabla_XT)(Y, Z)}\\
		\mathfrak{G}{(\nabla_ZR)(X, Y) + R(T(X, Y), Z)} = 0\\
	\end{gather*}
	が成り立つ。
\end{thm}