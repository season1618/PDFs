\part{曲線}

この章では空間曲線を分類する問題に取り組む。
\section{曲線の表示}
    曲線は1次元なので、一つのパラメータ$t$を用いて直交座標で、$p(t)=(x(t),y(t),z(t))$と表せる。ここで$p(t)$は滑らかな関数だとする。$p(t)$が動かなければ尖った部分ができてしまうので、$p'(t)\neq 0$と仮定する。このままでは曲線の表示がパラメータに依存してしまうので弧長パラメータを導入する。時刻が0から$t$まで動いたときの曲線の長さは、
        \[s(t) = \int_0^t |p'(t')|dt'\]
    で与えられる。ここで$s$は$t$の単調関数なので、逆関数$t(s)$が存在する。$s$を弧長パラメータと呼ぶ。

\section{Frenet-Serretの公式}
    $s$をパラメータとすれば、当然速さ$|p'(s)| = 1$である。曲線上を点$p$が一定の速さで進むとき、その進行方向の単位ベクトルを$e_1$とすると$e_1 = p'(s)$である。また加速度ベクトル$p''(s)$は速度ベクトルと直交する。これは$p'(s)\cdot p'(s) = 1$を微分すれば容易に示せる。速度ベクトルが単位球面上を動くと考えれば直感的にも分かる。点$p$が加速する方向を主法線方向と呼び、単位ベクトルを$e_2$とする。そして$e_3 = e_1\times e_2$の方向を従法線方向と呼ぶ。三つの単位ベクトル$(e_1,e_2,e_3)$は正規直交基底になっていて、Frenet-Serret標構と呼ばれる。

    ここから$(e_1',e_2',e_3')$と$(e_1,e_2,e_3)$の関係式を導く。まず$|p''(s)| = \kappa(s)(>0)$とすれば、
        \[e_1' = p'' = \kappa(s) e_2\]
    である。$|e_2| = 1$なので、$e_1$と同じように$e_2$と$e_2'$は直交する。つまり、$e_2'$は$e_1$と$e_3$のなす平面上にある。よって$e_2' = ke_1 + \tau e_3$となる関数$k,\tau$が決まる。$e_1\cdot e_2 = 0$の両辺を微分して、
    \begin{align*}
        e_1'\cdot e_2 + e_1\cdot e_2' = 0\\
        \kappa e_2\cdot e_2 + e_1\cdot (ke_1 + \tau e_3) = 0\\
        \kappa + k = 0
    \end{align*}
    よって$k = -\kappa$である。また$e_3 = e_1\times e_2$の両辺を微分して、
    \begin{align*}
        e_3'
        &= e_1'\times e_2+e_1\times e_2'\\
        &= \kappa e_2\times e_2+e_1\times (-\kappa e_1+\tau e_3)\\
        &= \tau e_1\times e_3\\
        &= -\tau e_2
    \end{align*}
    以上で公式が揃った。曲線$p(s)$が与えられたとき、$e_1 = p'(s),\kappa = |e_1'|$とおくと、連立方程式
        \[
            \begin{cases}
                e_1' = \qquad\qquad \kappa e_2\\
                e_2' = -\kappa e_1\quad\qquad +\tau e_3\\
                e_3' = \quad\qquad -\tau e_2
            \end{cases}
        \]
    が成立する。これをFrenet-Serretの公式という。$\kappa$を曲率、$\tau$を捩率(第二曲率)という。曲率と捩率はパラメータをどこを基準にして測るか、またどちらに進むかで変わってくる。始点をずらせば、$\kappa(s),\tau(s)$を$s$軸方向に平行移動させた$\kappa(\pm s-c),\tau(\pm s-c)$も同じ曲線を表すことになる。$p(s)$が与えられれば$\kappa(s)(>0),\tau(s)$が対称・平行移動を除いてただ一つに決まり、逆に、$\kappa(s)(>0),\tau(s)$が与えられれば$p(s)$が合同変換を除いてただ一つに決まる。後者については常微分方程式の解の一意性より従う。