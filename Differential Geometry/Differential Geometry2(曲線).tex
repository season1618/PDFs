\repart{曲線}
    この章では、空間曲線を合同なものは同じとみなすという基準のもと、分類する問題に取り
    組む。
    \section{曲線の表示}
        曲線は1次元なので、一つのパラメータ$t$を用いて直交座標で、
        $p(t)=(x(t),y(t),z(t))$と表せる。ここで$p(t)$は滑らかな関数だとする。
        $p(t)$が動かなければ尖った部分ができてしまうので、$p'(t)\neq 0$と仮定する
        。しかしこれでは、パラメータが変われば曲線の表示も変わってしまう。そこで弧長パラメ
        ータを導入する。時刻$t$が$a$から$b$まで動いたときの曲線の長さは、
            \[s(t) = \int_a^b |p'(t)|dt\]
        で与えられる。ここで$s$は$t$の関数だが、単射な関数なので、逆に$t$を$s$の関
        数とみることもできる。そこで$s$を新たにパラメータとして採用することにする。


    \section{Frenet-Serretの公式}
        $s$をパラメータとすれば、当然速さ$|p'(s)| = 1$である。曲線上を点$p$が一定
        の速さで進むとき、その進行方向の単位ベクトルを$e_1$とする。$e_1 = p'(s)$で
        ある。また加速度ベクトル$p''(s)$は速度ベクトルと直交する。これは
        $p'(s)\cdot p'(s) = 1$を微分すれば容易に示せるが、次のように考えれば直
        感的にわかる。速度ベクトルの絶対値は1なので、直線部分がなければ($|p''(s)|\neq 0$)
        ベクトルの終点は単位球面上を動く。つまり接ベクトル$p''(s)$は$p'(s)$と直交
        する。点$p$が加速する方向を主法線方向と呼び、単位ベクトルを$e_2$とする。
        そして$e_3 = e_1\times e_2$の方向を従法線方向と呼ぶ。三つの単位ベクト
        ル$(e_1,e_2,e_3)$は正規直交基底になっていて、Frenet-Serret標構と呼ば
        れる。\\
        ここから$(e_1',e_2',e_3')$と$(e_1,e_2,e_3)$の関係式を導く。まず、
        $|p''(s)|=\kappa(s)(>0)$とすれば、
            \[e_1' = p'' = \kappa(s) e_2\]
        である。$|e_2|=1$なので、$p'(s)$と同じように$e_2$と$e_2'$は直交する。つま
        り、$e_2'$は$e_1$と$e_3$とで決まる平面上にある。よって$e_2' = ke_1+\tau e_3$
        となる関数$k,\tau$が決まる。$e_1\cdot e_2 = 0$の両辺を微分して、
        \begin{eqnarray*}
            e_1'\cdot e_2+e_1\cdot e_2' = 0\\
            \kappa e_2\cdot e_2+e_1\cdot (ke_1+\tau e_3) = 0\\
            \kappa +k = 0
        \end{eqnarray*}
        よって$k = -\kappa$である。また$e_3 = e_1\times e_2$の両辺を微分して、
        \begin{eqnarray*}
            e_3' &=& e_1'\times e_2+e_1\times e_2'\\
            &=& \kappa e_2\times e_2+e_1\times (-\kappa e_1+\tau e_3)\\
            &=& \tau e_1\times e_3\\
            &=& -\tau e_2
        \end{eqnarray*}
        以上で公式が揃った。曲線$p(s)$が与えられたとき、$e_1 = p'(s),\kappa = |e_1'|$
        とおくと、連立方程式
        \[
            \begin{cases}
                e_1' = \qquad\qquad \kappa e_2\\
                e_2' = -\kappa e_1\quad\qquad +\tau e_3\\
                e_3' = \quad\qquad -\tau e_2
            \end{cases}
        \]
        が成立する。これをFrenet-Serretの公式という。$\kappa$を曲率、$\tau$を捩
        率(第二曲率)という。曲率と捩率はパラメータをどこを基準にして測るか、またどちらに
        進むかで変わってくる。スタート地点をずらせば、$\kappa(s),\tau(s)$を$s$軸方
        向に平行移動させた$\kappa(\pm s-c),\tau(\pm s-c)$も同じ曲線を表すこ
        とになる。$p(s)$が与えられれば、$\kappa(s)(>0),\tau(s)$が対称、平行移
        動を除いてただ一つに決まり、逆に、$\kappa(s)(>0),\tau(s)$が与えられれば、
        $p(s)$が合同変換を除いてただ一つに決まる。後者は、Frenet-Serretの公式の
        解の一意性が常微分方程式の基本定理より従う。