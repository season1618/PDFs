\part{曲面}

\section{曲面の表示}
    曲面を滑らかな関数によって
        \[D \ni (u,v) \mapsto p(u,v) = (x(u,v), y(u,v), z(u,v)) \in S\]
    と表す。また便宜上$(u,v)$を$(u_1,u_2)$、$(x,y,z)$を$(x_1,x_2,x_3)$と表すことがある。この曲面が点や曲線に退化しないでかつ、特異点を持たないよう以下の条件を設ける。以下の5つの条件は全て同値である。
    \begin{enumerate}
        \item $D$上に任意の曲線$s(t) = (u(t),v(t))$を与えると、$s'(t) \neq 0$ならば、$\de{t}p(s(t)) \neq 0$である。つまり$D$上の曲線の像はS上の曲線となる。
        \item $S$の点$p$における接ベクトル全体は平面をなす。これを接平面という。
        \item $x,y,z$の内、任意の二つを$x_i,x_j$とすると、ヤコビアン
                  \[\left|\pd[(x_i,x_j)]{(u,v)}\right|\neq 0\]
              である。このとき十分小さい$D$で、写像
                  \[(u,v) \mapsto (x_i,x_j)\]
              は全単射で、逆写像も滑らかである(陰関数定理)。
        \item $x,y,z$の内、任意の二つを$x_i,x_j$とすると、
                  \[dx_i \wedge dx_j \neq 0\]
        \item ベクトル$\pd[p]{u}$と$\pd[p]{v}$は線形独立。
    \end{enumerate}
    曲面を表示するに当たって、正規直交枠とガウス標構を導入する。曲面上の点$p$を原点とし、その点ににおける法線方向を$z$軸とする直交座標系を$(x,y,z)$とする。それぞれの単位ベクトルを$e_1,e_2,e_3$とする。$e_3$は$p$が曲面上を移動するとき、一定の向きを持つように決める。$e_1,e_2$は$e_1 \times e_2 = e_3$となるようにとる。また、$x$軸と$y$軸は回転の分の自由度を残しておく。$z = f(x,y)$とすれば、必然的に$f(0,0) = f_x(0,0) = f_y(0,0) = 0$が成り立つ。$(e_1,e_2,e_3)$を正規直交枠と呼ぶ。一方$B_1 = \pd[p]{u}, B_2 = \pd[p]{v}$で、$n$を$B_1 \times B_2$の方向の単位法線ベクトルとしたとき、$(B_1,B_2,n)$をガウス標構と呼ぶ。

\section{曲面の基本量}
    \subsection{第一基本形式}
        曲面上における二点間の微小長さは、
        \begin{align*}
            ds^2 &= dp^2\\
            &= \(\pd[p]{u}du + \pd[p]{v}dv\)^2\\
            &= p_u^2du^2 + 2p_up_vdudv + p_v^2dv^2
        \end{align*}
        で与えられる。$g_{ij} = \pd[p]{u_i} \cdot \pd[p]{u_j}$とおいて
        \begin{gather*}
            ds^2 = g_{11}du^2 + 2g_{12}dudv + g_{22}dv^2 \\
            I = 
            \begin{pmatrix}
                g_{11} & g_{12}\\
                g_{21} & g_{22}
            \end{pmatrix}
        \end{gather*}
        これを第一基本形式という。また$I$をリーマン計量(計量テンソル)と呼ぶ。曲面上の計量をユークリッド計量によって定義する。

    \subsection{第二基本形式}
        曲面上の点$p(u,v)$とその接平面を考える。点$p(u + du, v + dv)$と接平面の距離は
        \begin{align*}
            dz &= (p(u + du, v + dv) - p(u, v)) \cdot n
            \intertext{二次の項までテイラー展開すると、}
            &= \(\pd[p]{u}du + \pd[p]{v}dv + \frac{1}{2}\pd[^2p]{u^2}du^2 + \frac{\partial^2 p}{\partial u\partial v}dudv + \frac{1}{2}\pd[^2p]{v^2}dv^2\) \cdot n\\
        \end{align*}
        $p_u, p_v \perp n$であることに注意すると、
            \[dz = \(\frac{1}{2}p_{uu}du^2 + p_{uv}dudv + \frac{1}{2}p_{vv}dv^2\) \cdot n\]
        となる。$h_{ij} = \ppd[p]{u_i}{v_i}$とおいて
        \begin{gather*}
            2dz = h_{11}du^2 + 2h_{12}dudv + h_{22}dv^2\\
            II =
            \begin{pmatrix}
                h_{11} & h_{12}\\
                h_{21} & h_{22}
            \end{pmatrix}
        \end{gather*}
        これを第二基本形式という。

        第一基本形式は曲面における計量を表すという意味で内在的であり、第二基本形式は曲面の外部の空間への入り方を表しているという意味で外在的である。
        
        
\section{曲面の曲率}
    \subsection{測地的曲率と法曲率}
        曲面上の曲線を弧長パラメータによって$p(u,v) = p(u(s),v(s))$と表す。これに正規直交枠$(e_1,e_2,e_3)$を導入する。$e_1$を曲線の接線方向、$e_3$を曲面の法線方向の単位ベクトルとすると、$p''(s) = e_1' = k_ge_2+k_ne_3$と表すことができる。このとき$k_g, k_n$をそれぞれ測地的曲率、法曲率という。$|p''|$は空間曲線としての曲率$k(s)$である。ここで、
        \begin{align*}
            p' &= p_u\de[u]{s} + p_v\de[v]{s}\\
            p'' &= \de[p_u]{s}\de[u]{s} + p_u\de[^2u]{s^2} + \de[p_v]{s}\de[v]{s} + p_v\de[^2v]{s^2}\\
            &= p_{uu}\(\de[u]{s}\)^2 + 2p_{uv}\de[u]{s}\de[v]{s} + p_{vv}\(\de[v]{s}\)^2 + p_u\de[^2u]{s^2} + p_v\d3[^2v]{s^2}\\
        \end{align*}
        なので$p_u \cdot e_3 = p_v \cdot e_3 = 0$より、
        \begin{align*}
            k_n &= p'' \cdot e_3\\
            &= (p_{uu} \cdot e_3)\(\di[u]{s}\)^2 + 2(p_{uv} \cdot e_3)\de[u]{s}\de[v]{s} + (p_{vv} \cdot e_3)\(\de[v]{s}\)^2\\
            &= h_{11}\(\di[u]{s}\)^2 + 2h_{12}\de[u]{s}\de[v]{s} + h_{22}\(\di[v]{s}\)^2\\
            &= \frac{h_{11}du^2 + 2h_{12}dudv + h_{22}dv^2}{g_{11}du^2 + 2g_{12}dudv + g_{22}dv^2}\\
        \end{align*}
        となる。つまり曲線上の点$p$における法曲率は$\lambda = dv/du$のみに依存する。

    \subsection{主曲率}
        曲面上の点において、法線ベクトルを含む平面と曲面が交わってできる曲線を法切断または直截線という。点$p$における法切断の測地的曲率は0なので、曲率は法曲率に等しい。法切断の曲率が最大値最小値をとるとき、その曲率$k_1,k_2$を主曲率、接ベクトルを主方向$X_1,X_2$と呼ぶ。法曲率が$k$となる条件は、
        \begin{gather*}
            k = \frac{h_{11} + 2h_{12}\lambda + h_{22}\lambda^2}{g_{11} + 2g_{12}\lambda + g_{22}\lambda^2}\\
            (h_{22} - kg_{22})\lambda^2 + 2(h_{12} - kg_{12})\lambda + (h_{11} - kg_{11}) = 0
        \end{gather*}
        の解が存在することであり、この二次方程式の判別式$D$が0以上になることである。
        \begin{align*}
            \frac{D}{4} &= (h_{12} - kg_{12})^2 - (h_{11} - kg_{11})(h_{22} - kg_{22})\\
            &= (g_{12}^2 - g_{11}g_{22})k^2 + (g_{11}h_{22} - 2g_{12}h_{12} + g_{22}h_{11})k + (h_{12}^2 - h_{11}h_{22}) \geq 0
        \end{align*}
        $g$は正定値行列なので$g_{12}^2 - g_{11}g_{22} < 0$である。よって$k$が極値を取るのは$D = 0$のときである。
        \begin{align*}
            (h_{11} - kg_{11})(h_{22} - kg_{22}) - (h_{12} - kg_{12})(h_{21} - kg_{21}) = 0\\
            \det(II - kI) = 0\\
            \det(I^{-1}II - kE) = 0\\
        \end{align*}
        となる。これは$I^{-1}II$の固有方程式である。$I^{-1}II$をシェイプ作用素という。つまり二つの主曲率はシェイプ作用素の固有値である。

    \subsection{ガウス曲率と平均曲率}
        二つの主曲率$k_1,k_2$の積をガウス曲率$K$、平均を平均曲率$H$という。主曲率の満たす二次方程式の解と係数の関係より、
        \begin{align*}
            K &= k_1k_2 = \frac{h_{11}h_{22} - h_{12}^2}{g_{11}g_{22} - g_{12}^2}\\
            2H &= k_1 + k_2 = \frac{g_{11}h_{22} - 2g_{12}h_{12} + g_{22}h_{11}}
            {g_{11}g_{22} - g_{12}^2}\\
        \end{align*}
        となる。ガウス曲率が正の点を楕円点、負の点を双曲点という。また0の点を放物点と呼ぶこともある。

    \subsection{オイラーの定理(微分幾何)}
        ここでは微分幾何におけるオイラーの定理を証明する。まず補題を一つ証明する。
        \begin{lem}
            平面曲線$y = y(x)$で$y'(0) = 0$のとき、原点における曲率は、$y''(0)$
            である$($平面曲線の場合、曲率に符号を付ける$)$。
        \end{lem}

        \begin{proof}
            $p = (x,y)$、弧長パラメータを$s$とすれば曲率は$\de[p]{s}$である。
                \[\de[p]{x} = (1,y'),\de[s]{x} = \sqrt{1 + y'^2}\]
            より、
            \begin{gather*}
                \de[p]{s} = \frac{(1, y')}{\sqrt{1 + y'^2}}\\
                \de{x}\de[p]{s} = \frac{(0, y'')\sqrt{1 + y'^2} - (1, y')y'y''/\sqrt{1 + y'^2}}{1 + y'^2}\\
                \intertext{$y'(0) = 0$なので、}
                \de[p]{s} = \frac{(0, y'')(1 + y'^2) - (1, y')y'y''}{(1 + y'^2)^2} = (0,y'')
            \end{gather*}
            よって原点における曲率は、$y''$となる。
        \end{proof}

        \begin{thm}{オイラーの定理(微分幾何)}
            二つの主方向は直交し、更に$X_1$に対して角$\theta$をなす法切断の曲率を$k_\theta$とすれば、
                \[k_\theta = k_1\cos^2\theta + k_2\sin^2\theta\]
            である。
        \end{thm}
        \begin{proof}
            曲面上に正規直交枠をとる。補題より、$x,y$軸の法切断の曲率は偏微分となり$\pd[^2z]{x^2},\pd[^2z]{y^2}$である。$x$軸と角$\theta$をなす方向を$v=(\cos\theta,\sin\theta)$とすれば、曲率$k_\theta$は方向微分となり、
            \begin{align*}
                k_\theta &= \pd[^2z]{v^2}\\
                &= \pd{v} (z_x\cos\theta + z_y\sin\theta)\\
                &= (z_{xx}\cos\theta + z_{xy}\sin\theta,
                z_{xy}\cos\theta + z_{yy}\sin\theta)
                \cdot(\cos\theta,\sin\theta)\\
                &= z_{xx}\cos^2\theta + 2z_{xy}\sin\theta\cos\theta + z_{yy}\sin^2\theta
            \end{align*}
            この二次形式の最大値と最小値は行列$\begin{pmatrix}z_{xx} & z_{xy}\\ z_{yx} & z_{yy}\end{pmatrix}$の固有値に等しい。また主方向はそれぞれの固有値の固有ベクトルである。対称行列の固有ベクトルは直交するので主方向も直交する。この固有ベクトルで主軸変換を行えば、(主方向との成す角を新たに$\theta$とおいて)
                \[k_\theta = k_1\cos^2\theta + k_2\sin^2\theta\]
        \end{proof}

\section{曲面の分類と驚異の定理}

    曲面を分類する。

    まずChristoffel記号を導入する。
    \begin{align*}
        [jk,l] &= \frac{1}{2}\(\pd[g_{kl}]{u^j} + \pd[g_{jl}]{u^k} - \pd[g_{jk}]{u^l}\)\\
        \chr{i}{jk} &= \frac{1}{2}g^{ih}\(\pd[g_{jh}]{u^k} + \pd[g_{kh}]{u^j} - \pd[g_{jk}]{u^h}\)
    \end{align*}
    ここで$[jk,l] = g_{il}\chr{i}{jk}, \chr{i}{jk} = g^{ih}[jk,h]$が成り立っている。$[jk,l], \chr{i}{jk}$をそれぞれ第一種及び第二種のChristoffel記号という。

    \subsection{Gauss-Weingardenの方程式}
        Gauss標構$\{B_1,B_2,n\}$をとる。ここで
            \[n = \frac{B_1 \times B_2}{|B_1 \times B_2|} = \frac{B_1 \times B_2}{\sqrt{g_{11}g_{22} - g_{12}^2}}\]
        である。三つのベクトルは線形独立なので、
            \[\pd[B_j]{u^k} = \Gamma_{jk}^iB_i + h_{jk}n\]
        と書ける。$h_{jk} = \pd[B_j]{u^k} \cdot n$は定義より第二基本量である。また$g_{ij} = B_i \cdot B_j$を微分した$\pd[B_i]{u^k} \cdot B_j + B_i\cdot \pd[B_j]{u^k} = \pd[g_{ij}]{u^k}$に代入して、
            \[\Gamma_{ik}^hg_{hj} + \Gamma_{jk}^hg_{ih} = \pd[g_{ij}]{u^k}\]
        を得る。$\Gamma_{jk|i} = \Gamma_{jk}^hg_{ih}$と置けば、
            \[\Gamma_{ik|j} + \Gamma_{jk|i} = \pd[g_{ij}]{u^k}\]
        となる。$\pd[B_j]{u^k} = \ppd{p}{u^j}{u^k}$より$j,k$に関して対称である。そこで上の式の添え字を循環的に入れ替えて、$\Gamma_{jk|i}$に関する連立方程式とみれば、
        \begin{align*}
            \Gamma_{jk|i} &= \frac{1}{2}\(\pd[g_{ki}]{u^j} + \pd[g_{ij}]{u^k} - \pd[g_{jk}]{u^i}\) = [jk,i]\\
            \Gamma_{jk}^i &= g^{ih}\Gamma_{jk|h} = \chr{i}{jk}\\
        \end{align*}
        となることが分かる。次に$n \cdot n = 1$を微分すると$\pd[n]{u^j} \cdot n = 0$だから
            \[\pd[n]{u^k} = r_k^hB_h\]
        となる$r_j^h$が決まる。$B_i \cdot n = 0$を微分した$\pd[B_i]{u^k} \cdot n + B_i \cdot \pd[n]{u^k} = 0$に先程の式と代入して、
            \[h_{ij} + r_j^hg_{ih} = 0\]
        これは行列の積を成分ごとに表したものなので、両辺に右から$g$の逆行列をかけて、
            \[r_k^i = -h_{kl}g^{li}\]
        二つを合わせて
        \begin{align*}
            \pd[B_j]{u^k} &= \chr{i}{jk}B_i + h_{jk}n \tag{Gaussの方程式}\\
            \pd[n]{u^k} &= -h_{kl}g^{li}B_i \tag{Weingartenの方程式}
        \end{align*}


    \subsection{Gauss-Codazziの積分可能条件}
        Gauss-Weingartenの方程式は連立一次偏微分方程式である。係数は第一及び第二基本量から求められ、この方程式を解くことで$\{B_1,B_2,n\}$を求めることができる。しかしここで$B_j = \pd[p]{u^j}$なので、求められたガウス標構が実際の曲面と矛盾しないためにはいくつか条件が必要である。具体的には、
        \begin{align*}
            \ppd{p}{u^j}{u^i} &= \pd{p}{u^i}{u^j}\\
            \ppd{B_i}{u^k}{u^j} &= \pd{B_i}{u^j}{u^k}\\
            \ppd{n}{u^j}{u^i} &= \ppd{n}{u^i}{u^j}
        \end{align*}
        である。第一式に関しては既に成り立っていることがわかる。なぜなら
            \[\ppd{p}{u^j}{u^i} = \pd[B_i]{u^j} = \chr{h}{ij}B_h + h_{ij}n\]
        は$i,j$に関して対称だからである。次に第二式の条件を求める。
        \begin{align*}
            \ppd{B_i}{u^k}{u^j}
            &= \pd{u^k}\(\chr{h}{ij}B_h + h_{ij}n\)\\
            &= \pd{u^k}\chr{h}{ij}B_h + \chr{h}{ij}\pd[B_h]{u^k} + \pd[h_{ij}]{u^k}n + h_{ij}\pd[n]{u^k}
        \end{align*}
        $\pd[B_h]{u^k}$と$\pd[n]{u^k}$にそれぞれ代入し、
        \begin{align*}
            & = \sum_h \pd{u^k}\chr{h}{ij}B_h
            + \sum_{h,l} \chr{h}{ij}\chr{l}{hk}B_l + \sum_h \chr{h}{ij}h_{hk}n\\
            &\qquad + \pd[h_{ij}]{u^k}n + h_{ij}\sum_{l,h}(-h_{kl}g^{lh}B_h)\\
        \end{align*}
        第二項で添え字$h,l$を付け替えると、
        \begin{align*}
            &= \llr{\pd{u^k}\chr{h}{ij} + \chr{h}{kl}\chr{l}{ij} - h_{ij}h_{kl}g^{lh}}B_h\\
            &\qquad + \llr{\pd[h_{ij}]{u^k} + \chr{h}{ij}h_{hk}}n\\
        \end{align*}
        となる。これと$j,k$を入れ替えたものが等しくなる。また$B_h,n$は線形独立なので、
        \begin{align*}
            R^h_{ijk} = h_{ik}h_{jl}g^{lh} - h_{ij}h_{kl}g^{lh} \tag{Gaussの積分可能条件}\\
            \pd[h_{ij}]{u^k} - \pd[h_{ik}]{u^j} + \chr{h}{ij}h_{hk} - \chr{h}{ik}h_{hj} = 0 \tag{Codazziの積分可能条件}
        \end{align*}
        となる。ただし
            \[R^h_{ijk} = \pd{u^j}\chr{h}{ik} - \pd{u^k}\chr{h}{ij} + \chr{l}{ik}\chr{h}{jl} - \chr{l}{ij}\chr{h}{kl}\]
        である。また両辺に$g^{hl}$をかけて縮約を取ったものを、
            \[R_{lijk} = g^{hl}R^h_{ijk} = \pd{u^j}\chr{h}{ik}g^{hl} - \pd{u^k}\chr{h}{ij}g^{hl} + \chr{a}{ik}\chr{h}{ja}g^{hl} - \chr{a}{ij}\chr{h}{ka}g^{hl}\]
        これらをリーマン曲率テンソルと呼ぶ。リーマン曲率テンソルを用いてGaussの積分可能条件を表すと、
        \begin{align*}
            g^{ha}R^h_{ijk} &= h_{ik}h_{ja}g^{ah}g^{hl} - h_{ij}h_{ka}g^{ah}g^{hl}\\
            R_{lijk} &= h_{ik}h_{ja}\delta_l^a - h_{ij}h_{ka}\delta_l^a\\
            &= h_{ik}h_{jl} - h_{ij}h_{kl}
        \end{align*}
        また第三式については、Codazziの積分可能条件と同じものが導かれるので、条件はこれで十分である。よって次の定理が成り立つ。
        \begin{thm}{曲面論の基本定理}
            対称テンソル$g_{ij}, h_{ij}$が与えられ、$g_{ij}$は正定値であるとする。これらがGauss-Codazziの積分可能条件を満たすとき、これらを第一及び第二基本量とする曲面$p(u,v)$が剛体運動を除いて一意的に存在する。
        \end{thm}


    \subsection{驚異の定理}
        上の式は右辺が第二基本量のみに依存している。元々リーマン曲率テンソル及びその定義に含まれるクリストッフェル記号は第一基本量のみから求めることができた。つまり第一基本テンソルと第二基本テンソルは独立ではない。特に
            \[R_{1212} = h_{11}h_{22} - h_{12}^2\]
        であり、これを用いると、
            \[K = \frac{h_{11}h_{22} - h_{12}^2}{g_{11}g_{22} - g_{12}^2} = \frac{R_{1212}}{g_{11}g_{22}-g_{12}^2}\]
        のように曲面のガウス曲率を第一基本テンソルだけから求めることができる。これを驚異の定理(Theorem Egregium)という。元々外在的な量から定義されたガウス曲率が、内在的な量のみから決定できることが示された。外界の情報を用いることなく空間の曲がり方を考察することができることを示唆している。


    \paragraph{等長写像}
        二点間の距離を保つ写像を等長写像という。二つの曲面が等長写像によって移りあうとき、それらは局所等長的であるという。局所等長的な二つの曲面は、伸縮したりせず連続的に折り曲げるだけで変形させることができる。また等長変換の前後で第一基本形式は変化せず、従ってガウス曲率も各点で変化しない。対偶をとれば、ガウス曲率の一致しない曲面は局所等長的ではない。平面のガウス曲率は任意の点で0であり、半径$r$の球面では$\frac{1}{r^2}$なので、球面を歪ませることなく平面に展開することはできない。つまり地球の正確な平面図を作成することは不可能である。また折り紙では、平面を折り曲げることでできる曲面(可展面)を作ることになる。
    \paragraph{共形写像}
        始点が同じ任意の二つのベクトルの成す角を保存する写像を共形写像(等角写像)という。
    \paragraph{等温座標}
        パラメータ$(u,v)$から平面への共形写像が存在するとき、$(u,v)$を等温座標という。これは第一基本形式が$ds^2 = E(du^2+dv^2)$となることと同値である。複素数$z=u+iv$を使えば、$ds^2=E|dz|^2$とも書ける。曲面上の任意の点で局所的には等温座標が存在することが証明できる。ここで、
        \begin{align*}
            \partial = \pd{z} = \frac{1}{2}\(\pd{u} - i\pd{v}\)\\
            \overline{\partial} = \pd{\overline{z}} = \frac{1}{2}\lr{\pd[]{u}+i\pd[]{v}}
        \end{align*}
        とするとガウス曲率は、
            \[K = -\frac{2\partial\overline{\partial}\log E}{E} = -\frac{\Delta \log E}{2E}\]
        となる。
        
        
\section{測地線}
    曲面上の二点を結ぶ曲線のうち、最も短いものを最短線という。曲線が最短線となる必要条件を考える。助変数を$(u_1(t),u_2(t))(a\leq t\leq b)$とし、対応する端点を$A,B$とする。すると$AB$の長さは汎関数
        \[s[r(t)] = \int_a^b F(u(t),\dot{u}(t))dt\]
    で与えられる。ただし$F(u(t),\dot{u}(t)) = \sqrt{g_{ij}(u)\dot{u}^i\dot{u}^j}$である。また$ds = Fdt$である。汎関数$s[r(t)]$が極値をとる条件は$F$がEuler-Lagrange方程式を満たすことである。そのような曲線を測地線と呼ぶ。最短線は測地線だが測地線は最短線ではない。まずEuler-Lagrange方程式は、
        \[\de{t}\pd[F]{\dot{u}^i} - \pd[F]{u^i} = 0\]
    で表される。第一項は、
        \[\pd[F]{\dot{u}^l} = \frac{1}{2F}g_{jk}\left\{\pd[\dot{u}^j]{\dot{u}^l}\dot{u}^k + \dot{u}^j\pd[\dot{u}^k]{\dot{u}^l}\right\}\]
    $\pd[\dot{u}^j]{\dot{u}^l}$は$j = l$のときに限り1なので、
    \begin{align*}
        &= \frac{1}{2F}\(\sum_k g_{lk}\dot{u}^k + \sum_j g_{jl}\dot{u}^j\)\\
        &= \frac{1}{F}g_{il}\dot{u}^i = g_{il}\di[u^i]{s}\\
    \end{align*}
    従って
    \begin{align*}
        \de{t}\(\pd[F]{\dot{u}^l}\) &= \(\sum_i g_{il}\de[^2u^i]{s^2} + \sum_{j,k} \pd[g_{jl}]{u^k}\de[u^k]{s}\de[u^j]{s}\)\de[s]{t}\\
    \end{align*}
    $\pd[g_{jl}]{u^k} = [jk,l] + [lk,j]$より、
    \begin{align*}
        &= \llr{g_{il}\de[u^i]{s} + ([jk,l] + [lk,j])\de[u^j]{s}\de[u^k]{s}}\de[s]{t}\\
    \end{align*}
    そして第二項は、
    \begin{align*}
        \pd[F]{u^l} &= \frac{1}{2F}\pd[g_{jk}]{u^l}\dot{u}^j\dot{u}^k\\
        \intertext{$[jk,l] = [kj,l], \pd[g_{jk}]{u^l} = [jl,k] + [kl,j]$であることに注意すれば、}
        &= \frac{1}{2F}([jl,k] + [kl,j])\dot{u}^j\dot{u}^k\\
        &= \frac{1}{F}[kl,j]\dot{u}^j\dot{u}^k\\
        &= [kl,j]\de[u^j]{s}\de[u^k]{s}\de[s]{t}\\
    \end{align*}
    なので、
    \begin{align*}
        \de{t}\pd[F]{\dot{u}^l} - \pd[F]{u^l}
        &= \(g_{il}\de[^2u^i]{s^2} + [jk,l]\de[u^j]{s}\de[u^k]{s}\)\de[s]{t}\\
        &= g_{il}\(\de[^2u^i]{s^2}+\chr{i}{jk}\de[u^j]{s}\de[u^k]{s}\)\de[s]{t}\\
        &= 0\\
    \end{align*}
    $\det g_{il} \neq 0, ds/dt = F \neq 0$であるので、
        \[\de[^2u^i]{s^2} + \chr{i}{jk}\de[u^j]{s}\de[u^k]{s} = 0\quad (i=1,2,...,n)\]
    これを測地線の方程式という。またこれを
    \begin{align*}
        \de[u^i]{s} &= v^i\\
        \de[v^i]{s} &= -\chr{i}{jk}v^jv^k
    \end{align*}
    と書き換えると、1階の連立常微分方程式となるので、次の定理が導かれる。
    \begin{thm}{測地線}
        リーマン多様体$M$の任意の点で任意の方向にただ1本の測地線が引ける。
    \end{thm}

    \subsection{Weierstrassの表現}
        二次元リーマン多様体、つまり曲面の場合を考える。このとき$k > 0$として$F(u,v,k\dot{u},k\dot{v}) = kF(u,v,\dot{u},\dot{v})$となって$F$は$\dot{u},\dot{v}$について正斉次である。この式を$k$で微分し、$k = 1$とすると、
            \[\pd[F]{\dot{u}}\dot{u} + \pd[F]{\dot{v}}\dot{v} = F\]
        となる。両辺を$\dot{u^i}$で微分して
        \begin{align*}
            \pd[^2F]{\dot{u}^2}\dot{u}+\ppd{F}{\dot{v}}{\dot{u}}\dot{v} = 0\\
            \ppd{F}{\dot{u}}{\dot{v}}\dot{u}+\pd[^2F]{\dot{v}^2}\dot{v} = 0\\
        \end{align*}
        が導かれ、
            \[\pd[^2F]{\dot{u}^2} : \ppd{F}{\dot{u}}{\dot{v}} : \pd[^2F]{\dot{v}^2} = \dot{v}^2 : -\dot{u}\dot{v} : \dot{u}^2\]
        となる。比例因子を$c$として
        \begin{align*}
            \pd[^2F]{\dot{u}^2} = c\dot{v}^2\\
            \ppd{F}{\dot{u}}{\dot{v}} &= -c\dot{u}\dot{v}\\
            \pd[^2F]{\dot{v}^2} = c\dot{u}^2\\
        \end{align*}
        また
        \begin{align*}
            \pd[F]{\dot{u}} = \frac{g_{11}\dot{u} + g_{12}\dot{v}}{F}\\
            \pd[F]{\dot{v}} = \frac{g_{21}\dot{u} + g_{22}\dot{v}}{F}\\
        \end{align*}
        なので
        \begin{align*}
            c &= \frac{F_{\dot{u}\dot{u}}}{\dot{v}^2}\\
            &= \frac{1}{\dot{v}^2}\frac{g_{11}F - (g_{11}\dot{u} + g_{12}\dot{v})F_{\dot{u}}}{F^2}\\
            &= \frac{1}{\dot{v}^2}\frac{g_{11}F^2 - (g_{11}\dot{u} + g_{12}\dot{v})^2}{F^3}\\
            &= \frac{g_{11}g_{22} - g_{12}^2}{F^3}
        \end{align*}
        これをEuler-Lagrange方程式に代入すると
        \begin{align*}
            F_{u} - \de{t}F{\dot{u}} = \dot{v}T
            F_{v} - \de{t}F{\dot{v}} = -\dot{u}T
        \end{align*}
        ただし
            \[T = F_{u\dot{v}} - F_{\dot{u}v} + c(\dot{u}\ddot{v} - \dot{v}\ddot{u})\]
        従って曲面の場合、測地線の方程式は$T = 0$と同値である。これをWeierstrassの表現という。
            
\section{極小曲面}        
\section{Gauss-Bonnetの定理}
    \begin{thm}{Gauss-Bonnetの定理}
        $M$をコンパクトな二次元リーマン多様体とする。$K$を$M$のガウス曲率、$k_g$を$\partial M$の測地的曲率とすると、
            \[\int_M KdA + \int_{\partial M} k_gds = 2\pi\chi(M)\]
        ただし$\chi(M)$は$M$のオイラー標数である。
    \end{thm}