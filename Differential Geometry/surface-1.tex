\part{曲面}

\section{曲面の表示}
    曲面を滑らかな関数によって
        \[D \ni (u,v) \mapsto p(u,v) = (x(u,v), y(u,v), z(u,v)) \in S\]
    と表す。また便宜上$(u,v)$を$(u_1,u_2)$、$(x,y,z)$を$(x_1,x_2,x_3)$と表すことがある。この曲面が点や曲線に退化しないでかつ、特異点を持たないよう以下の条件を設ける。以下の5つの条件は全て同値である。
    \begin{enumerate}
        \item $D$上に任意の曲線$s(t) = (u(t),v(t))$を与えると、$s'(t) \neq 0$ならば、$\de{t}p(s(t)) \neq 0$である。つまり$D$上の曲線の像はS上の曲線となる。
        \item $S$の点$p$における接ベクトル全体は平面をなす。これを接平面という。
        \item $x,y,z$の内、任意の二つを$x_i,x_j$とすると、ヤコビアン
                  \[\left|\pd[(x_i,x_j)]{(u,v)}\right|\neq 0\]
              である。このとき十分小さい$D$で、写像
                  \[(u,v) \mapsto (x_i,x_j)\]
              は全単射で、逆写像も滑らかである(陰関数定理)。
        \item $x,y,z$の内、任意の二つを$x_i,x_j$とすると、
                  \[dx_i \wedge dx_j \neq 0\]
        \item ベクトル$\pd[p]{u}$と$\pd[p]{v}$は線形独立。
    \end{enumerate}
    曲面を表示するに当たって、正規直交枠とガウス標構を導入する。曲面上の点$p$を原点とし、その点ににおける法線方向を$z$軸とする直交座標系を$(x,y,z)$とする。それぞれの単位ベクトルを$e_1,e_2,e_3$とする。$e_3$は$p$が曲面上を移動するとき、一定の向きを持つように決める。$e_1,e_2$は$e_1 \times e_2 = e_3$となるようにとる。また、$x$軸と$y$軸は回転の分の自由度を残しておく。$z = f(x,y)$とすれば、必然的に$f(0,0) = f_x(0,0) = f_y(0,0) = 0$が成り立つ。$(e_1,e_2,e_3)$を正規直交枠と呼ぶ。一方$B_1 = \pd[p]{u}, B_2 = \pd[p]{v}$で、$n$を$B_1 \times B_2$の方向の単位法線ベクトルとしたとき、$(B_1,B_2,n)$をガウス標構と呼ぶ。

\section{曲面の基本量}
    \subsection{第一基本形式}
        曲面上における二点間の微小長さは、
        \begin{align*}
            ds^2 &= dp^2\\
            &= \(\pd[p]{u}du + \pd[p]{v}dv\)^2\\
            &= p_u^2du^2 + 2p_up_vdudv + p_v^2dv^2
        \end{align*}
        で与えられる。$g_{ij} = \pd[p]{u_i} \cdot \pd[p]{u_j}$とおいて
        \begin{gather*}
            ds^2 = g_{11}du^2 + 2g_{12}dudv + g_{22}dv^2 \\
            I = 
            \begin{pmatrix}
                g_{11} & g_{12}\\
                g_{21} & g_{22}
            \end{pmatrix}
        \end{gather*}
        これを第一基本形式という。また$I$をリーマン計量(計量テンソル)と呼ぶ。曲面上の計量をユークリッド計量によって定義する。

    \subsection{第二基本形式}
        曲面上の点$p(u,v)$とその接平面を考える。点$p(u + du, v + dv)$と接平面の距離は
        \begin{align*}
            dz &= (p(u + du, v + dv) - p(u, v)) \cdot n
            \intertext{二次の項までテイラー展開すると、}
            &= \(\pd[p]{u}du + \pd[p]{v}dv + \frac{1}{2}\pd[^2p]{u^2}du^2 + \frac{\partial^2 p}{\partial u\partial v}dudv + \frac{1}{2}\pd[^2p]{v^2}dv^2\) \cdot n\\
        \end{align*}
        $p_u, p_v \perp n$であることに注意すると、
            \[dz = \(\frac{1}{2}p_{uu}du^2 + p_{uv}dudv + \frac{1}{2}p_{vv}dv^2\) \cdot n\]
        となる。$h_{ij} = \ppd[p]{u_i}{v_i}$とおいて
        \begin{gather*}
            2dz = h_{11}du^2 + 2h_{12}dudv + h_{22}dv^2\\
            II =
            \begin{pmatrix}
                h_{11} & h_{12}\\
                h_{21} & h_{22}
            \end{pmatrix}
        \end{gather*}
        これを第二基本形式という。

        第一基本形式は曲面における計量を表すという意味で内在的であり、第二基本形式は曲面の外部の空間への入り方を表しているという意味で外在的である。
        
        
\section{曲面の曲率}
    \subsection{測地的曲率と法曲率}
        曲面上の曲線を弧長パラメータによって$p(u,v) = p(u(s),v(s))$と表す。これに正規直交枠$(e_1,e_2,e_3)$を導入する。$e_1$を曲線の接線方向、$e_3$を曲面の法線方向の単位ベクトルとすると、$p''(s) = e_1' = k_ge_2+k_ne_3$と表すことができる。このとき$k_g, k_n$をそれぞれ測地的曲率、法曲率という。$|p''|$は空間曲線としての曲率$k(s)$である。ここで、
        \begin{align*}
            p' &= p_u\de[u]{s} + p_v\de[v]{s}\\
            p'' &= \de[p_u]{s}\de[u]{s} + p_u\de[^2u]{s^2} + \de[p_v]{s}\de[v]{s} + p_v\de[^2v]{s^2}\\
            &= p_{uu}\(\de[u]{s}\)^2 + 2p_{uv}\de[u]{s}\de[v]{s} + p_{vv}\(\de[v]{s}\)^2 + p_u\de[^2u]{s^2} + p_v\d3[^2v]{s^2}\\
        \end{align*}
        なので$p_u \cdot e_3 = p_v \cdot e_3 = 0$より、
        \begin{align*}
            k_n &= p'' \cdot e_3\\
            &= (p_{uu} \cdot e_3)\(\di[u]{s}\)^2 + 2(p_{uv} \cdot e_3)\de[u]{s}\de[v]{s} + (p_{vv} \cdot e_3)\(\de[v]{s}\)^2\\
            &= h_{11}\(\di[u]{s}\)^2 + 2h_{12}\de[u]{s}\de[v]{s} + h_{22}\(\di[v]{s}\)^2\\
            &= \frac{h_{11}du^2 + 2h_{12}dudv + h_{22}dv^2}{g_{11}du^2 + 2g_{12}dudv + g_{22}dv^2}\\
        \end{align*}
        となる。つまり曲線上の点$p$における法曲率は$\lambda = dv/du$のみに依存する。

    \subsection{主曲率}
        曲面上の点において、法線ベクトルを含む平面と曲面が交わってできる曲線を法切断または直截線という。点$p$における法切断の測地的曲率は0なので、曲率は法曲率に等しい。法切断の曲率が最大値最小値をとるとき、その曲率$k_1,k_2$を主曲率、接ベクトルを主方向$X_1,X_2$と呼ぶ。法曲率が$k$となる条件は、
        \begin{gather*}
            k = \frac{h_{11} + 2h_{12}\lambda + h_{22}\lambda^2}{g_{11} + 2g_{12}\lambda + g_{22}\lambda^2}\\
            (h_{22} - kg_{22})\lambda^2 + 2(h_{12} - kg_{12})\lambda + (h_{11} - kg_{11}) = 0
        \end{gather*}
        の解が存在することであり、この二次方程式の判別式$D$が0以上になることである。
        \begin{align*}
            \frac{D}{4} &= (h_{12} - kg_{12})^2 - (h_{11} - kg_{11})(h_{22} - kg_{22})\\
            &= (g_{12}^2 - g_{11}g_{22})k^2 + (g_{11}h_{22} - 2g_{12}h_{12} + g_{22}h_{11})k + (h_{12}^2 - h_{11}h_{22}) \geq 0
        \end{align*}
        $g$は正定値行列なので$g_{12}^2 - g_{11}g_{22} < 0$である。よって$k$が極値を取るのは$D = 0$のときである。
        \begin{align*}
            (h_{11} - kg_{11})(h_{22} - kg_{22}) - (h_{12} - kg_{12})(h_{21} - kg_{21}) = 0\\
            \det(II - kI) = 0\\
            \det(I^{-1}II - kE) = 0\\
        \end{align*}
        となる。これは$I^{-1}II$の固有方程式である。$S = I^{-1}II$をシェイプ作用素という。つまり二つの主曲率はシェイプ作用素の固有値である。

        一般の方向における法切断の曲率を求める。
        \begin{lem}
            平面曲線$y = y(x)$で$y'(0) = 0$のとき、原点における曲率は、$y''(0)$である$($平面曲線の場合、曲率に符号を付ける$)$。
        \end{lem}
        \begin{proof}
            $p = (x,y)$、弧長パラメータを$s$とすれば曲率は$\de[p]{s}$である。
                \[\de[p]{x} = (1,y'), \de[s]{x} = \sqrt{1 + y'^2}\]
            より、
            \begin{align*}
                \de[p]{s} &= \frac{(1, y')}{\sqrt{1 + y'^2}}\\
                \de{x}\de[p]{s} &= \frac{(0, y'')\sqrt{1 + y'^2} - (1, y')y'y''/\sqrt{1 + y'^2}}{1 + y'^2}\\
            \end{align*}
            $y'(0) = 0$なので
                \[\de[p]{s} = \frac{(0, y'')(1 + y'^2) - (1, y')y'y''}{(1 + y'^2)^2} = (0,y'')\]
            よって原点における曲率は$y''$となる。
        \end{proof}

        \begin{thm}{オイラーの定理(微分幾何)}
            二つの主方向は直交し、更に$X_1$に対して角$\theta$をなす法切断の曲率を$k_\theta$とすれば、
                \[k_\theta = k_1\cos^2\theta + k_2\sin^2\theta\]
            である。
        \end{thm}
        \begin{proof}
            曲面上に正規直交枠をとる。補題より、$x,y$軸の法切断の曲率は偏微分となり$\pd[^2z]{x^2},\pd[^2z]{y^2}$である。$x$軸と角$\theta$をなす方向を$v = (\cos\theta, \sin\theta)$とすれば、曲率$k_\theta$は方向微分となり、
            \begin{align*}
                k_\theta &= \pd[^2z]{v^2}\\
                &= \pd{v}(z_x\cos\theta + z_y\sin\theta)\\
                &= (z_{xx}\cos\theta + z_{xy}\sin\theta, z_{xy}\cos\theta + z_{yy}\sin\theta) \cdot (\cos\theta, \sin\theta)\\
                &= z_{xx}\cos^2\theta + 2z_{xy}\sin\theta\cos\theta + z_{yy}\sin^2\theta
            \end{align*}
            この二次形式の最大値と最小値は行列$\begin{pmatrix}z_{xx} & z_{xy}\\ z_{yx} & z_{yy}\end{pmatrix}$の固有値であり、主方向はそれぞれの固有値の固有ベクトルである。対称行列の固有ベクトルは直交するので主方向も直交する。この固有ベクトルで主軸変換を行えば、主方向との成す角を新たに$\theta$とおいて
                \[k_\theta = k_1\cos^2\theta + k_2\sin^2\theta\]
        \end{proof}

    \subsection{ガウス曲率と平均曲率}
        二つの主曲率$k_1,k_2$の積をガウス曲率$K$、平均を平均曲率$H$という。主曲率の満たす二次方程式の解と係数の関係より、
        \begin{align*}
            K &= k_1k_2 = \det S = \frac{h_{11}h_{22} - h_{12}^2}{g_{11}g_{22} - g_{12}^2}\\
            2H &= k_1 + k_2 = \tr S = \frac{g_{11}h_{22} - 2g_{12}h_{12} + g_{22}h_{11}}{g_{11}g_{22} - g_{12}^2}\\
        \end{align*}
        となる。ガウス曲率が正の点を楕円点、負の点を双曲点という。また0の点を放物点と呼ぶこともある。

\section{曲面の方程式}

    まずクリストッフェル記号を導入する。
    \begin{align*}
        [jk,l] &= \frac{1}{2}\(\pd[g_{kl}]{u^j} + \pd[g_{jl}]{u^k} - \pd[g_{jk}]{u^l}\)\\
        \chr{i}{jk} &= \frac{1}{2}g^{ih}\(\pd[g_{jh}]{u^k} + \pd[g_{kh}]{u^j} - \pd[g_{jk}]{u^h}\)
    \end{align*}
    ここで$[jk,l] = g_{il}\chr{i}{jk}, \chr{i}{jk} = g^{ih}[jk,h]$が成り立っている。$[jk,l], \chr{i}{jk}$をそれぞれ第一種及び第二種のクリストッフェル記号という。

    \subsection{ガウス=ワインガルテンの公式}
        曲線におけるフレネ=セレの公式に当たるものを導出する。
        
        ガウス標構$\{B_1, B_2, n\}$をとる。ここで
            \[n = \frac{B_1 \times B_2}{|B_1 \times B_2|} = \frac{B_1 \times B_2}{\sqrt{g_{11}g_{22} - g_{12}^2}}\]
        である。三つのベクトルは線形独立なので、
            \[\pd[B_j]{u^k} = \Gamma_{jk}^iB_i + h_{jk}n\]
        と書ける。$h_{jk} = \pd[B_j]{u^k} \cdot n$は定義より第二基本量である。また$g_{ij} = B_i \cdot B_j$を微分した$\pd[B_i]{u^k} \cdot B_j + B_i\cdot \pd[B_j]{u^k} = \pd[g_{ij}]{u^k}$に代入して、
            \[\Gamma_{ik}^hg_{hj} + \Gamma_{jk}^hg_{ih} = \pd[g_{ij}]{u^k}\]
        を得る。$\Gamma_{jk|i} = \Gamma_{jk}^hg_{ih}$と置けば、
            \[\Gamma_{ik|j} + \Gamma_{jk|i} = \pd[g_{ij}]{u^k}\]
        となる。$\pd[B_j]{u^k} = \ppd{p}{u^j}{u^k}$より$j,k$に関して対称である。そこで上の式の添え字を循環的に入れ替えて、$\Gamma_{jk|i}$に関する連立方程式とみれば、
        \begin{align*}
            \Gamma_{jk|i} &= \frac{1}{2}\(\pd[g_{ki}]{u^j} + \pd[g_{ij}]{u^k} - \pd[g_{jk}]{u^i}\) = [jk,i]\\
            \Gamma_{jk}^i &= g^{ih}\Gamma_{jk|h} = \chr{i}{jk}\\
        \end{align*}
        となることが分かる。次に$n \cdot n = 1$を微分すると$\pd[n]{u^j} \cdot n = 0$だから
            \[\pd[n]{u^k} = r_k^hB_h\]
        となる$r_j^h$が決まる。$B_i \cdot n = 0$を微分した$\pd[B_i]{u^k} \cdot n + B_i \cdot \pd[n]{u^k} = 0$に先程の式と代入して、
            \[h_{ij} + r_j^hg_{ih} = 0\]
        これは行列の積を成分ごとに表したものなので、両辺に右から$g$の逆行列をかけて、
            \[r_k^i = -h_{kl}g^{li}\]
        二つを合わせて
        \begin{align*}
            \pd[B_j]{u^k} &= \chr{i}{jk}B_i + h_{jk}n \tag{ガウスの公式}\\
            \pd[n]{u^k} &= -h_{kl}g^{li}B_i \tag{ワインガルテンの公式}
        \end{align*}

    \subsection{ガウス=コダッチの方程式}
        ガウス=ワインガルテンの公式の係数は第一及び第二基本量から求められる。この方程式を解くことで求められたガウス標構$\{B_1, B_2, n\}$が実際の曲面と矛盾しないためには以下の条件が必要となる。
        \begin{align*}
            \ppd{p}{u^j}{u^i} &= \pd{p}{u^i}{u^j}\\
            \ppd{B_i}{u^k}{u^j} &= \pd{B_i}{u^j}{u^k}\\
            \ppd{n}{u^j}{u^i} &= \ppd{n}{u^i}{u^j}
        \end{align*}
        である。第一式に関しては既に成り立っている。なぜなら
            \[\ppd{p}{u^j}{u^i} = \pd[B_i]{u^j} = \chr{h}{ij}B_h + h_{ij}n\]
        は$i,j$に関して対称だからである。次に第二式の条件を求める。
        \begin{align*}
            \ppd{B_i}{u^k}{u^j}
            &= \pd{u^k}\(\chr{h}{ij}B_h + h_{ij}n\)\\
            &= \pd{u^k}\chr{h}{ij}B_h + \chr{h}{ij}\pd[B_h]{u^k} + \pd[h_{ij}]{u^k}n + h_{ij}\pd[n]{u^k}\\
            &= \sum_h \pd{u^k}\chr{h}{ij}B_h + \sum_{h,l} \chr{h}{ij}\chr{l}{hk}B_l + \sum_h \chr{h}{ij}h_{hk}n\\
            &\qquad + \pd[h_{ij}]{u^k}n + h_{ij}\sum_{l,h}(-h_{kl}g^{lh}B_h)\\
            \intertext{第二項で添え字$h,l$を入れ替えると}
            &= \llr{\pd{u^k}\chr{h}{ij} + \chr{h}{kl}\chr{l}{ij} - h_{ij}h_{kl}g^{lh}}B_h\\
            &\qquad + \llr{\pd[h_{ij}]{u^k} + \chr{h}{ij}h_{hk}}n\\
        \end{align*}
        となる。これが$j,k$に関して対称となる。$B_h,n$は線形独立なので、
        \begin{gather*}
            R^h_{ijk} = h_{ik}h_{jl}g^{lh} - h_{ij}h_{kl}g^{lh} \tag{ガウスの方程式}\\
            \pd[h_{ij}]{u^k} - \pd[h_{ik}]{u^j} + \chr{h}{ij}h_{hk} - \chr{h}{ik}h_{hj} = 0 \tag{コダッチの方程式}
        \end{gather*}
        となる。ただし
            \[R^h_{ijk} = \pd{u^j}\chr{h}{ik} - \pd{u^k}\chr{h}{ij} + \chr{l}{ik}\chr{h}{jl} - \chr{l}{ij}\chr{h}{kl}\]
        である。また両辺に$g^{hl}$をかけて縮約を取ったものを、
            \[R_{lijk} = g^{hl}R^h_{ijk} = \pd{u^j}\chr{h}{ik}g^{hl} - \pd{u^k}\chr{h}{ij}g^{hl} + \chr{a}{ik}\chr{h}{ja}g^{hl} - \chr{a}{ij}\chr{h}{ka}g^{hl}\]
        これらをリーマン曲率テンソルと呼ぶ。リーマン曲率テンソルを用いてガウスの方程式を表すと、
        \begin{align*}
            g^{ha}R^h_{ijk} &= h_{ik}h_{ja}g^{ah}g^{hl} - h_{ij}h_{ka}g^{ah}g^{hl}\\
            R_{lijk} &= h_{ik}h_{ja}\delta_l^a - h_{ij}h_{ka}\delta_l^a\\
            &= h_{ik}h_{jl} - h_{ij}h_{kl}
        \end{align*}
        また第三式については、コダッチの方程式と同じものが導かれるため、条件はこれで十分である。よって次の定理が成り立つ。
        \begin{thm}{曲面論の基本定理}
            対称テンソル$g_{ij}, h_{ij}$が与えられ、$g_{ij}$は正定値であるとする。これらがガウス=コダッチの方程式を満たすとき、これらを第一及び第二基本量とする曲面$p(u,v)$が剛体運動を除いて一意的に存在する。
        \end{thm}


    \subsection{驚異の定理}
        上の式は右辺が第二基本量のみに依存している。元々リーマン曲率テンソル及びその定義に含まれるクリストッフェル記号は第一基本量のみから求めることができた。つまり第一基本テンソルと第二基本テンソルは独立ではない。特に
            \[R_{1212} = h_{11}h_{22} - h_{12}^2\]
        である。これを用いると、
            \[K = \frac{h_{11}h_{22} - h_{12}^2}{g_{11}g_{22} - g_{12}^2} = \frac{R_{1212}}{g_{11}g_{22}-g_{12}^2}\]
        となる。曲面のガウス曲率を第一基本テンソルだけから求めることができる。これを驚異の定理(Theorem Egregium)という。元々外在的な量から定義されたガウス曲率が、内在的な量のみから決定できることが示された。外界の情報を用いることなく空間の曲がり方を考察することができることを示唆している。


    \paragraph{等長写像}
        二つの曲面が距離を保ったまま変形できるとき、等長的であると言い、そのような写像を等長写像という。等長的な二つの曲面は、伸縮せずに折り曲げることで変形できる。折り紙は平面と等長的な曲面(可展面)を作る遊びである(ウェットフォールディングをする場合はこの限りではない)。等長変換によって第一基本形式およびガウス曲率は不変である。対偶を取ると、ガウス曲率の一致しない曲面は等長的でない。平面のガウス曲率は任意の点で0であり、半径$r$の球面では$\frac{1}{r^2}$なので、球面を歪ませることなく平面に展開することはできない。つまり地球の正確な平面図を作成することは不可能である。
    \paragraph{共形写像}
        始点が同じ任意の二つのベクトルの成す角を保存する写像を共形写像(等角写像)という。
    \paragraph{等温座標}
        パラメータ$(u,v)$から平面への共形写像が存在するとき、$(u,v)$を等温座標という。これは第一基本形式が$ds^2 = E(du^2+dv^2)$となることと同値である。複素数$z=u+iv$を使えば、$ds^2=E|dz|^2$とも書ける。曲面上の任意の点で局所的には等温座標が存在することが証明できる。ここで、
        \begin{align*}
            \partial = \pd{z} = \frac{1}{2}\(\pd{u} - i\pd{v}\)\\
            \overline{\partial} = \pd{\overline{z}} = \frac{1}{2}\lr{\pd[]{u}+i\pd[]{v}}
        \end{align*}
        とするとガウス曲率は、
            \[K = -\frac{2\partial\overline{\partial}\log E}{E} = -\frac{\Delta \log E}{2E}\]
        となる。 
       
\section{ガウス=ボネの定理}
    \begin{thm}{ガウス=ボネの定理}
        $M$をコンパクトな二次元リーマン多様体とする。$K$を$M$のガウス曲率、$k_g$を$\partial M$の測地的曲率とすると、
            \[\int_M KdA + \int_{\partial M} k_gds = 2\pi\chi(M)\]
        ただし$\chi(M)$は$M$のオイラー標数である。
    \end{thm}