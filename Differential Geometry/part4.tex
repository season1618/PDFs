\part{リーマン幾何学}

\begin{dfn}{レヴィ=チヴィタ接続,リーマン幾何学の基本定理}
    リーマン多様体$(M, g)$について、捩れのない計量接続(計量テンソルを保存する接続)がただ一つ存在する。これをレヴィ=チヴィタ接続という。
    \begin{enumerate}
        \item $\nabra_XY - \nabra_YX = [X, Y]$
        \item $Xg(Y, Z) = g(\nabra_XY, Z) + g(Y, \nabra_XZ)$
    \end{enumerate}
\end{dfn}

\section{共変微分}
    反変ベクトル$A^i(X^j)$を座標変換したものが$a^i(x^j)$であるとする。
    \begin{align*}
        \pd[a^i]{x^j} &= \pd{x^j}\lr{\pd[x^i]{X^l}A^l}\\
        &= \ppd{x^i}{x^j}{X^l}A^l
        +\pd[x^i]{X^l}\pd[X^k]{x^j}\pd[A^l]{X^k}\\
        \intertext{共変ベクトルも同様に、}
        \pd[a_i]{x^j} &= \pd{x^j}\lr{\pd[X^l]{x^i}A^l}\\
        &= \ppd{X^l}{x^i}{x^j}A_l
        + \pd[X^l]{x^i}\pd[X^k]{x^j}\pd[A_l]{X^k}
    \end{align*}
    第二項だけを見ればそれぞれ混合テンソル、共変テンソルのようになっている。
    \begin{align*}
        \na_ja^i &= \pd[a^i]{x^j}-\ppd{x^i}{x^j}{X^l}A^l\\
        &= \pd[a^i]{x^j}-\ppd{x^i}{x^j}{X^l}\pd[X^l]{x^k}a^k\\
        \na_ja_i &= \pd[a_i]{x^j}-\ppd{X^l}{x^i}{x^j}A_l\\
        &= \pd[a_i]{x^j}-\ppd{X^l}{x^i}{x^j}\pd[x^k]{X^l}a_k
    \end{align*}
    とすれば$X$と$x$の二つの座標系間の変換で不変となる。これを共変微分という。第二項の$a^k,a_k$の係数は接続係数またはアフィン係数と呼ばれる。このままでは他の座標系に依存してしまうので$X$をデカルト座標で固定する。しかし、曲がった空間ではデカルト座標との関係が不定であり、そもそも存在を前提とするわけにはいかないので、計量を使って書き直す。ユークリッド計量を$\delta_{ij}$(クロネッカーのデルタ)とすると、
        \[g_{ij} = \pd[X^m]{x^i}\pd[X^n]{x^j}\delta_{mn}\]
    これを微分して、
        \[\pd[g_{ij}]{x^k} = \delta_{mn}
        \lr{\ppd{X^m}{x^i}{x^k}\pd[X^n]{x^j}
        +\pd[X^m]{x^i}\ppd{X^n}{x^j}{x^k}}\]
    計量は対称テンソルなので、添え字を巡回的に入れ替えて足し引きする。
        \[\rec{2}\lr{\pd[g_{jk}]{x^i}+\pd[g_{ki}]{x^j}-\pd[g_{ij}]{x^k}}
        = \delta_{mn}\ppd{X^m}{x^i}{x^j}\pd[X^n]{x^k}\]
    とすると左辺は第一種クリストッフェル記号そのものであり$\Ga_{kij}$と書く。第二種クリストッフェル記号を$\Ga^l_{ij}$と書き、
    \begin{align*}
        \Ga^l_{ij} &= g^{lk}\Ga_{kij}\\
        &= g^{lk}\delta_{mn}\ppd{X^m}{x^i}{x^j}\pd[X^n]{x^k}\\
        &= \pd[x^l]{X^u}\pd[x^k]{X^v}\delta^{uv}
        \delta_{mn}\ppd{X^m}{x^i}{x^j}\pd[X^n]{x^k}\\
        \intertext{$\delta^{ij},\delta_{ij}=0(i\neq j)$なので}
        &= \pd[x^l]{X^u}\pd[x^k]{X^u}\ppd{X^m}{x^i}{x^j}\pd[X^m]{x^k}\\
        &= \ppd{X^m}{x^i}{x^j}\pd[x^l]{X^u}\pd[X^m]{X^u}\\
        \intertext{$\pd[X^m]{X^u}=\delta^m_u$より}
        &= \ppd{X^m}{x^i}{x^j}\pd[x^l]{X^m}
    \end{align*}
    となって共変ベクトルの共変微分の接続係数となる。一方、反変ベクトルの共変微分の接続係数は、
    \begin{align*}
        \ppd{x^i}{x^j}{X^l}\pd[X^l]{x^k} &= \pd{x^j}\lr{\pd[x^i]{X^l}}\pd[X^l]{x^k}\\
        &= \pd{x^j}\lr{\pd[x^i]{X^l}\pd[X^l]{x^k}}-\pd[x^i]{X^l}\pd{x^j}\pd[X^l]{x^k}\\
        &= \ppd{x^i}{x^j}{x^k}-\ppd{X^l}{x^j}{x^k}\pd[x^i]{X^l}\\
        &= -\Ga^i_{jk}
    \end{align*}
    となる。つまり接続係数はどちらもクリストッフェル記号で表すことができる。共変微分の定義を書き直すと、
    \begin{gather*}
        \na_ja^i = \pd[a^i]{x^j}+\Ga^i_{jk}a^k\\
        \na_ja_i = \pd[a_i]{x^j}-\Ga^k_{ij}a_k
    \end{gather*}
    となる。一般のテンソルに対しても同様に定義することができる。二階のテンソルの場合は、
    \begin{align*}
        \na_kT^{ij} &= \pd[T^{ij}]{x^k}+\Ga^i_{km}T^{mj}+\Ga^j_{km}T^{mj}\\
        \na_kT^i_j &= \pd[T^i_j]{x^k}+\Ga^i_{km}T^m_j-\Ga^m_{jk}T^i_m\\
        \na_kT_{ij} &= \pd[T_{ij}]{x^k}-\Ga^m_{ik}T_{mj}-\Ga^m_{jk}T_{im}
    \end{align*}
    となる。

    以下に共変微分の公式を書いておく。ただし添え字は省略する。
    \begin{itembox}[l]{共変微分の公式}\begin{align*}
        \na_i(T_1+T_2) &= \na_iT_1+\na_iT_2\\
        \na_i(kT) &= k\na_iT\ (kはスカラー)\\
        \na_i(T_1T_2) &= (\na_iT_1)T_2+T_1(\na_iT_2)
    \end{align*}\end{itembox}
    \paragraph{計量条件}
        計量テンソルを共変微分すると、デカルト座標で考えることにより、
            \[\na_kg_{ij} = \pa_k\delta_{ij} = 0\]
        となる。これを計量条件という。反変の計量テンソルについても同様に0となる。


\section{曲率}
    共変微分の交換は、
    \begin{align*}
        [\na_l,\na_k]A_j &= \na_l\na_kA_j-\na_k\na_lA_j\\
        &= [\pa_k\Ga^i_{jl}-\pa_i\Ga^l_{jk}
        +\Ga^m_{jl}\Ga^i_{mk}-\Ga^m_{jk}\Ga^i_{ml}]A_i
    \end{align*}
    である。係数は一階反変三階共変のテンソルであり、
        \[R^i_{jkl}
        =\pa_k\Ga^i_{jl}-\pa_l\Ga^i_{jk}
        +\Ga^m_{jl}\Ga^i_{km}-\Ga^m_{jk}\Ga^i_{lm}\]
    及びこれを縮約した
        \[R_{ijkl} = g_{ih}R^h_{jkl}\]
    をリーマン曲率テンソルと呼ぶ。さらに
    \begin{gather*}
        R_{ij} = R^k_{ikj}\\
        R = g^{ij}R_{ij}
    \end{gather*}
    をそれぞれリッチテンソル、リッチスカラー(スカラー曲率)という。


\section{ビアンキの恒等式}
    共変微分の交換関係より、$\na_jA_i = a_jb_i$と書けるので
    \begin{align*}
        [\na_k,\na_l]\na_jA_i &= (\na_l\na_k-\na_k\na_l)(a_jb_i)\\
        &= \na_l\llr{(\na_ka_j)b_i+a_j(\na_kb_i)}-\na_k\llr{(\na_la_j)b_i+a_j(\na_lb_i)}\\
        &= ([\na_k,\na_l]a_j)b_i+a_j([\na_k,\na_l]b_i)\\
        &= R^h_{jkl}a_hb_i+R^h_{ikl}a_jb_h\\
        &= R^h_{jkl}\na_hA_i+R^h_{ikl}\na_jA_h
    \end{align*}
    また、
    \begin{align*}
        \na_j[\na_k,\na_l]A_i &= \na_j(R^h_{ikl}A_h)\\
        &= \na_jR^h_{ikl}A_h+R^h_{ikl}\na_jA_h
    \end{align*}
    辺々引けば、
        \[[\na_j,[\na_k,\na_l]]A_i = R^h_{jkl}\na_hA_i-\na_jR^h_{ikl}A_h\]
    となる。添え字を循環的に入れ替えて足すと、左辺はヤコビの恒等式と呼ばれるものになり、恒等的に0である。つまり、
        \[(R^h_{jkl}+R^h_{klj}+R^h_{ljk})\na_hA_i
        -(R^h_{ijk,l}+R^h_{ikl,j}+R^h_{ilj,k})A_h = 0\]
    任意の$A_i$についてこれが成り立つので、
    \begin{gather*}
        R^i_{jkl}+R^i_{klj}+R^i_{ljk} = 0\\
        R^h_{ijk,l}+R^h_{ikl,j}+R^h_{ilj,k} = 0
    \end{gather*}
    である。これをそれぞれビアンキの第一及び第二恒等式という。

    ビアンキの第二恒等式を変形する。
    \begin{align*}
        \na_kR^h_{ilj}-\na_jR^h_{ilk}+\na_lR^h_{ijk} &= 0\\
        \intertext{$h=l$として縮約し、第三項を書き換える。}
        \na_kR_{ij}-\na_jR^h_{ik}+\na_lg^{ih}R_{hijk} &= 0\\
        \intertext{両辺に$g^{ij}$を掛ける。計量条件より共変微分の中に入れることが
        できて、}
        \na_kR-\na_jR^j_k-\na_lg^{ih}R^j_{hjk}
        &= \na_kR-\na_jR^j_k-\na_lR^i_k\\
        &= \na_kR-2\na_jR^j_k\\
        \intertext{第一項を書き換えて、}
        &= \na_j(\delta^j_kR)-2\na_jR^j_k = 0
    \end{align*}
    となる。
        \[G^j_k = R^j_k-\rec{2}\delta^j_kR\]
    と置く。これをアインシュタインテンソルという。添え字を上げて二階の反変テンソルにすれば、
        \[G^{ij} = g^{ik}G^j_k = R^{ij}-\rec{2}Rg^{ij}\]
    計量条件より、
        \[\na_jG^{ij} = 0\]
    である。