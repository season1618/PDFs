\section{測地線}
    曲面上の二点を結ぶ曲線のうち、最も短いものを最短線という。曲線が最短線となる必要条件を考える。助変数を$(u_1(t),u_2(t))(a\leq t\leq b)$とし、対応する端点を$A,B$とする。すると$AB$の長さは汎関数
        \[s[r(t)] = \int_a^b F(u(t),\dot{u}(t))dt\]
    で与えられる。ただし$F(u(t),\dot{u}(t)) = \sqrt{g_{ij}(u)\dot{u}^i\dot{u}^j}$である。また$ds = Fdt$である。汎関数$s[r(t)]$が極値をとる条件は$F$がEuler-Lagrange方程式を満たすことである。そのような曲線を測地線と呼ぶ。最短線は測地線だが測地線は最短線ではない。まずEuler-Lagrange方程式は、
        \[\de{t}\pd[F]{\dot{u}^i} - \pd[F]{u^i} = 0\]
    で表される。第一項は、
        \[\pd[F]{\dot{u}^l} = \frac{1}{2F}g_{jk}\left\{\pd[\dot{u}^j]{\dot{u}^l}\dot{u}^k + \dot{u}^j\pd[\dot{u}^k]{\dot{u}^l}\right\}\]
    $\pd[\dot{u}^j]{\dot{u}^l}$は$j = l$のときに限り1なので、
    \begin{align*}
        &= \frac{1}{2F}\(\sum_k g_{lk}\dot{u}^k + \sum_j g_{jl}\dot{u}^j\)\\
        &= \frac{1}{F}g_{il}\dot{u}^i = g_{il}\di[u^i]{s}\\
    \end{align*}
    従って
    \begin{align*}
        \de{t}\(\pd[F]{\dot{u}^l}\) &= \(\sum_i g_{il}\de[^2u^i]{s^2} + \sum_{j,k} \pd[g_{jl}]{u^k}\de[u^k]{s}\de[u^j]{s}\)\de[s]{t}\\
    \end{align*}
    $\pd[g_{jl}]{u^k} = [jk,l] + [lk,j]$より、
    \begin{align*}
        &= \llr{g_{il}\de[u^i]{s} + ([jk,l] + [lk,j])\de[u^j]{s}\de[u^k]{s}}\de[s]{t}\\
    \end{align*}
    そして第二項は、
    \begin{align*}
        \pd[F]{u^l} &= \frac{1}{2F}\pd[g_{jk}]{u^l}\dot{u}^j\dot{u}^k\\
        \intertext{$[jk,l] = [kj,l], \pd[g_{jk}]{u^l} = [jl,k] + [kl,j]$であることに注意すれば、}
        &= \frac{1}{2F}([jl,k] + [kl,j])\dot{u}^j\dot{u}^k\\
        &= \frac{1}{F}[kl,j]\dot{u}^j\dot{u}^k\\
        &= [kl,j]\de[u^j]{s}\de[u^k]{s}\de[s]{t}\\
    \end{align*}
    なので、
    \begin{align*}
        \de{t}\pd[F]{\dot{u}^l} - \pd[F]{u^l}
        &= \(g_{il}\de[^2u^i]{s^2} + [jk,l]\de[u^j]{s}\de[u^k]{s}\)\de[s]{t}\\
        &= g_{il}\(\de[^2u^i]{s^2}+\chr{i}{jk}\de[u^j]{s}\de[u^k]{s}\)\de[s]{t}\\
        &= 0\\
    \end{align*}
    $\det g_{il} \neq 0, ds/dt = F \neq 0$であるので、
        \[\de[^2u^i]{s^2} + \chr{i}{jk}\de[u^j]{s}\de[u^k]{s} = 0\quad (i=1,2,...,n)\]
    これを測地線の方程式という。またこれを
    \begin{align*}
        \de[u^i]{s} &= v^i\\
        \de[v^i]{s} &= -\chr{i}{jk}v^jv^k
    \end{align*}
    と書き換えると、1階の連立常微分方程式となるので、次の定理が導かれる。
    \begin{thm}{測地線}
        リーマン多様体$M$の任意の点で任意の方向にただ1本の測地線が引ける。
    \end{thm}

    \subsection{Weierstrassの表現}
        二次元リーマン多様体、つまり曲面の場合を考える。このとき$k > 0$として$F(u,v,k\dot{u},k\dot{v}) = kF(u,v,\dot{u},\dot{v})$となって$F$は$\dot{u},\dot{v}$について正斉次である。この式を$k$で微分し、$k = 1$とすると、
            \[\pd[F]{\dot{u}}\dot{u} + \pd[F]{\dot{v}}\dot{v} = F\]
        となる。両辺を$\dot{u^i}$で微分して
        \begin{align*}
            \pd[^2F]{\dot{u}^2}\dot{u}+\ppd{F}{\dot{v}}{\dot{u}}\dot{v} = 0\\
            \ppd{F}{\dot{u}}{\dot{v}}\dot{u}+\pd[^2F]{\dot{v}^2}\dot{v} = 0\\
        \end{align*}
        が導かれ、
            \[\pd[^2F]{\dot{u}^2} : \ppd{F}{\dot{u}}{\dot{v}} : \pd[^2F]{\dot{v}^2} = \dot{v}^2 : -\dot{u}\dot{v} : \dot{u}^2\]
        となる。比例因子を$c$として
        \begin{align*}
            \pd[^2F]{\dot{u}^2} = c\dot{v}^2\\
            \ppd{F}{\dot{u}}{\dot{v}} &= -c\dot{u}\dot{v}\\
            \pd[^2F]{\dot{v}^2} = c\dot{u}^2\\
        \end{align*}
        また
        \begin{align*}
            \pd[F]{\dot{u}} = \frac{g_{11}\dot{u} + g_{12}\dot{v}}{F}\\
            \pd[F]{\dot{v}} = \frac{g_{21}\dot{u} + g_{22}\dot{v}}{F}\\
        \end{align*}
        なので
        \begin{align*}
            c &= \frac{F_{\dot{u}\dot{u}}}{\dot{v}^2}\\
            &= \frac{1}{\dot{v}^2}\frac{g_{11}F - (g_{11}\dot{u} + g_{12}\dot{v})F_{\dot{u}}}{F^2}\\
            &= \frac{1}{\dot{v}^2}\frac{g_{11}F^2 - (g_{11}\dot{u} + g_{12}\dot{v})^2}{F^3}\\
            &= \frac{g_{11}g_{22} - g_{12}^2}{F^3}
        \end{align*}
        これをEuler-Lagrange方程式に代入すると
        \begin{align*}
            F_{u} - \de{t}F{\dot{u}} = \dot{v}T
            F_{v} - \de{t}F{\dot{v}} = -\dot{u}T
        \end{align*}
        ただし
            \[T = F_{u\dot{v}} - F_{\dot{u}v} + c(\dot{u}\ddot{v} - \dot{v}\ddot{u})\]
        従って曲面の場合、測地線の方程式は$T = 0$と同値である。これをWeierstrassの表現という。
            
\section{極小曲面}