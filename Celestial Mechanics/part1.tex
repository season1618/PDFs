\repart{天文学史}

\section{古代の天文学}
    初めて地球の大きさを測ろうと試みたのはエラトステネスだった。またアリスタルコス(B.C.310-230頃)は半月になったときの太陽、地球、月の位置関係から太陽と地球の距離が地球と月の距離の18-20倍であると推測している。数値は大きく違うが方法自体は正しい。一方で天体間の絶対的な距離を測定するには地球上の二点から三角測量する必要があるが、当時の観測技術ではまともな数値は得られなかった。ニュートンの時代までは基本的に天体間の相対的な距離のみを用いて計算が行われていた。
	
	円錐曲線論で有名なアポロ二ウス(B.C.250-200頃)は周転円を導入し、惑星の順行・逆行及びそのときの明るさの変化、惑星の運行の不規則性を説明した。ヒッパルコスは離心円の導入や周転円を重ねることで惑星の軌道をより正確に近似した。プトレマイオス(トレミー)は『アルマゲスト』において、天動説を含めた当時の天文学を体系的に解説している。プトレマイオスは、惑星の不規則な運行の説明として、周転円と離心円に加え、円の中心からずれた点からの角速度が一定になるような運行を考えた。このずれた点のことをエカントという。彼はさらに出差(月の軌道の摂動・歪み)を発見し、周転円と離心円によって説明した。天動説ではそれぞれの惑星の軌道半径を決定できず任意に拡大縮小できる。『アルマゲスト』においては、基本的に恒星に対して一周する時間の短い順に月、水星、金星、太陽、木星、土星の順に並べられた。水星、金星、太陽は恒星に対して一周する時間がほぼ一年と明確な違いはないが、最も支持されたのがこの配列であった。また星の見かけの明るさ(相対等級)を目に見える一番明るい星と一番暗い星の間の6段階に分けている。プトレマイオス以後の天文学は周転円・離心円・エカントをさらに複雑に組み合わせたモデルが考えられたが、惑星の運行と正確に一致するようなものは得られなかった。
\section{近代の宇宙観}
	地動説を提唱したコペルニクスは地球の運動を除いては基本的に伝統的な宇宙観を受け継いでいる。『天体の回転について』の形式はプトレマイオスの『アルマゲスト』を踏襲している。コペルニクスによれば、地球は太陽を中心とする回転する天球に固定されており、そのままでは公転とともに地軸も回転してしまう。実際には北極星は動かず地軸は回転していないので、公転と逆向きの円錐運動を考えて地軸の回転を相殺している。コペルニクスの体系では周転円を用いずに惑星の逆行を説明できる。惑星が逆行するときは惑星が地球に最も接近するときであり、必然的に明るさも最大となる。特に外惑星が逆行するときは太陽と反対側の位置にある。プトレマイオスの体系では外惑星の逆行時に太陽の反対側に来るようにするには導円と周転円が連動して動く必要があった。またコペルニクスの体系は内惑星の運行を自然に説明できる。水星や金星はその他の惑星と違い、太陽から一定以上離れることがない。プトレマイオスの体系ではそのために地球と太陽を結ぶ直線状に惑星の周転円の中心を置く必要があった。しかしコペルニクスの体系では、水星と金星は地球より内側の軌道を回っているとすれば太陽から最大離角以上離れることはない。惑星の公転周期は逆行の周期から算出できる。また惑星の軌道半径の比も決定することができる。よって少なくとも定性的にはコペルニクスの体系はプトレマイオスの体系より自然に惑星の運行を説明することができる。しかし惑星の運動をより正確に説明ために結局は周転円や離心円をプトレマイオスの体系の半分程度導入することとなった。一方で正確さはプトレマイオスの体系と比べてあまり違いはなかった。また地動説の証拠である年周視差は当時の技術では観測できず、1838年のベッセルの観測まで待たねばならなかった。\\
	ティコ・ブラーエ(1546-1601)は年周視差を観測できなかったことからコペルニクスの体系を受け入れることはなかったが、その代わりに独自の体系を提案した。ティコの体系では地球は宇宙の中心に静止しており、太陽と月は地球の周りを回っているが、それ以外の惑星は全て太陽の周りを回っている。ティコの体系はコペルニクスの体系と完全に数学的に同等であり、地球の静止以外の全ての性質を受け継いでいる。その損なわれた対称性のために多くのコペルニクス主義者には支持されなかった代わりに、多くの非コペルニクス主義者の支持を集めた。
\section{暦}
	太陽暦が初めて使われたのは古代エジプトとされている。エジプトでは夏季モンスーンによって一年に一度ナイル川が氾濫していた。氾濫の時期がシリウスが日の出直前に上ってくる時期と一致していたことから、彼らは星を観察することで氾濫の時期を予測した(シリウス暦)。彼らは一年が365.25日であることを知っており、4年に一度余分な日を設けた。\\
	ヒッパルコス(B.C.190-125頃)はロードス島に天文台を建設し、地球の歳差運動を発見した。地軸は黄道面に対して傾いており、トルクが発生し赤道が移動する。これを赤道の歳差という。また他の惑星の引力の影響から公転面つまり黄道が傾く。これを黄道の歳差という。これらを合わせたものを一般歳差という。黄道の歳差は赤道の歳差に比べて無視できるほど小さい。歳差によって地軸は公転と逆方向に25772年かけて一周することになる。地軸の回転に伴って春分点や北極星などは徐々に移動する。\\
    時間の単位である秒・分・時は天体の運動に依らず厳密に定められている。しかし日・月・年などの回転の周期にはどの星を基準に取るかによって違いがある。\\
    地球の自転周期の場合、太陽に対して一周する時間(正午から次の正午)を太陽日と言い約24時間、恒星に対して一周する時間(春分点の南中から次の南中)を恒星日と言い約23時間56分4秒である。歳差・章動によって恒星の一周する時間は恒星によって異なるので春分点を用いる。地球の公転は一日約1度で、公転と自転は同じ向きだから太陽日は恒星日より、24h/360° = 4mほど長い。地球の自転は潮汐力の影響で遅くなっており、自転周期が伸びている。
    地球の公転周期の場合、太陽に対して再び同じ面を向けるまでの時間(分点・至点から次の分点・至点)を太陽年(回帰年)と言い約365.2422太陽日、軌道上を恒星に対して一周する時間を恒星年と言い約365.2564太陽日である。地球の公転と歳差運動は逆方向だから太陽年は恒星年より360°/25772*4 = 20分24秒ほど短い。
	暦においては生活上重要な日照や季節を考え、太陽日・太陽年を使う。
    月に関しては、公転と自転が同期している。月が太陽に対して一周する時間を朔望月と言い約29.5太陽日、恒星に対して一周する時間を恒星月と言い約27.3太陽日である。地球と月の公転の向きが同じなので、30日*30°/360° = 2日ほど長い
	%閏秒のタイミングは決まっており、追加される場合は8時59分60秒、削除される場合は8時59分59秒である。
	%1582年10月15日金曜日からグレゴリオ暦が施行され、世界中で広く使われている。これは400年に97回の閏年を設けることで当時算出されていた平均太陽年365.2422日を近似するものである。