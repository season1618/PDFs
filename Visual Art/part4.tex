\repart{光学現象}
	\section{太陽}
		大気圏外の太陽光スペクトルは6000Kの黒体放射スペクトルとほぼ一致している。プランクの法則より、黒体から放射される電磁波の分光放射輝度は
			\[I = \frac{2hc^2}{\lambda^5}\rec{e^{hc/\lambda kT} - 1}\]
		で与えられる。
	\section{空}
		太陽が観測者の真上にあり、大気を高さ$h$の直方体とする。また、レイリー散乱を引き起こす微粒子は大気中に一様に分布しているとする。頭上から角$\theta$の地点にある微粒子から観測者の目に届く光の散乱角は$\theta$である。微粒子の数はその方向に大気を横切る線分の長さに比例するので、青空の輝度は
		\begin{align*}
			c \propto \frac{1 + \cos^2\theta}{2}\cdot \frac{h}{\cos\theta}\\
			= \frac{h}{2}\lr{\cos\theta + \rec{\cos\theta}}
		\end{align*}
		となる。$\theta$に依存する項は$\theta = 0$で最小値を取り、以降は単調に増加していく。つまり遠くの景色ほど明るく青みがかっていく。これが空気遠近法の原理である。
	\section{雲}
	\section{海}
		海の青は空の色の反射による効果もあるが、主に水自身の光の吸収によるものであることが分かっている。入射角が大きいほど反射率は高くなるので、地平線に近いほど空の色は濃くなる。
	\section{雨と雪}
	\section{植物}