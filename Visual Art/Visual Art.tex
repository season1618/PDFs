\documentclass{jsarticle}
\usepackage{amssymb,amsmath}
\usepackage{mathrsfs,ascmac}
\begin{document}
\title{視覚芸術とメディアの科学}
\author{高校n年 season07001674}
\date{2020/05/19}
\maketitle
%\tableofcontents
\newcommand{\repart}[1]{\part{#1}\setcounter{section}{0}}
\newcommand{\lr}[1]{\left(#1 \right)}
\newcommand{\mlr}[1]{\left\{#1 \right\}}
\newcommand{\llr}[1]{\left[#1 \right]}
\newcommand{\rec}[1]{\frac{1}{#1}}
\newcommand{\de}[2][]{\frac{d #1}{d #2}}
\newcommand{\dd}[2][]{\frac{d^2 #1}{d #2^2}}
\newcommand{\pd}[2][]{\frac{\partial #1}{\partial #2}}
\newcommand{\ppd}[3]{\frac{\partial^2 #1}{\partial #2\partial #3}}
\newcommand{\pmat}[1]{\begin{pmatrix} #1 \end{pmatrix}}
\newcommand{\co}[1][]{\cos^{#1}\theta}
\newcommand{\si}[1][]{\sin^{#1}\theta}
\newcommand{\grad}{\mathrm{grad}}
\newcommand{\dive}{\mathrm{div}}
\newcommand{\rot}{\mathrm{rot}}
\newcommand{\pa}{\partial}
\newcommand{\na}{\nabla}

\repart{視覚系と光学系}
	\section{視覚系の構造と機能}
		光彩を通った光は水晶体と硝子体を経て網膜へと達する。水晶体は若年層では無色だが、高齢者では黄みを帯び、灰色と青の区別が困難になる。
		網膜は手前から順に神経節細胞、アマクリン細胞、双極細胞、水平細胞、錐体・桿体細胞で構成されている。神経節細胞は視神経に繋がっている。これらの細胞と神経の結合の度合いは網膜の場所によって著しく異なる。中心部に比べて周辺部で生じた信号はかなり集約されて中枢へと伝わるので解像度は低くなる。\\
		入ってきた光は直接錐体と桿体を刺激する。錐体にはL錐体、M錐体、S錐体の三種類あり、それぞれ$564, 534, 420nm$をピークに持っている。錐体は波長によって感度が異なり、それによって色を感じ取ることができる。\\
		受容細胞の信号を受けた水平細胞は周囲の双極細胞の活動を抑制する。これを側抑制という。隣接する二つの受容細胞に異なる強度の光が入ってきたとき、光の強い方が抑制効果が強いため、明暗の差は刺激の差よりも大きくなる。これが明るさ対比や色対比が生じるメカニズムである。\\
		一つの波長のみからなる光を単色光というが、全ての色が単色光で表せるわけではない。例えば白はスペクトル上には存在しない。また特定の色を表現する単色光の組合せも一意ではない。赤と青の混合した光も紫の単色光も同じ色に見える。この現象をメタメリズムという。言い換えればスペクトルからLMSを返す関数は単射ではない。\\
		錐体の波長に対する相対的な感度のグラフを見てみると、錐体を一つだけ刺激するような単色光がないことが分かる。これは純粋な原色は物理的に存在しないということを示している。例えば光を用いてL錐体だけを刺激することはできない。コンピュータに表示されている$(255, 0, 0)$の色も純粋な赤ではない。論理的には可視光を赤、緑、青に分解することはできるが、それぞれを任意の割合で混ぜ合わせることはできない。\\
		\subsection{明るさ}
			物理的な明るさが同じでも感覚的な明るさは波長によって異なる。緑の光は同じ明るさの赤や青に比べて明るく感じる。また暗所と明所における比視感度には少しずれがある。暗所では青が見えやすく、明所では赤が見えやすい。これをプルキンエ現象という。
		\subsection{刺激時間}
			一定強度の光で目を刺激をすると、反応するまでに0.05-0.2秒かかる。これを感覚化時間という。その後反応は低下し、刺激が中断してから0.05-0.2秒は持続する。これを残効時間という。ピーク時の反応は低下後の2倍程度である。反応が生じる速さと大きさはほぼ刺激の強度に依存する。そして感覚がピークに達する時間は波長によっても異なり、長波長の方が短波長より素早い。信号に赤が使われている理由の一種である。\\
			ベンハムの独楽。
	\section{カメラと写真}

\repart{色彩学}
	\section{標準光源}
		CIE(国際照明委員会)は色を測定するのに用いる証明光の取り決めを行っている。1931年に3種類の標準光源を導入し、後に付け加えられた。
		\begin{itemize}
			\item A:色温度2856K。白熱電球に相当。
			\item B:昼光(直接光)相当の光源
			\item C:昼光(間接光)相当の光源
			\item D65:色温度6504K
		\end{itemize}
		光源の色をある温度の黒体から放射される光の色と対応させたとき、その黒体の温度を色温度という。BとC光源はA光源に液状フィルターをかけて得られる光だが、紫外波長域の光を加えた$D_{65}$に取って代わられつつある。$D_{65}$は欧州・北欧における平均的な正午の光(直射日光と晴天の空による拡散光)に対応している。太陽表面におけるスペクトルは5800Kの黒体放射に近いが、X線は大気でほぼ遮断され、紫外線もオゾン層で90%以上カットされる。また可視光や赤外線は大気圏での反射・散乱・吸収により平均4割減衰する。太陽光のスペクトルには、ところどころ大気を構成する元素による吸収線(フラウンホーファー線)が見られる。したがって地上における太陽光は、大まかには紫外領域の波長が弱まった6504Kの黒体放射に相当する。
	\section{表色系}
		色を定量的に表す体系を表色系という。混色系は、色を心理物理量とみなし、色刺激の特性によって表す。顕色系は、色を三つの特徴に従って配列し、間隔を調整して尺度を示すものである。人間には三種類の錐体細胞が備わっているため、数学的には三つの変数で色を表現することができる。
		\subsection{LMS表色系}
			LMS色空間は最も自然な色空間であり、三つの錐体細胞の刺激値をもって色を表現する。LMS色空間は人の複雑な色覚をモデル化しているため、XYZとの単一の変換行列は存在しない。
		\subsection{RGB表色系}
			混色に用いる原色をR(700nm), G(546.1nm), B(435.8nm)の三つの単色光とする。コンピュータのディスプレイで最も多く用いられる。単色光と等色するのに必要な三原色の放射輝度を測定する実験が行われた。輝度の積分が1になるように正規化したものをRGB等色関数(color matching function)という。曲線が負の側に飛び出しているのは、三原色で等色できない単色光があることを示している。錐体細胞の感度曲線は微妙に重なっており、それはRGBの波長でも同じである。従って各錐体の感度を独立に調整することはできず、不可能な色が生まれてしまう。
		\subsection{XYZ表色系}
			RGBを用いる代わりに仮想の原刺激XYZを想定し、あらゆる色をこれらの正量混合で表せるようにしたのが、XYZ表色系である。しかしXYZとLMSは一致しない。人は明るい場所では、同じ強さの赤や青よりも緑の色を圧倒的に強く感じる。そのため波長ごとの明るさの感じ方である比視感度(分光視感効率)はM錐体の感度曲線と似ている。そこでYは輝度とほぼ一致するように設計されている。また白色点において$X = Y = Z$となるように決められた。XYZとRGBの線形変換は次の通りである。
			\begin{align*}
				\begin{bmatrix} X & Y & Z \end{bmatrix}
				=
				\begin{bmatrix}
					2.7689 & 1.7517 & 1.1302 \\
					1.0000 & 4.5907 & 0.0601 \\
					0.0000 & 0.0565 & 5.5943 \\
				\end{bmatrix}
				\begin{bmatrix} R & G & B \end{bmatrix}\\
				\begin{bmatrix} R & G & B \end{bmatrix}
				=
				\begin{bmatrix}
					0.4185 & -0.1587 & -0.0828 \\
					-0.0912 & 0.2524 & 0.0157 \\
					0.0000 & -0.0025 & 0.1786 \\
				\end{bmatrix}
				\begin{bmatrix} X & Y & Z \end{bmatrix}
			\end{align*}
			また分光放射輝度を$L(\lambda)$とすれば
			\begin{align*}
				X = \int L(\lambda)\overline{x}(\lambda)d\lambda\\
				Y = \int L(\lambda)\overline{y}(\lambda)d\lambda\\
				Z = \int L(\lambda)\overline{z}(\lambda)d\lambda
			\end{align*}
		\subsection{xyY表色系}
			XYZ表色系では色度が分かりにくい。そこで
				\[x = \frac{X}{X + Y + Z}, y = \frac{Y}{X + Y + Z}, z = \frac{Z}{X + Y + Z}\]
			として、xyによって表される色度Yの明るさで色を表す。
		\subsection{マンセル表色系}
			色彩を色の三属性(色相・明度・彩度)によって表す。
			\begin{description}
				\item[色相(Hue)]:色の様相。
				\item[明度(Value)]:色の明るさ。
				\item[彩度(Chroma)]:色の鮮やかさ。つまり分光組成の起伏具合である。正確には色空間における中央軸からの距離であり、距離の定義は色空間によって異なる。
			\end{description}
	\section{混色}
		混色には網膜内で起こる生理的混色と網膜の外で起こる物理的混色がある。前者は混合するほど明るくなるので加法混色といい、後者は混合するほど暗くなるので減法混色という。また絵具をパレット上で混ぜ合わせるような場合は、両方の過程が複雑に絡み合っているため着色剤混合という。
		\subsection{加法混色}
			複数の色刺激が網膜の同じ部位に入るとその分光組成は単純に加算されたものとして感じられる。加法混色の法則はグラスマンの法則としてまとめられている。
			\begin{description}
				\item[加算の法則]:混合色の輝度は成分色の輝度の和に等しい。
				\item[比例の法則]:成分色の輝度を同じ数だけ乗じても混合色の色相は変わらない。
				\item[結合の法則]:混合色は混色する順番に依らない。
			\end{description}
			グラスマンの法則は色空間がベクトル空間に似た性質を持っていることを示している。実際逆元の存在を除く全ての条件が満たされている。ここから
			\begin{enumerate}
				\item あらゆる色は互いに独立(どの二つを混ぜても三つ目の色を作れない)な三つの色の混合により得られる。
				\item 三色の混合比が等しければ色度は同じ。
			\end{enumerate}
			が導かれる。新印象派の点描画法は並置的加法混色の応用である。絵具を混ぜると暗くなってしまうため明るい色を表現するのに適さない。そこで純色をそのままキャンバスに載せて網膜上で混色させることで、明るさを損なうことなく色を表現できる。
		\subsection{減法混色}
			光源にフィルターをかけて色を作る場合を考える。二枚のフィルターを重ねたときの分光透過率は個々のフィルターの透過率の積に等しい。したがってこれは乗法混色とでもいうべきものだが、一般に透過率は1未満なので減法混色と呼ばれている。減法混色の三原色にはRGBの補色が用いられる。すなわちC(G + B, シアン)、M(B + R, マゼンタ)、Y(R + G, イエロー)である。このようにすると例えばマゼンタとイエローの混合は
			\begin{align*}
				(R + G)(R + B) &= R(R + G + B) + BG\\
				&= RW + BG \\
				&\fallingdotseq R
			\end{align*}
			となる。
		\subsection{着色剤混合}
			透明ないし半透明な物体の場合、顔料の粒子の大きさが光の波長より十分大きければ入射光の一部は反射される。光は反射されるものと更に深層へと届くものに分かれ、二つの混色が複雑に絡み合った状態となる。

\repart{メディア}
	\section{色素}
		着色に用いる粉末の内、水や油に溶けるものを染料、溶けないものを顔料という。
		染料
		\begin{description}
			\item[天然染料] 多くは動物や植物から抽出した色素である。
			\item[合成染料]
		\end{description}
		顔料
		\begin{description}
			\item[無機顔料] 遷移金属の金属錯体は配位子の影響を受けてd軌道が分裂/縮退することでエネルギー準位に差が生まれ発色する。
				\begin{description}
					\item[天然鉱物顔料] 
					\item[合成無機顔料] 
				\end{description}
			\item[有機顔料] 電子のエネルギー準位に相当する光の多くは紫外領域に属し、分子の電気双極子の振動に相当する光は赤外領域に属する。ところが共役系の非局在化電子は一般にエネルギー準位の差が小さく(井戸型ポテンシャルの下で波動関数を計算すると分かる)、光の吸収波長が長波長側へずれ呈色するものが多い。
				\begin{description}
					\item[アゾ顔料] 
					\item[多環顔料] 
					\item[レーキ顔料] 染料を電離させ、担体となる金属イオンと電気的に結合させると不溶になる。これを不溶化またはレーキ化という。
				\end{description}
		\end{description}
	\section{染料と顔料の歴史}
		ラスコー洞窟の壁画。絵画は赤色と黄色の黄土、赤鉄鉱、二酸化マンガン、炭で描かれている。赤土・木炭を獣脂・血・樹液で溶かして混ぜ、黒・赤・黄・茶・褐色の顔料を作っていた。一部では遠近法も用いられている。\\
		人類最古の合成顔料とされているのは、鉛白とエジプシャンブルーである。鉛白は鉛と酢酸を混ぜ、二酸化炭素を加えて作る。エジプシャンブルーは紀元前2200年頃に作られた。カルシウム銅ケイ酸塩であり、藍銅鉱(アズライト)や孔雀石(マラカイト)といった銅鉱石と砂を混ぜた石灰石から作られる。陶器や偶像、ファラオの墓にも使われた。\\
		最初の鉱物顔料は青色顔料のウルトラマリンであり、6-7世紀のアフガニスタンの寺院の洞窟画に使われている。ウルトラマリンはラピスラズリ(瑠璃)から作られる。エジプト人たちも塗料として使おうと試みたが失敗した。輝度が高く劣化しにくいため大変貴重で、金と同等の価値を持っていたとされる。フェルメール(1632-1675)やレンブラントが好んで使っていた。\\
		コバルトブルーは8世紀頃から使われ始め、特に中国の磁器に使用された。1802年により純粋なものが得られるようになり商品化された。ルノワールやゴッホが高価なウルトラマリンの代わりに使用した。\\
		セルリアンブルーはラテン語で空色を意味するcaelumが語源である。\\
		インディゴ\\
		ネイビーブルー\\
		最初の青の合成顔料はプルシアンブルー(紺青)で、1704年にベルリンで発見された。日本では平賀源内が紹介した。清の商人が日本に転売したことで広まった。葛飾北斎(1760-1849)や歌川広重(1797-1858)などが使用した。プルシアンブルーは当時使われていた天然ウルトラマリンに取って代わり広く使われるようになった。\\
		1826年に合成ウルトラマリンが開発されてからは広く使われるようになった。\\
	\section{絵具}
		絵具は顕色材と展色材からなる。
		\begin{description}
			\item[顕色材] 発色成分。色素のことであり、普通は顔料が用いられる。
			\item[展色材] 色を定着させる。
				\begin{description}
					\item[固着材(バインダー)] 色素を固着させる。
					\item[溶剤] 溶媒。
				\end{description}
			\item[助剤] 乾燥促進剤や防腐剤など。
		\end{description}
		固着材には以下のようなものがある。
		\begin{description}
			\item[溶剤の蒸発で乾燥] 膠、アラビアゴム、アクアエマルションのアクリル樹脂
			\item[溶剤の蒸発と化学変化で乾燥] テンペラの卵黄
			\item[酸化重合で硬化] 単結合のみの油脂は直鎖上であるため融点が高く、二重結合を含む分子は折れ曲がり融点が低い。液体であった不飽和脂肪酸が酸化することで二重結合が解消され、融点が下がり固体となる。乾性油、アルキド樹脂。
			\item[熱可塑性で硬化] 冷えて固まる。蝋。
		\end{description}
		絵具には水性と油性がある。
		\begin{description}
			\item[水性絵具]
			\item[油性絵具]
		\end{description}
	\section{墨}
	\section{ブラウン管}
		ブラウン管は陰極線管(cathode-ray tube, CRT)を利用した映像装置であり、CRT管ともよばれる。テレビ受像機、コンピュータのディスプレイ、オシロスコープなどに使われている。真空管内で電子銃によって電子ビームを照射する。電子は高電圧によって加速され、蛍光物質を塗布した蛍光面に衝突することで発光する。電子は電界または磁界によって偏向される。走査方式には次のようなものがある。
		\begin{description}
			\item[ラスタスキャン] 画面を、垂直または水平方向に一列に並んだピクセル(走査線)ごとに分割して、線の端から端へと走査する。ビデオカメラで撮影すると上から下へと走査している様子が分かる。オシロスコープでは水平方向には一定速度で、垂直方向には電圧に応じて偏向する。
			\item[ベクタスキャン] 電子ビームやレーザーを直接図形の形状に沿って描画する。
			\item[ラジアルスキャン]
		\end{description}
		電界偏向の方が磁界偏向より高い周波数で走査でき、印加電圧を低くできるが、広い範囲での偏向には不向きである。またカラー画像の場合は、三種類の蛍光物質を使い、電子銃もRGBの三本ずつ用意する。\\
		ブラウン管を長時間使う続けると焼き付きが発生し画質が低下する。同じ画像を映し続けると蛍光面に跡が残る。問題が認識されてからは、こまめに全画面を書き換えるアトラクトと呼ばれる動画を流したり、スクリーンセーバーが利用されるようになった。\\
		ブラウン管は三極管の特性を持つため、電子ビームと発光強度の間に指数的な特性がある。
	\section{液晶ディスプレイ}

\repart{自然}
	\section{太陽}
		大気圏外の太陽光スペクトルは6000Kの黒体放射スペクトルとほぼ一致している。プランクの法則より、黒体から放射される電磁波の分光放射輝度は
			\[I = \frac{2hc^2}{\lambda^5}\rec{e^{hc/\lambda kT} - 1}\]
		で与えられる。
	\section{空}
		太陽が観測者の真上にあり、大気を高さ$h$の直方体とする。また、レイリー散乱を引き起こす微粒子は大気中に一様に分布しているとする。頭上から角$\theta$の地点にある微粒子から観測者の目に届く光の散乱角は$\theta$である。微粒子の数はその方向に大気を横切る線分の長さに比例するので、青空の輝度は
		\begin{align*}
			c \propto \frac{1 + \cos^2\theta}{2}\cdot \frac{h}{\cos\theta}\\
			= \frac{h}{2}\lr{\cos\theta + \rec{\cos\theta}}
		\end{align*}
		となる。$\theta$に依存する項は$\theta = 0$で最小値を取り、以降は単調に増加していく。つまり遠くの景色ほど明るく青みがかっていく。これが空気遠近法の原理である。
	\section{雲}
	\section{海}
		海の青は空の色の反射による効果もあるが、主に水自身の光の吸収によるものであることが分かっている。入射角が大きいほど反射率は高くなるので、地平線に近いほど空の色は濃くなる。
	\section{雨と雪}
	\section{植物}
\end{document}