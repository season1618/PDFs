\repart{色彩学}
	\section{標準光源}
		CIE(国際照明委員会)は色を測定するのに用いる証明光の取り決めを行っている。1931年に3種類の標準光源を導入し、後に付け加えられた。
		\begin{itemize}
			\item A:色温度2856K。白熱電球に相当。
			\item B:昼光(直接光)相当の光源
			\item C:昼光(間接光)相当の光源
			\item D65:色温度6504K
		\end{itemize}
		光源の色をある温度の黒体から放射される光の色と対応させたとき、その黒体の温度を色温度という。BとC光源はA光源に液状フィルターをかけて得られる光だが、紫外波長域の光を加えた$D_{65}$に取って代わられつつある。$D_{65}$は欧州・北欧における平均的な正午の光(直射日光と晴天の空による拡散光)に対応している。太陽表面におけるスペクトルは5800Kの黒体放射に近いが、X線は大気でほぼ遮断され、紫外線もオゾン層で90%以上カットされる。また可視光や赤外線は大気圏での反射・散乱・吸収により平均4割減衰する。太陽光のスペクトルには、ところどころ大気を構成する元素による吸収線(フラウンホーファー線)が見られる。したがって地上における太陽光は、大まかには紫外領域の波長が弱まった6504Kの黒体放射に相当する。
	\section{表色系}
		色を定量的に表す体系を表色系という。混色系は、色を心理物理量とみなし、色刺激の特性によって表す。顕色系は、色を三つの特徴に従って配列し、間隔を調整して尺度を示すものである。人間には三種類の錐体細胞が備わっているため、数学的には三つの変数で色を表現することができる。
		\subsection{LMS表色系}
			LMS色空間は最も自然な色空間であり、三つの錐体細胞の刺激値をもって色を表現する。LMS色空間は人の複雑な色覚をモデル化しているため、XYZとの単一の変換行列は存在しない。
		\subsection{RGB表色系}
			混色に用いる原色をR(700nm), G(546.1nm), B(435.8nm)の三つの単色光とする。コンピュータのディスプレイで最も多く用いられる。単色光と等色するのに必要な三原色の放射輝度を測定する実験が行われた。輝度の積分が1になるように正規化したものをRGB等色関数(color matching function)という。曲線が負の側に飛び出しているのは、三原色で等色できない単色光があることを示している。錐体細胞の感度曲線は微妙に重なっており、それはRGBの波長でも同じである。従って各錐体の感度を独立に調整することはできず、不可能な色が生まれてしまう。
		\subsection{XYZ表色系}
			RGBを用いる代わりに仮想の原刺激XYZを想定し、あらゆる色をこれらの正量混合で表せるようにしたのが、XYZ表色系である。しかしXYZとLMSは一致しない。人は明るい場所では、同じ強さの赤や青よりも緑の色を圧倒的に強く感じる。そのため波長ごとの明るさの感じ方である比視感度(分光視感効率)はM錐体の感度曲線と似ている。そこでYは輝度とほぼ一致するように設計されている。また白色点において$X = Y = Z$となるように決められた。XYZとRGBの線形変換は次の通りである。
			\begin{align*}
				\begin{bmatrix} X & Y & Z \end{bmatrix}
				=
				\begin{bmatrix}
					2.7689 & 1.7517 & 1.1302 \\
					1.0000 & 4.5907 & 0.0601 \\
					0.0000 & 0.0565 & 5.5943 \\
				\end{bmatrix}
				\begin{bmatrix} R & G & B \end{bmatrix}\\
				\begin{bmatrix} R & G & B \end{bmatrix}
				=
				\begin{bmatrix}
					0.4185 & -0.1587 & -0.0828 \\
					-0.0912 & 0.2524 & 0.0157 \\
					0.0000 & -0.0025 & 0.1786 \\
				\end{bmatrix}
				\begin{bmatrix} X & Y & Z \end{bmatrix}
			\end{align*}
			また分光放射輝度を$L(\lambda)$とすれば
			\begin{align*}
				X = \int L(\lambda)\overline{x}(\lambda)d\lambda\\
				Y = \int L(\lambda)\overline{y}(\lambda)d\lambda\\
				Z = \int L(\lambda)\overline{z}(\lambda)d\lambda
			\end{align*}
		\subsection{xyY表色系}
			XYZ表色系では色度が分かりにくい。そこで
				\[x = \frac{X}{X + Y + Z}, y = \frac{Y}{X + Y + Z}, z = \frac{Z}{X + Y + Z}\]
			として、xyによって表される色度Yの明るさで色を表す。
		\subsection{マンセル表色系}
			色彩を色の三属性(色相・明度・彩度)によって表す。
			\begin{description}
				\item[色相(Hue)]:色の様相。
				\item[明度(Value)]:色の明るさ。
				\item[彩度(Chroma)]:色の鮮やかさ。つまり分光組成の起伏具合である。正確には色空間における中央軸からの距離であり、距離の定義は色空間によって異なる。
			\end{description}
	\section{混色}
		混色には網膜内で起こる生理的混色と網膜の外で起こる物理的混色がある。前者は混合するほど明るくなるので加法混色といい、後者は混合するほど暗くなるので減法混色という。また絵具をパレット上で混ぜ合わせるような場合は、両方の過程が複雑に絡み合っているため着色剤混合という。
		\subsection{加法混色}
			複数の色刺激が網膜の同じ部位に入るとその分光組成は単純に加算されたものとして感じられる。加法混色の法則はグラスマンの法則としてまとめられている。
			\begin{description}
				\item[加算の法則]:混合色の輝度は成分色の輝度の和に等しい。
				\item[比例の法則]:成分色の輝度を同じ数だけ乗じても混合色の色相は変わらない。
				\item[結合の法則]:混合色は混色する順番に依らない。
			\end{description}
			グラスマンの法則は色空間がベクトル空間に似た性質を持っていることを示している。実際逆元の存在を除く全ての条件が満たされている。ここから
			\begin{enumerate}
				\item あらゆる色は互いに独立(どの二つを混ぜても三つ目の色を作れない)な三つの色の混合により得られる。
				\item 三色の混合比が等しければ色度は同じ。
			\end{enumerate}
			が導かれる。新印象派の点描画法は並置的加法混色の応用である。絵具を混ぜると暗くなってしまうため明るい色を表現するのに適さない。そこで純色をそのままキャンバスに載せて網膜上で混色させることで、明るさを損なうことなく色を表現できる。
		\subsection{減法混色}
			光源にフィルターをかけて色を作る場合を考える。二枚のフィルターを重ねたときの分光透過率は個々のフィルターの透過率の積に等しい。したがってこれは乗法混色とでもいうべきものだが、一般に透過率は1未満なので減法混色と呼ばれている。減法混色の三原色にはRGBの補色が用いられる。すなわちC(W - R, シアン)、M(W - G, マゼンタ)、Y(W - B, イエロー)である。このようにすると例えばマゼンタとイエローの混合は
				\[(W - G)(W - B) = W(W - G - B) + BG = R\]
			となる。
		\subsection{着色剤混合}
			透明ないし半透明な物体の場合、顔料の粒子の大きさが光の波長より十分大きければ入射光の一部は反射される。光は反射されるものと更に深層へと届くものに分かれ、二つの混色が複雑に絡み合った状態となる。顔料の混合などによる吸収や散乱を表すモデルとしてKubelka-Munk理論が知られている。