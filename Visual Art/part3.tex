\repart{メディア}
	\section{色素}
		着色に用いる粉末の内、水や油に溶けるものを染料、溶けないものを顔料という。
		染料
		\begin{description}
			\item[天然染料] 多くは動物や植物から抽出した色素である。
			\item[合成染料]
		\end{description}
		顔料
		\begin{description}
			\item[無機顔料] 遷移金属の金属錯体は配位子の影響を受けてd軌道が分裂/縮退することでエネルギー準位に差が生まれ発色する。
				\begin{description}
					\item[天然鉱物顔料] 
					\item[合成無機顔料] 
				\end{description}
			\item[有機顔料] 電子のエネルギー準位に相当する光の多くは紫外領域に属し、分子の電気双極子の振動に相当する光は赤外領域に属する。ところが共役系の非局在化電子は一般にエネルギー準位の差が小さく(井戸型ポテンシャルの下で波動関数を計算すると分かる)、光の吸収波長が長波長側へずれ呈色するものが多い。
				\begin{description}
					\item[アゾ顔料] 
					\item[多環顔料] 
					\item[レーキ顔料] 染料を電離させ、担体となる金属イオンと電気的に結合させると不溶になる。これを不溶化またはレーキ化という。
				\end{description}
		\end{description}
	\section{染料と顔料の歴史}
		ラスコー洞窟の壁画。絵画は赤色と黄色の黄土、赤鉄鉱、二酸化マンガン、炭で描かれている。赤土・木炭を獣脂・血・樹液で溶かして混ぜ、黒・赤・黄・茶・褐色の顔料を作っていた。一部では遠近法も用いられている。\\
		人類最古の合成顔料とされているのは、鉛白とエジプシャンブルーである。鉛白は鉛と酢酸を混ぜ、二酸化炭素を加えて作る。エジプシャンブルーは紀元前2200年頃に作られた。カルシウム銅ケイ酸塩であり、藍銅鉱(アズライト)や孔雀石(マラカイト)といった銅鉱石と砂を混ぜた石灰石から作られる。陶器や偶像、ファラオの墓にも使われた。\\
		最初の鉱物顔料は青色顔料のウルトラマリンであり、6-7世紀のアフガニスタンの寺院の洞窟画に使われている。ウルトラマリンはラピスラズリ(瑠璃)から作られる。エジプト人たちも塗料として使おうと試みたが失敗した。輝度が高く劣化しにくいため大変貴重で、金と同等の価値を持っていたとされる。フェルメール(1632-1675)やレンブラントが好んで使っていた。\\
		コバルトブルーは8世紀頃から使われ始め、特に中国の磁器に使用された。1802年により純粋なものが得られるようになり商品化された。ルノワールやゴッホが高価なウルトラマリンの代わりに使用した。\\
		セルリアンブルーはラテン語で空色を意味するcaelumが語源である。\\
		インディゴ\\
		ネイビーブルー\\
		最初の青の合成顔料はプルシアンブルー(紺青)で、1704年にベルリンで発見された。日本では平賀源内が紹介した。清の商人が日本に転売したことで広まった。葛飾北斎(1760-1849)や歌川広重(1797-1858)などが使用した。プルシアンブルーは当時使われていた天然ウルトラマリンに取って代わり広く使われるようになった。\\
		1826年に合成ウルトラマリンが開発されてからは広く使われるようになった。\\
	\section{絵具}
		絵具は顕色材と展色材からなる。
		\begin{description}
			\item[顕色材] 発色成分。色素のことであり、普通は顔料が用いられる。
			\item[展色材] 色を定着させる。
				\begin{description}
					\item[固着材(バインダー)] 色素を固着させる。
					\item[溶剤] 溶媒。
				\end{description}
			\item[助剤] 乾燥促進剤や防腐剤など。
		\end{description}
		固着材には以下のようなものがある。
		\begin{description}
			\item[溶剤の蒸発で乾燥] 膠、アラビアゴム、アクアエマルションのアクリル樹脂
			\item[溶剤の蒸発と化学変化で乾燥] テンペラの卵黄
			\item[酸化重合で硬化] 単結合のみの油脂は直鎖上であるため融点が高く、二重結合を含む分子は折れ曲がり融点が低い。液体であった不飽和脂肪酸が酸化することで二重結合が解消され、融点が下がり固体となる。乾性油、アルキド樹脂。
			\item[熱可塑性で硬化] 冷えて固まる。蝋。
		\end{description}
		絵具には水性と油性がある。
		\begin{description}
			\item[水性絵具]
			\item[油性絵具]
		\end{description}
	\section{墨}
	\section{ブラウン管}
		ブラウン管は陰極線管(cathode-ray tube, CRT)を利用した映像装置であり、CRT管ともよばれる。テレビ受像機、コンピュータのディスプレイ、オシロスコープなどに使われている。真空管内で電子銃によって電子ビームを照射する。電子は高電圧によって加速され、蛍光物質を塗布した蛍光面に衝突することで発光する。電子は電界または磁界によって偏向される。走査方式には次のようなものがある。
		\begin{description}
			\item[ラスタスキャン] 画面を、垂直または水平方向に一列に並んだピクセル(走査線)ごとに分割して、線の端から端へと走査する。ビデオカメラで撮影すると上から下へと走査している様子が分かる。オシロスコープでは水平方向には一定速度で、垂直方向には電圧に応じて偏向する。
			\item[ベクタスキャン] 電子ビームやレーザーを直接図形の形状に沿って描画する。
			\item[ラジアルスキャン]
		\end{description}
		電界偏向の方が磁界偏向より高い周波数で走査でき、印加電圧を低くできるが、広い範囲での偏向には不向きである。またカラー画像の場合は、三種類の蛍光物質を使い、電子銃もRGBの三本ずつ用意する。\\
		ブラウン管を長時間使う続けると焼き付きが発生し画質が低下する。同じ画像を映し続けると蛍光面に跡が残る。問題が認識されてからは、こまめに全画面を書き換えるアトラクトと呼ばれる動画を流したり、スクリーンセーバーが利用されるようになった。\\
		ブラウン管は三極管の特性を持つため、電子ビームと発光強度の間に指数的な特性がある。
	\section{液晶ディスプレイ}