\repart{西洋美術史}

\section{ルネサンス}
	古代ローマで用いられた線遠近法は中世の間に失われてしまった。ジョットやフランチェスカ。レオン・バッティスタ・アルベルティは遠近法を解説した『絵画論』(1435)を著した。その後遠近法をレオナルド・ダ・ヴィンチによって完成された。レオナルドは手稿の中で線的遠近法・色彩遠近法・消失遠近法の三つに分類している。色彩遠近法を遠くなるに連れて色彩が変化することを言い、消失遠近法は像が曖昧になることを意味する。これに加えてレオナルドは空気遠近法を発明し、遠景をより青く霞んだように描くことによって表現した。
\section{バロック}
	宗教改革によって生まれたプロテスタントでは絵画を含む偶像崇拝を禁止していた。そのため絵画の題材も以前のような歴史画や宗教画から身近な風景画や風俗画、静物画へと変わっていった。当時はスペインやそこから独立したオランダが繁栄しており、絵画も隆盛していた。
\section{17世紀オランダ}
	ヨハネス・フェルメールはカメラ・オブスキュラを用いて絵を描いたとされる。カメラ・オブスキュラを用いて絵を描くと焦点の合わない領域が生まれるので写真的になる。また色の恒常性が緩和されるため色彩を正確に捉えることができる。当時のレンズでは球面収差のために全体的に輪郭がぼやけている。
\section{印象派}
	ボードレールは一時的で主観的な美を主張した。
	クレメント・グリーンバーグはモダニズムという言葉を適用し、絵画は絵画以外に要素が排除され、形態や色彩などの純粋な要素が前面に表れてくると分析した。印象派から始まった平面性の追求は象徴主義やキュビスムなどを経て抽象絵画へと向かっていく。
	19世紀初頭に写真が発明され
	絵具は減法混色なので混ぜるほど画面が暗くなってしまっていた。そこで三原色及びそのうちの二色を混ぜた第一混色を主に使い、キャンバスに並置して網膜上で混合を行うことで明るさを損なうことなく色彩を表現することが出来た。

	ジョルジュ・スーラはストロークではなく点で描くことによって、筆触分割をより完璧にした点描画法を確立した。当時は視覚の生理学や色彩についての研究が進み、ミシェル=ウジェーヌ・シュヴルール『色彩の同時対照の法則』(1839)、シャルル・ブラン『デッサン芸術の文法』(1867)、ルード『近代色彩論』(1879)などの著作が出版された。スーラはこれらの著作を参考にして点描画法を研究した。このような傾向は新印象派と呼ばれる。点描画法の研究は大作\paint{グランド・ジャット島の日曜日の午後}(1886)に結実することになる。スーラは若くして亡くなってしまうが、その後はポール・シニャックがその理論を『ウジェーヌ・ドラクロワから新印象主義まで』(1899)にまとめた。
\section{キュビスム}
\section{シュールレアリスム}