\documentclass{jsarticle}
\usepackage{amssymb,amsmath}
\usepackage{mathrsfs,ascmac}

\newcommand{\repart}[1]{\part{#1}\setcounter{section}{0}}
\newcommand{\lr}[1]{\left(#1 \right)}
\newcommand{\mlr}[1]{\left\{#1 \right\}}
\newcommand{\llr}[1]{\left[#1 \right]}
\newcommand{\rec}[1]{\frac{1}{#1}}
\newcommand{\de}[2][]{\frac{d #1}{d #2}}
\newcommand{\dd}[2][]{\frac{d^2 #1}{d #2^2}}
\newcommand{\pd}[2][]{\frac{\partial #1}{\partial #2}}
\newcommand{\ppd}[3]{\frac{\partial^2 #1}{\partial #2\partial #3}}
\newcommand{\pmat}[1]{\begin{pmatrix} #1 \end{pmatrix}}
\newcommand{\co}[1][]{\cos^{#1}\theta}
\newcommand{\si}[1][]{\sin^{#1}\theta}
\newcommand{\grad}{\mathrm{grad}}
\newcommand{\dive}{\mathrm{div}}
\newcommand{\rot}{\mathrm{rot}}
\newcommand{\pa}{\partial}
\newcommand{\na}{\nabla}

\title{視覚芸術とメディア}
\author{season07001674}
\date{\today(初版 2020/05/19)}

\begin{document}
\maketitle
\tableofcontents

\input part1.tex % 視覚系と光学系 知覚の哲学、西洋美術史
\input part2.tex % 色彩学
\input part3.tex % メディア
\input part4.tex % 光学現象

\begin{thebibliography}{99}
	\item 千々岩英彰『色彩学概説』
	\item レオナルド・ダ・ヴィンチ『レオナルド・ダ・ヴィンチの手記(上・下)』
	\item 『視覚系の情報処理』
	\item 『色彩のメッセージ 三原色と補色の絵画史』
	\item 金子隆芳『色彩の科学』
	\item ジョナサン・クレーリー『観察者の系譜 視覚空間の変容とモダニティ』
	\item 『フェルメールと天才科学者 17世紀オランダの光と視覚の革命』
	\item 『絵画空間を考える』
	\item 
\end{thebibliography}

\end{document}