\repart{視覚系と光学系}
	\section{視覚系の構造と機能}
		光は角膜、虹彩の瞳孔、水晶体を通って網膜へと投射する。網膜は手前から順に神経節細胞、アマクリン細胞、双極細胞、水平細胞、視細胞(錐体・桿体)で構成されている。しかし光を受容するのは最も奥の視細胞であり、そこから手前に向かって信号が伝わる。網膜神経節細胞の軸索が伸びた視神経が視蓋前核を経由して外側膝状体や上丘へと繋がっている。これらの細胞と神経の結合の度合いは網膜の場所によって著しく異なる。中心部に比べて周辺部で生じた信号はかなり集約されて中枢へと伝わるので解像度は低くなる。\\
		虹彩は光の強さによって瞳孔の直径を調節する。これを対光反射という。視蓋前核からの軸索は動眼神経副核に接続し、動眼神経副核の軸索は左右の動眼神経に接続する。動眼神経の副交感神経が毛様体神経節細胞と繋がり、虹彩の毛様体を収縮させる。環境光と瞳孔の大きさの関係は非線形微分方程式で表せる。\\
		視細胞には視物質(ロドプシン、フォトプシン)が蓄えられている。視物質は光を吸収することで化学反応を起こし、細胞膜のイオンチャネルを開閉させることでイオン電流を発生させる。錐体にはL錐体、M錐体、S錐体の三種類あり、それぞれ$564, 534, 420nm$をピークに持っている。錐体は波長によって感度が異なり、それによって色を感じ取ることができる。桿体は錐体に比べて圧倒的に感度が高く、一個の光子に対しても反応することができる。\\
		水平細胞は双極細胞と視細胞のシナプスの間を水平に繋いでおり、視細胞が活動するとき周囲の双極細胞の活動を抑制する機能を持つ。これを側抑制という。隣接する二つの視細胞に異なる強度の光が入ってきたとき、強度の強い方がより周囲の双極細胞を抑制する。つまりヒトが感じる明暗の差は実際の刺激の差よりも大きくなる。これが明るさ対比や色対比が生じるメカニズムである。\\
		このようにヒトの網膜では単純な処理のみを行っているが、中枢神経系の未発達な動物では運動物体の検出などの高度な情報処理も網膜で行っている。硝子体の内部や網膜の前面には毛細血管や神経線維が張り巡らされており、常時網膜に陰を落としている。視細胞や脳はそれら網膜上で動かない物体を無視するなどの補正を行っているが、特定の条件下では適切な処理が出来ず、飛蚊症やブルーフィールド内視現象などの原因となっている。\\
		一つの波長のみからなる光を単色光というが、全ての色が単色光で表せるわけではない。例えば白はスペクトル上には存在しない。また特定の色を表現する単色光の組合せも一意ではない。赤と青の混合した光も紫の単色光も同じ色に見える。この現象をメタメリズムという。言い換えればスペクトルからLMSを返す関数は単射ではない。\\
		錐体の波長に対する相対的な感度のグラフを見てみると、錐体を一つだけ刺激するような単色光がないことが分かる。これは純粋な原色は物理的に存在しないということを示している。例えば光を用いてL錐体だけを刺激することはできない。コンピュータに表示されている$(255, 0, 0)$の色も純粋な赤ではない。論理的には可視光を赤、緑、青に分解することはできるが、それぞれを任意の割合で混ぜ合わせることはできない。\\
		\subsection{明るさ}
			物理的な明るさが同じでも感覚的な明るさは波長によって異なる。緑の光は同じ明るさの赤や青に比べて明るく感じる。また暗所と明所における比視感度には少しずれがある。暗所では青が見えやすく、明所では赤が見えやすい。これをプルキンエ現象という。
		\subsection{刺激時間}
			一定強度の光で目を刺激をすると、反応するまでに0.05-0.2秒かかる。これを感覚化時間という。その後反応は低下し、刺激が中断してから0.05-0.2秒は持続する。これを残効時間という。ピーク時の反応は低下後の2倍程度である。反応が生じる速さと大きさはほぼ刺激の強度に依存する。そして感覚がピークに達する時間は波長によっても異なり、長波長の方が短波長より素早い。信号に赤が使われている理由の一種である。\\
			ベンハムの独楽。
	\section{カメラ}
		カメラの撮像素子は普通光電効果により光を電気に変換する。
	\section{画像処理}