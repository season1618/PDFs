\repart{視覚系と光学系}
	\section{視覚系の構造と機能}
		光は角膜、虹彩の瞳孔、水晶体を通って網膜へと投射する。網膜は手前から順に神経節細胞、アマクリン細胞、双極細胞、水平細胞、視細胞(錐体・桿体)で構成されている。しかし光を受容するのは最も奥の視細胞であり、そこから手前に向かって信号が伝わる。網膜神経節細胞の軸索が伸びた視神経が視蓋前核を経由して外側膝状体や上丘へと繋がっている。これらの細胞と神経の結合の度合いは網膜の場所によって著しく異なる。中心部に比べて周辺部で生じた信号はかなり集約されて中枢へと伝わるので解像度は低くなる。\\
		虹彩は光の強さによって瞳孔の直径を調節する。これを対光反射という。視蓋前核からの軸索は動眼神経副核に接続し、動眼神経副核の軸索は左右の動眼神経に接続する。動眼神経の副交感神経が毛様体神経節細胞と繋がり、虹彩の毛様体を収縮させる。環境光と瞳孔の大きさの関係は非線形微分方程式で表せる。\\
		視細胞には視物質(ロドプシン、フォトプシン)が蓄えられている。視物質は光を吸収することで化学反応を起こし、細胞膜のイオンチャネルを開閉させることでイオン電流を発生させる。錐体にはL錐体、M錐体、S錐体の三種類あり、それぞれ$564, 534, 420nm$をピークに持っている。錐体は波長によって感度が異なり、それによって色を感じ取ることができる。桿体は錐体に比べて圧倒的に感度が高く、一個の光子に対しても反応することができる。\\
		水平細胞は双極細胞と視細胞のシナプスの間を水平に繋いでおり、視細胞が活動するとき周囲の双極細胞の活動を抑制する機能を持つ。これを側抑制という。隣接する二つの視細胞に異なる強度の光が入ってきたとき、強度の強い方がより周囲の双極細胞を抑制する。つまりヒトが感じる明暗の差は実際の刺激の差よりも大きくなる。これが明るさ対比や色対比が生じるメカニズムである。\\
		このようにヒトの網膜では単純な処理のみを行っているが、中枢神経系の未発達な動物では運動物体の検出などの高度な情報処理も網膜で行っている。硝子体の内部や網膜の前面には毛細血管や神経線維が張り巡らされており、常時網膜に陰を落としている。視細胞や脳はそれら網膜上で動かない物体を無視するなどの補正を行っているが、特定の条件下では適切な処理が出来ず、飛蚊症やブルーフィールド内視現象などの原因となっている。\\
		一つの波長のみからなる光を単色光というが、全ての色が単色光で表せるわけではない。例えば白はスペクトル上には存在しない。また特定の色を表現する単色光の組合せも一意ではない。赤と青の混合した光も紫の単色光も同じ色に見える。この現象をメタメリズムという。言い換えればスペクトルからLMSを返す関数は単射ではない。\\
		錐体の波長に対する相対的な感度のグラフを見てみると、錐体を一つだけ刺激するような単色光がないことが分かる。これは純粋な原色は物理的に存在しないということを示している。例えば光を用いてL錐体だけを刺激することはできない。コンピュータに表示されている$(255, 0, 0)$の色も純粋な赤ではない。論理的には可視光を赤、緑、青に分解することはできるが、それぞれを任意の割合で混ぜ合わせることはできない。\\
		\subsection{明るさ}
			物理的な明るさが同じでも感覚的な明るさは波長によって異なる。緑の光は同じ明るさの赤や青に比べて明るく感じる。また暗所と明所における比視感度には少しずれがある。暗所では青が見えやすく、明所では赤が見えやすい。これをプルキンエ現象という。
		\subsection{刺激時間}
			一定強度の光で目を刺激をすると、反応するまでに0.05-0.2秒かかる。これを感覚化時間という。その後反応は低下し、刺激が中断してから0.05-0.2秒は持続する。これを残効時間という。ピーク時の反応は低下後の2倍程度である。反応が生じる速さと大きさはほぼ刺激の強度に依存する。そして感覚がピークに達する時間は波長によっても異なり、長波長の方が短波長より素早い。信号に赤が使われている理由の一種である。\\
			ベンハムの独楽。
	\section{色覚}
		\subsection{ニュートン『光学』}
			アイザック・ニュートン(1642-1727)は太陽光をプリズムによって分光する実験を行い、太陽の光が様々な波長の光から構成されることを示した。ニュートンは光線とそれに対応する色を区別していた。ニュートンはそれぞれの波長の光を認識する無限個の細胞があると考えていた。ニュートンは音楽的調和から虹が七色であることにこだわった。『光学』の中で色を円環状に配置した図を載せている。
		\subsection{ヤング=ヘルムホルツの三色説}
			絵具を混ぜると色が変わることは以前から知られていた。特定の色を基本色として据え、それ以外の色を全て基本色から生み出すような組合せを探すのは当然の試みだったに違いない。アリストテレスやアルベルティ、ダ・ヴィンチもそういった基本色を考えている。かくして17世紀には色の三原色は赤・黄・青との見解が一致していた。また絵画の制作にも応用された。\\
			トーマス・ヤング(1773-1829)は絵具の混色にヒントを得て、人間の目には赤・黄・青に対応する光を認識する三種類の網膜物質があると考えた。それぞれの物質の感じるスペクトルには幅があり、また光が目に入ることで網膜物質と共鳴を起こすとした。これを三色説という。ヤングは翌年に見つかった太陽スペクトルの暗線(後のフラウンホーファー線)を誤って解釈し、三原色を赤・緑・青と修正したが、これは誤謬である。それ以降も光の混色と色の混色は混同され続けた。\\
			ヘルマン・フォン・ヘルムホルツ(1821-1894)はヤングの三色説について定量的な実験を行った。ヘルムホルツは混色には加法混色と減法混色があることに気付き、光の三原色を赤・緑・青と正しく訂正した。1868年には各錐体の分光感度を求めている。\\
			1956年、Gunnar Svaetichinは魚の網膜が三つの異なる波長域に特異的に反応することを発見した。これは三色説の最初の生物学的な裏付けであった。\\
		\subsection{へリングの反対色説}
			エヴァルト・へリング(1834-1918)は白・黒・赤・緑・黄・青を純色とした。そして白-黒、赤-緑、黄-青がそれぞれ相補的な役割を担い、三種類の網膜物質の合成と分解によって発生すると考えた。これを反対色説という。反対色説は色の対比や残像現象を上手く説明することができる。
		\subsection{アダムスの段階説}
			1923年、アダムスは三色説と反対色説の神経回路の変換モデルを考えた。シュレーディンガーも三色説と反対色説の色の次元はどちらも3であり矛盾なく成立することを指摘した。\\
			眼の構造を振り返ってみると、錐体と桿体が三色説の三つの刺激を感知し、その前面の双極細胞が反対色の反応を示すことが分かった。
	\section{感覚生理学と認識論}
		17世紀、認識の起源について主に二つの立場があった。経験論とは認識の起源を経験に求めるものである。経験論の哲学者にはジョン・ロック、ジョージ・バークリー、デイビット・ヒュームなどがいる。経験論の最も極端なものは、人間は生まれたときにはタブラ・ラサ(白紙状態)であり、あらゆる認識は経験によって得られるとするものである。合理論とは、認識の起源を理性に求め、外界からの知覚なしに思考することができるというものである。\\
		イマヌエル・カント(1724-1804)は人間の認識には元々備わっている先験的なものと経験から得られるものがあるとする折衷案を提唱し、経験論と合理論を統一した。カントは認識を洞察するに当たって、現象と物自体という概念を導入した。物自体は感覚を生み出す原因となるものであり、人間の主観とは無関係に存在する。現象は主観と物自体の作用によって生まれる。\\
		ヨハネス・ミュラー(1801-1858)は特殊神経エネルギー説を提唱した。神経を伝わるのは光や音とは別の神経特有のエネルギーであり、特定の感覚神経は特定の感覚体験しか引き起こさない。人間の持つこれら感覚神経はカントの言う先験的な認識形式とされる。
		ヘルマン・フォン・ヘルムホルツはカントの生理学的な再解釈を行った。これらの運動は新カント派と呼ばれる。
	\section{カメラ}
		カメラの撮像素子は普通光電効果により光を電気に変換する。
	\section{画像処理}